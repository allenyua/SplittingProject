
The complex cobordism spectrum $MU$ has played a central role in homotopy theory ever since Quillen connected its homotopy with the theory of one-dimensional formal group laws.  As the Thom spectrum of the canonical bundle over $BU$, $MU$ naturally acquires an immense amount of structure: it is an $\mathbb{E}_\infty$-ring spectrum.  To this day, however, much remains unknown about the full nature of this $\mathbb{E}_\infty$ structure.  For example, almost nothing is known about the k-invariants of the spectrum of units $gl_1(MU)$.

As a spectrum, $MU$ is classically constructed as the direct limit of the sequence
$$\mathbb{S}=MU(0) \longrightarrow \Sigma^{-2} MU(1) \longrightarrow \cdots \longrightarrow \Sigma^{-2n} MU(n) \longrightarrow \cdots,$$
where $MU(n)$ is the Thom spectrum of the canonical bundle over $BU(n)$.

This suggests that perhaps the more elemental object is $$\bigvee MU(n),$$ the Thom spectrum of the $J$-homomorphism
$$\textbf{Vect} \stackrel{J}{\longrightarrow} \text{Pic}(\mathbb{S})$$
that takes a vector space $V$ to its one-point compactification $J(V)=S^V$.  Since $$J(V \oplus W) \simeq S^V \smsh S^W,$$ the Thom spectrum $\bigvee MU(n)$
is naturally an $\mathbb{E}_\infty$-ring spectrum.  Inverting the Bott element $\beta \in \pi_2(MU(1)) \cong \pi_2(\mathbb{CP}^{\infty})$, one obtains the \textit{periodic} complex cobordism spectrum
$$MUP \simeq \left(\bigvee MU(n) \right)[\beta^{-1}].$$

This periodic spectrum $MUP$ is a minor variation on $MU$ itself--there is a wedge decomposition $$MUP \simeq \bigvee_{a \in \mathbb{Z}} \Sigma^{2a} MU,$$ and the inclusion $MU \rightarrow MUP$ onto the $a=0$ factor is an an $\mathbb{E}_\infty$-ring homomorphism.

In 1979, Victor Snaith \cite{SnaithBook} gave a second, and fundamentally different, presentation of periodic complex cobordism:

\begin{thm}[Snaith] \label{SnaithSplitting}
As homotopy commutative ring spectra, $$MUP \simeq \Sigma^{\infty}_+ BU[\beta^{-1}].$$  More generally, there is an equivalence of homotopy commutative rings
$$\bigvee MU(n) \simeq \Sigma^{\infty}_+ BU.$$
\end{thm}

\begin{rmk} In the above, $\Sigma^{\infty}_+ BU$ acquires an $\mathbb{E}_\infty$-structure (and hence a homotopy commutative ring structure) from the fact that $BU$ is an infinite loop space.  One may think of $\Sigma^{\infty}_+ BU$ as the group ring of the topological group $BU$.
\end{rmk}

The genesis of this paper was an attempt to use Snaith's theorem to study the $\mathbb{E}_\infty$-ring structure on $MUP$.  This seemed like an especially reasonable idea in light of two facts:

\begin{enumerate}
\item By another theorem of Snaith, periodic $K$-theory $KU$ may be constructed as $$KU \simeq \Sigma^{\infty}_+ \mathbb{CP}^{\infty}[\beta^{-1}].$$  This is an equivalence of $\mathbb{E}_\infty$-ring spectra \textbf{CITE}.
\item In \cite{GepnerSnaith}, Gepner and Snaith prove a motivic analogue of Theorem \ref{SnaithSplitting}.  In particular, they prove an equivalence 
$$\Sigma^{\infty}_+ BGL[\beta^{-1}] \simeq PMGL.$$
They then use the natural $\mathbb{E}_\infty$-ring structure on $\Sigma^{\infty}_+ BGL[\beta^{-1}]$ to \textbf{define} an $\mathbb{E}_\infty$-ring structure on $PMGL$. 
\end{enumerate}

\textbf{Allen, there is something I am confused about.  In Gepner-Snaith, they claim that there is an $\mathbb{E}_\infty$ ring homomorphism from $PMGL$ to $KU$ given by inverting $\beta$ in the suspension of the determinant map
$$BU \rightarrow \mathbb{CP}^{\infty}.$$
I thought though that we decided there is no $\mathbb{E}_\infty$-ring map to $H\mathbb{Z}P$.  What's up with that?}

However, as it turns out (though it is not at all obvious from the modern literature and in particular not mentioned in \cite{GepnerSnaith}), another old theorem of Snaith \cite{SnaithNotMultiplicative} shows that our idea was entirely unreasonable:

\begin{thm}[Snaith]
As $\mathbb{E}_\infty$-rings,
$$MUP \not \simeq \Sigma^{\infty}_+ BU[\beta^{-1}].$$
\end{thm}

In the final Section \ref{sec:SnaithSplitting} of this paper, we refine Snaith's results into what we consider their definitive form:

\begin{thm}
There is an equivalence of $\mathbb{E}_2$-ring spectra
$$MUP \simeq \Sigma^{\infty}_+ BU[\beta^{-1}],$$
but $MUP \not \simeq \Sigma^{\infty}_+ BU[\beta^{-1}]$ as $\mathbb{E}_3$-ring spectra.  There is an equivalence of $\mathbb{A}_\infty$-ring spectra
$$\bigvee MU(n) \simeq \Sigma^{\infty}_+ BU,$$
but $\bigvee MU(n) \not \simeq \Sigma^{\infty}_+ BU$ as $\mathbb{E}_2$-ring spectra.
\end{thm}

With the focus now on $\mathbb{E}_2$-algebras, it is natural to view Snaith's splitting result as the limiting case of a sequence of other stable splittings.  Consider the filtration
$$* \simeq \Omega SU(1) \longrightarrow \Omega SU(2) \longrightarrow \cdots \longrightarrow \Omega SU(n) \longrightarrow \cdots \longrightarrow \Omega SU \simeq BU,$$
where the last equivalence is by Bott periodicity.  Taking suspension spectra, we obtain a filtration of $\mathbb{E}_2$-ring spectra
$$\mathbb{S} \longrightarrow \Sigma^{\infty}_+ \Omega SU(2) \longrightarrow \cdots \longrightarrow \Sigma^{\infty}_+ \Omega SU(n) \longrightarrow \cdots \longrightarrow \Sigma^{\infty}_+ BU.$$

It is a theorem of Crabb and Mitchell \cite{CrabbMitchell} that, for $n>1$, $\Sigma^{\infty}_+ \Omega SU(n)$ splits as an infinite wedge sum.  We study the coherence of their splitting in Sections \ref{sec:AooSplit} and \textbf{BLAH}:

\begin{thm}
The Crabb--Mitchell stable splitting of $\Sigma^{\infty}_+ \Omega SU(n)$ is a splitting of $\mathbb{A}_\infty$-ring spectra, but not of $\mathbb{E}_2$-ring spectra.
\end{thm}

To be precise about what we mean by a splitting of $\mathbb{E}_n$-ring spectra, we need to introduce a bit of abstract terminology.

\begin{rmk}
We freely use the language of $\infty$-categories throughout this paper, referring to an $\infty$-category simply as a category.
\end{rmk}

%I stole a bunch of the stuff written here, so maybe it can be omitted or simplified.  
In Section \ref{sec:FilGra} we will review the symmetric monoidal categories $\Fil$ and $\Gr$ of filtered and graded spectra, respectively.  A filtered spectrum is an infinite sequence
$$X_0 \longrightarrow X_1 \longrightarrow X_2 \longrightarrow X_3 \longrightarrow \cdots$$
of spectra.  The tensor product $$\left(X_0 \longrightarrow X_1 \longrightarrow X_2 \longrightarrow \cdots \right) \otimes \left(Y_0 \longrightarrow Y_1 \longrightarrow Y_2 \longrightarrow \cdots \right)$$
of two filtered spectra is computed as a Day convolution

\begin{center}
$X_0 \otimes Y_0 \longrightarrow \colim $
\adjustbox{scale=0.7} 
{$ \left(\begin{tikzcd} X_0 \smsh Y_1 \\  X_0 \smsh Y_0 \arrow{u} \arrow{r} & X_1 \smsh Y_0 \end{tikzcd} \right) $} 
$\longrightarrow \colim$
\adjustbox{scale=0.7} {$ \left( \begin{tikzcd} X_0 \smsh Y_2 \\ X_0 \smsh Y_1 \arrow{r} \arrow{u} & X_1 \smsh Y_1  \\ X_0 \smsh Y_0 \arrow{r} \arrow{u} & X_1 \smsh Y_0 \arrow{u} \arrow{r} & X_2 \smsh Y_0 \end{tikzcd} \right) $}
$\longrightarrow \cdots.$
\end{center}

A graded spectrum, on the other hand, is simply an ordered sequence $(A_0,A_1,A_2, \cdots)$ of spectra.  The tensor product  is computed as
$$(A_0,A_1,A_2,\cdots) \otimes (B_0,B_1,B_2,\cdots) \simeq \left( A_0 \smsh B_0, (A_1 \smsh B_0) \vee (A_0 \smsh B_1), \cdots, \bigvee_{i+j=n} A_i \smsh B_j, \cdots \right).$$

There is a sequence of symmetric monoidal functors
$$
\begin{tikzcd}[column sep=huge]
\Gr \arrow{r}{I} & \Fil \arrow{r}{\text{colim}} & \Sp,
\end{tikzcd}
$$
where $I$ sends the graded spectrum $(A_0,A_1,A_2,\cdots)$ to the filtered spectrum
$$
I(A_0,A_1,A_2,\cdots) = \left( A_0 \longrightarrow A_0 \vee A_1 \longrightarrow A_0 \vee A_1 \vee A_2 \longrightarrow \cdots\right).
$$

\begin{dfn}
We say that an $\mathbb{E}_n$-ring spectrum is $\mathbb{E}_n$-\textbf{split} if it is equivalent to the image of an $\mathbb{E}_n$-algebra in $\Gr$ under the composite $\text{colim} \circ I.$  Similarly, a filtered $\mathbb{E}_n$-algebra is $\mathbb{E}_n$-split if it is equivalent to the image under $I$ of an $\mathbb{E}_n$-algebra in $\Gr$.
\end{dfn}

\textbf{BLAH}

In section \ref{sec:MUE2} we will show:
\begin{thm}
The ??? filtration on $\Sigma^{\infty}_+ \Omega SU(n)$ is $\mathbb{E}_2$-split after smashing with $MU$.
\end{thm}

\textbf{BLAH}
%acknowledge
%1. Jacob, Mike
%2. arone? akhil, denis?, dyang?, 
%3. Arpon if he reads it,

%Notations, todo?
%\Sp for spectra, S for spaces? should also say that by default, these are given \smash, and \times.
