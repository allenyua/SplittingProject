We study the homotopy type of the affine Grassmannian of $SL_n(\mathbb{C})$, which is equivalent to the space $\Omega SU(n)$ of based loops in $SU(n)$.  There are essentially two multiplications on this homotopy type, one arising from the composition of loops and the other from the group multiplication on $SL_n(\mathbb{C})$.  Together, these two multiplications interact to give $\Omega SU(n)$ the structure of an $\mathbb{E}_2$ or fusion algebra.  In geometric representation theory, this structure is witnessed by the existence of the Beilinson--Drinfeld Grassmannian.

Either of the above (homotopy equivalent) products make $H_*(\Omega SU(n);\mathbb{Z})$ into a graded ring.  To describe this ring, let us first name some of its elements.  For each one-dimensional subspace $V \subset \mathbb{C}^n$, there is a loop $\lambda_V:S^1 \rightarrow U(n)$ given by the formula
$$\lambda_V(z)=\left( \begin{array}{cc} z & 0 \\ 0 & I \end{array} \right),$$
with the matrix presented in terms of the decomposition $\mathbb{C}^n \cong V \oplus V^{\perp}$.  Fixing a particular line $W \subset \mathbb{C}^n$, the construction $V \mapsto \lambda_W^{-1} \cdot \lambda_V$ defines a well-known map
$$\mathbb{CP}^{n-1} \rightarrow \Omega SU(n).$$
For $1 \le i \le n-1$, let $b_i \in H_{2i}(\Omega SU(n);\mathbb{Z})$ denote the image of the generator of $H_{2i}(\mathbb{CP}^{n-1};\mathbb{Z})$.  It is a result of Bott \cite{Bott} that
$$H_*(\Omega SU(n);\mathbb{Z}) \cong \mathbb{Z}[b_1,b_2,\cdots,b_{n-1}],$$
with the latter denoting the polynomial algebra on the classes $b_i$.

Notice, in particular, that $H_*(\Omega SU(n);\mathbb{Z})$ is a \textit{bigraded} ring: one grading is given by $*$, and the other by word length (assigning each $b_i$ degree $1$).  Mahowald observed that the action of the Steenrod algebra on $H_*(\Omega SU(n);\mathbb{F}_2)$ preserves word length, and he conjectured a geometric splitting to be responsible.  Indeed, it was eventually proven by Mitchell and Richter \cite[Theorem 2.1]{CrabbMitchell} that the suspension spectrum $\Sigma^{\infty} \Omega SU(n)$ splits as an infinite wedge sum:
$$\Sigma^{\infty}_+ \Omega SU(n) \simeq \mathbb{S} \vee \Sigma^{\infty} \mathbb{CP}^{n-1} \vee \cdots.$$

In order to prove this splitting, Mitchell \cite{MitchellSU(n)} (and, independently, Segal \cite{Segal}) first constructed a filtration of the space $\Omega SU(n)$.  Following Mitchell, we name this the \textit{Bott filtration} of $\Omega SU(n)$.  The first filtered piece is given by the above map $\mathbb{CP}^{n-1} \rightarrow \Omega SU(n)$, and the theorem of Mitchell and Richter is that the filtration stably splits.  The construction of the Bott filtration is somewhat involved, and we review it in Section \ref{sec:MRFil}--it is a subfiltration of the Bruhat ordering on (closures of) Iwahori orbits.

\begin{exm}
In the case $n=2$, the Bott filtration of $\Omega SU(2) \simeq \Omega S^3$ is the classical James filtration of $\Omega \Sigma S^2$.
\end{exm}

In Section \ref{sec:FilGra}, we review the symmetric monoidal structures on the ($\infty$)-categories of filtered and graded spectra; they are given by Day convolution.  This allows us to talk about $\E_n$-algebras in filtered and graded spectra, providing the language necessary to state our first main theorem (proven in Section \ref{sec:MRFil}):

\begin{thm} \label{thm:BottIsAoo}
The suspension of the Bott filtration 
$$\mathbb{S} \longrightarrow \Sigma_+^{\infty} \mathbb{CP}^{n-1} \simeq \Sigma_+^{\infty} F_{n,1} \longrightarrow \Sigma_+^{\infty} F_{n,2} \longrightarrow \cdots \longrightarrow \Sigma^{\infty}_+ \Omega SU(n).$$
is an $\mathbb{A}_\infty$-algebra object in filtered spectra.
\end{thm}

\begin{rmk}
The Bott filtration is multiplicative before suspension, but for technical reasons we prefer to phrase our results in terms of filtered spectra instead of filtered spaces.
\end{rmk}

\begin{qst} \label{qst:BottE2}
Is the Bott filtration an $\mathbb{E}_2$ filtration?  We do not know the answer--for some thoughts about the problem, see Remark \ref{rmk:E2fil}.
\end{qst}

The proof of Theorem \ref{thm:BottIsAoo} is fairly straightforward, once given access to the sophisticated machinery behind the Beilinson--Drinfeld Grassmannian.  For example, we will explain in Section \ref{sec:MRFil} that this machinery immediately dispenses with a conjecture of Mahowald and Richter \cite{MahowaldRichter}.  Nonetheless, there are some subtleties involved, and it is these subtleties that prevent us from determining if the Bott filtration is $\mathbb{E}_2$.  The problem is readily visible in the case $n=\infty$:

\begin{exm}\label{exm:BottFil}
The limiting case of the Bott filtration of $\Omega SU(n)$ as $n$ tends to $\infty$ is the filtration
$$* \longrightarrow BU(1) \longrightarrow BU(2) \longrightarrow BU(3) \longrightarrow \cdots \longrightarrow BU \simeq \Omega SU.$$
It is easy to see that $\coprod BU(n)$ is a graded $\mathbb{E}_2$-algebra in spaces (in fact, it is a graded $\mathbb{E}_\infty$-algebra, being the nerve of the category of vector spaces).  However, the filtered object is much more subtle.  For example, the squares
$$
\begin{tikzcd}
BU(i) \times BU(j) \arrow{d} \arrow{r} & BU(i) \times BU(j+1) \arrow{d} \\
BU(i+1) \times BU(j) \arrow{r} & BU(i+1) \times BU(j+1)
\end{tikzcd}
$$
do not commute on the nose, but only up to non-canonical homotopy.
\end{exm}

In Section \ref{sec:FilGra}, we discuss an \textit{associated graded} construction that transforms filtered $\mathbb{E}_n$-algebras into graded $\mathbb{E}_n$-algebras.  Combined with Theorem \ref{thm:BottIsAoo}, this endows the associated graded of the stable Bott filtration with a graded $\mathbb{A}_\infty$-algebra structure.

\begin{exm}
In the situation of Example \ref{exm:BottFil}, the associated graded of the stable Bott filtration is given by
$$\bigvee_n \Sigma^{\infty} BU(n+1)/BU(n) \simeq \bigvee_n MU(n),$$
where $MU(n)$ is the Thom spectrum of the canonical bundle over $BU(n)$.  In this case, the Mitchell--Richter splitting recovers an older splitting due to Snaith \cite{SnaithBook}
\begin{equation}\label{eqn:MunSplitting}
\Sigma^{\infty}_+ BU \simeq \bigvee_n MU(n). %it feels awkward to not mention the total Chern class and Snaith's MUP construction, now that we're here
\end{equation}
The left-hand side of (\ref{eqn:MunSplitting}) is an $\mathbb{A}_\infty$-algebra by virtue of the fact that $BU$ is a loop space.  On the other hand, the right-hand side obtains its $\mathbb{A}_\infty$-algebra structure from the Bott filtration.  As it turns out, this latter $\mathbb{A}_\infty$-structure essentially arises from the Thom construction applied to the $J$-homomorphism
$$\coprod BU(n) \stackrel{J}{\longrightarrow} Pic(\mathbb{S}).$$
In other words, the right-hand side of (\ref{eqn:MunSplitting}) admits an $\mathbb{A}_\infty$-algebra structure by virtue of the fact that the $J$ homomorphism is a loop map.

Of course, $BU$ is not just a loop space, but in fact an infinite loop space.  Similarly, $J$ is not just a loop map, but furthermore an infinite loop map.  Thus, both sides of (\ref{eqn:MunSplitting}) are naturally $\mathbb{E}_\infty$-ring spectra.  Perhaps surprisingly, these $\mathbb{E}_\infty$-rings are \textbf{not} equivalent.
\end{exm}

Though we do not know if the Bott filtration is $\mathbb{E}_2$, in the above example we find a natural graded $\mathbb{E}_2$-algebra structure on its associated graded.  We sketch, at the end of Section \ref{sec:MRFil}, a construction showing that something similar always happens:

\begin{cnstr} \label{cnstr:IntroGr}
Let $\text{gr}(\Sigma^{\infty}_+\{F_{n,k}\})$ denote the associated graded of the Bott filtration of $\Sigma^{\infty}_+ \Omega SU(n)$.  There exists a graded $\mathbb{E}_2$-algebra structure on the graded spectrum $\text{gr}(\Sigma^{\infty}_+ \{F_{n,k}\})$ that extends the canonical graded $\mathbb{A}_\infty$-algebra structure.
\end{cnstr}

The remainder of the paper is concerned with the stable splitting of the Bott filtration.  Our main results are as follows:

\begin{thm} \label{thm:MainAoo}
As an $\mathbb{A}_\infty$-algebra object in filtered spectra, the Bott filtration of $\Sigma^{\infty}_+ \Omega SU(n)$ is equivalent to its associated graded.
\end{thm}

\begin{cor}
For any multiplicative homology theory $E$, $E_*(\Omega SU(n))$ is a bigraded ring.  One grading is given by $*$, and the other by the associated graded of the Bott filtration.
\end{cor}

\begin{thm} \label{thm:MainObstruction}
Suppose $n \ge 4$.  If the Bott filtration of $\Sigma^{\infty}_+ \Omega SU(n)$ may be made into an $\mathbb{E}_2$-algebra object in filtered spectra, then it is \textbf{not} equivalent to its $\mathbb{E}_2$ associated graded.  More generally, any extension of the graded $\mathbb{A}_\infty$-algebra of Theorem \ref{thm:MainAoo} to a graded $\mathbb{E}_2$-algebra must fail to have the usual $\mathbb{E}_2$-algebra structure on its underlying ungraded $\mathbb{E}_2$-ring.
\end{thm}

\begin{thm} \label{thm:MainMUE2}
Let $MU$ denote the $\mathbb{E}_\infty$-ring spectrum of complex bordism.  For any $\mathbb{E}_2$-algebra structure on the underlying (ungraded) $\mathbb{A}_\infty$-ring $\text{gr}(\Sigma^{\infty}_+\{F_{n,k}\})$, there is an equivalence of $\mathbb{E}_2$-$MU$-algebras
$$MU \smsh \Sigma^{\infty}_+ \Omega SU(n) \simeq MU \smsh \text{gr}(\Sigma^{\infty}_+\{F_{n,k}\}).$$
\end{thm}

\begin{rmk}
It is Construction \ref{cnstr:IntroGr} that gives Theorem \ref{thm:MainMUE2} its teeth.  The result may be seen as a once-looped analogue of work of Kitchloo \cite{Kitchloo}.   Kitchloo studied a splitting, due to Miller \cite{MillerSplitting}, of $\Sigma^{\infty}_+ SU(n)$.  His theorem is that, \textit{for complex-oriented $E$}, the corresponding direct sum decomposition of $E_*(SU(n))$ is multiplicative.
\end{rmk}

Our proof of Theorem \ref{thm:MainMUE2} is by obstruction theory.  We show in Section \ref{sec:MUE2} that all obstructions to an $\mathbb{E}_2$-equivalence vanish.  On the other hand, we prove Theorem \ref{thm:MainObstruction} by explicitly calculating a non-zero obstruction in Section \ref{sec:Obstruction}.  It remains to discuss Theorem \ref{thm:MainAoo}, the $\mathbb{A}_\infty$ splitting, which is the central result of our paper.

To prove that a filtered spectrum
$$A_0 \longrightarrow A_1 \longrightarrow A_2 \longrightarrow A_3 \longrightarrow \cdots$$
splits, it suffices to provide splitting maps in the form of a ``cofiltered spectrum''
$$A_0 \longleftarrow A_1 \longleftarrow A_2 \longleftarrow A_3 \longleftarrow \cdots$$
In Section \ref{sec:FilGra} we make this statement precise (with the proof in Appendix \ref{app:SplittingMachine}) by explaining the following theorem:
%Appendix \ref{app:SplittingMachine} we make this statement precise (for rigorous definitions of the terms below, see Section \ref{sec:FilGra}):

\begin{customthm}{\ref{thm:SplitMachine}}%\label{thm:introSplitMachine}%maybe this is an iff 
Let $X\in \Alg_{\E_n}(\Fil(\Sp))$ be an $\E_n$ filtered spectrum.  Suppose there exists an $\E_n$ cofiltered spectrum $Y\in \Alg_{\E_n}(\Cofil(\Sp))$ with the following two properties:
\begin{enumerate}
\item There is an equivalence $\mathrm{colim } X \simeq \lim Y$ of $\E_n$-algebras in spectra.
\item The resulting natural maps $X_i \to Y_i$ are equivalences.  
\end{enumerate}
Then, the filtered spectrum $X$ is $\E_n$-split.
\end{customthm}

We wish to apply this theorem in the case where $X$ is the $\mathbb{A}_\infty$ filtered spectrum in Theorem \ref{thm:BottIsAoo}.  In Section \ref{sec:MultWeiss}, we produce the corresponding $\mathbb{A}_\infty$ cofiltered spectrum; the proof is then finished in Section \ref{sec:AooSplit}.  To do this, we extend the methods of \cite{Arone}, who used Weiss calculus to give an elegant second proof of the Mitchell--Richter splitting (without multiplicative structure).  

Arone's idea is to use additional functoriality present in the filtration of $\Omega SU(n)$.  Let $\J$ denote the topological category of complex vector spaces and embeddings, fix a complex vector space $V$, and consider the functor $$G:\J \longrightarrow \mathcal{S}\text{paces}$$
given by $G(W)=\J(V,V \oplus W)$.  Observe that the special unitary group $SU(V \oplus \mathbb{C})$ arises as the value $G(\mathbb{C}) = \J (V,V\oplus \mathbb{C})$.  Weiss calculus provides a toolbox with which to study functors similar to $G$ -- a brief review of the theory is provided at the beginning of Section \ref{sec:MultWeiss}.  %is the argument actually reviewed?

The Bott filtration in fact arises from a sequence of functors $$F_0(W) \to F_1(W) \to F_2(W) \to \cdots \to F(W) := \Sigma^{\infty}_+ \Omega \J(V, V\oplus W)$$ from $\J$ to spectra.  The crux of Arone's argument is that the Weiss polynomial approximation $P_n F(W)$ is precisely the functor $F_n(W)$ \cite{Arone}.  The approximations $$P_n F(W) \rightarrow P_{n-1} P_n F(W) \simeq P_{n-1} F(W)$$ provide Arone with splitting maps.

To obtain an $\mathbb{A}_\infty$ splitting, it is no longer sufficient to merely provide splitting maps.  As explained by Theorem \ref{thm:SplitMachine}, we instead require an $\mathbb{A}_\infty$ structure on the whole system of splitting maps, considered as a cofiltered spectrum.  This will arise from combining two statements: the first is that the functor $F(W)$ takes values in $\mathbb{A}_\infty$ ring spectra, and thus gives an $\mathbb{A}_\infty$ object in the category $\Sp^{\J}$ of functors from $\J$ to spectra.  The second is that the Weiss tower can be viewed as a symmetric monoidal functor from $\Sp^{\J}$ to cofiltered objects in $\Sp^{\J}$.  This observation, which may be of independent interest, is likely known to experts and was suggested to us by Jacob Lurie, but we could not locate in the literature.  We have proven it in the following form, where $\Sp^{\J}_{\text{conv}}$ denotes restriction to certain conveniently convergent functors:
\begin{customthm}{\ref{thm:weissmonoidal}}
The Weiss tower defines a symmetric monoidal functor $$\text{Tow}: \Sp^{\J}_{\text{conv}} \to \Cofil(\Sp^{\J}_{\text{conv}}).$$
\end{customthm}
\begin{rmk*}
The proof works just as well in the context of Goodwillie calculus (see Remark \ref{rmk:goodwilliecase}).  
\end{rmk*}

There are a number of natural and presumably approachable open questions suggested by our work here.  In addition to Question \ref{qst:BottE2}, we highlight the following:

\begin{qst}
Is the Mitchell--Richter filtration of the loop space of a Stiefel manifold always filtered $\mathbb{A}_\infty$?  This is the only obstruction to promoting Theorem \ref{thm:MainAoo} to a result about all such loop spaces.
\end{qst}

\begin{qst}
What are the proper motivic analogues of our results, phrased in the category of $\mathbb{A}^1$-invariant Nisnevich sheaves?
\end{qst}

\begin{qst}
What are the proper equivariant analogues of our result?  See for example \cite{Ullman}, \cite{Tynan}.
\end{qst}

\subsection*{Acknowledgements}
The authors thank Greg Arone, Lukas Brantner, Dennis Gaitsgory, Akhil Mathew, and Haynes Miller for helpful conversations.  We were saddened to learn of the passing of Steve Mitchell during the preparation of this manuscript--his papers were a deep inspiration.  Special thanks are due to the authors' PhD advisors, Mike Hopkins and Jacob Lurie, both for their mathematical expertise and their consistent encouragement; many of the ideas in this paper grew out of their suggestions.  Special thanks are also due to Justin Campbell, James Tao, David Yang, and Yifei Zhao, all of whom spent numerous and invaluable hours answering naive questions about the Beilinson--Drinfeld Grassmannian.  The authors were supported by NSF Graduate Fellowships under Grants DGE-114415 and 1122374.

\subsection*{Notation:} We use $\Sp$ to denote the category of spectra, and $\mathcal{S}$ the category of spaces.  As through the introduction, we will freely use the word category to refer to a not-necessarily truncated $\infty$-category.
