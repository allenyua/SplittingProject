The periodic complex bordism spectrum $MUP$ is the Thom spectrum of the tautological bundle over $BU \times \mathbb{Z}$.  This is not just a spectrum, but a ring spectrum, and the connection between homotopy commutative ring maps $MUP \longrightarrow E$ and formal group laws forms the basis of chromatic homotopy theory.  In fact, $MUP$ is not just a homotopy commutative ring spectrum, but an $\mathbb{E}_\infty$-ring spectrum.  A large amount of attention in recent years \cite{HopLaw,Ando,Walker,Moll,Zhu} has been put into variations of the following fundamental question:

\begin{qst}
What $\mathbb{E}_\infty$-ring spectra $E$ receive $\mathbb{E}_\infty$-ring homomorphisms $MUP \longrightarrow E$?
\end{qst}

To make sense of the above question, one must be very clear about how the $\mathbb{E}_\infty$-ring structure on $MUP$ is defined.  The first definition that the authors recalled is the following, due to \cite{BMMS}:
%The story the authors first heard of this definition runs roughly as follows:

Consider the unstable $J$-homomorphism, which is the symmetric monoidal functor
$$J:\coprod_{n} BU(n) \simeq \{ \text{Complex Vector Spaces}\}^{\simeq} \longrightarrow \text{Spaces}$$
that sends a $d$-dimensional complex vector space $V$ to its one-point compactification $S^V$, a $2d$-dimensional sphere.  After further composing with the suspension spectrum functor $$\Sigma^{\infty}:\text{Spaces} \longrightarrow \text{Spectra},$$ the $J$-homomorphism takes values in the subgroupoid of invertible spectra and automorphisms, known as the Picard category of Spectra.  This implies that $J$ factors through the group completion of the category of complex vector spaces, giving a stable $J$ functor 
$$J:BU \times \mathbb{Z} \simeq \{ \text{Virtual Complex Vector Spaces} \}^{\simeq} \longrightarrow \text{Pic}(\text{Spectra}) \subset \text{Spectra}.$$
Since this stable $J$ functor is symmetric monoidal, its colimit $MUP$ acquires an $\mathbb{E}_\infty$-ring structure \cite{LMS, OmarToby}.

\begin{cnv}
Throughout this paper, whenever we refer to $MUP$ as an $\mathbb{E}_\infty$-ring spectrum, we always give it the $\mathbb{E}_\infty$-ring structure constructed above.  We will refer to this as the \textit{standard} $\mathbb{E}_\infty$-ring structure on $MUP$.
\end{cnv}

In contrast, consider the following alternate construction of the periodic complex bordism spectrum by Snaith \cite{SnaithOriginal}:

Since $BU \simeq \Omega^{\infty} \Sigma^2 ku$ is an infinite loop space, $\Sigma^{\infty}_+ BU$ is an $\mathbb{E}_\infty$-ring spectrum.  Given an element in the homotopy of an $\mathbb{E}_\infty$-ring spectrum, we may invert that element to obtain another $\mathbb{E}_\infty$-ring CITE.  In particular, we may invert the suspension of the Bott element $\beta:S^2 \longrightarrow BU$, considered as an element in stable homotopy $\beta \in \pi_2(\Sigma^{\infty}_+ BU)$.

\begin{thm}[Snaith \cite{SnaithOriginal}]
There is an equivalence of homotopy commutative ring spectra
$$\Sigma^{\infty}_+ BU[\beta^{-1}] \simeq MUP.$$
\end{thm}

Our aim, when beginning this project, was to upgrade Snaith's theorem to an equivalence of $\mathbb{E}_\infty$-ring spectra.  We were thus surprised to obtain the following result, which is the main theorem of this paper:

\begin{thm} \label{thm:main}
There is an equivalence of $\mathbb{E}_2$-ring spectra
$$\Sigma^{\infty}_+ BU[\beta^{-1}] \simeq MUP,$$
but there is \textbf{NOT} an equivalence of $\mathbb{E}_5$-ring spectra.  In particular, there is no equivalence of $\mathbb{E}_\infty$-rings.
\end{thm}

\begin{rmk}
In order to prove Theorem \ref{thm:main}, we will construct a certain $\mathbb{E}_\infty$-ring structure on $H\mathbb{Z}P$, by which we denote periodic integral homology.  Essentially by construction, there will be an $\mathbb{E}_\infty$-ring homomorphism $MUP \longrightarrow H\mathbb{Z}P$, but we will show that there can be no $\mathbb{E}_5$-ring homomorphism $\Sigma^{\infty}_+ BU[\beta^{-1}] \longrightarrow H\mathbb{Z}P$.  As we will explain, one interpretation of our result is that `no choice of total Chern class can be a five-fold loop map.'  This is related to another result of Snaith \cite{SnaithNotMultiplicative}, which states roughly that `the usual choice of total Chern class is not an infinite loop map.'
\end{rmk}

Despite the plethora of work that has gone into understanding the standard multiplication on $MUP$ CITE, there are some reasons to prefer the `exotic' multiplication that is $\Sigma^{\infty}_+ BU[\beta^{-1}]$.  Perhaps the main reason is the latter's clear connection to topological complex $K$-theory, as mediated through another theorem of Snaith:

\begin{thm}[Snaith \cite{SnaithOriginal}]
There is an equivalence of homotopy commutative ring spectra 
$$\Sigma^{\infty}_+ \mathbb{CP}^{\infty}[\beta^{-1}] \simeq KU.$$
\end{thm}

\begin{rmk}
Unlike with periodic complex bordism, the above equivalence can be lifted to one of $\mathbb{E}_\infty$-ring spectra.  This folklore result follows from the existence of a symmetric monoidal inclusion functor
$$\mathbb{CP}^{\infty} \simeq \{\text{Complex Lines},\otimes \}^{\simeq} \longrightarrow \{\text{Virtual Vector Spaces},\otimes \}^{\simeq} \simeq BU \times \mathbb{Z}.$$
\end{rmk}

Since the determinant map $BU \longrightarrow BU(1)\simeq \mathbb{CP}^{\infty}$ is an infinite loop map, one has the following theorem

\begin{thm}[Snaith]
There is a map of $\mathbb{E}_\infty$-ring spectra
$$\Sigma^{\infty}_+ BU[\beta^{-1}] \longrightarrow \Sigma^{\infty}_+ \mathbb{CP}^{\infty}[\beta^{-1}] \simeq KU.$$
\end{thm}

This construction of an $\mathbb{E}_\infty$-orientation of $KU$ is so canonical that it can be ported into other settings, such as motivic homotopy theory CITE.

In contrast, while it seems to be folklore that there is an $\mathbb{E}_\infty$-ring homomorphism $MUP \longrightarrow KU$, the authors could find no proof of this in the literature.  We provide in the final section of this document a computational proof of the slightly weaker result that there is an $\mathbb{E}_\infty$-ring map from $MUP$ into the $2$-completion of $KU$.  We prove this as part of a more general story.

\begin{thm} THIS MAYBE BULLSHIT
Let $E_{(k,\mathbb{G})}$ denote a Morava $E$-theory associated to a height $1$ formal group law $\mathbb{G}$ over a perfect field $k$ of characteristic $2$.  Then there is an $\mathbb{E}_\infty$-ring homomorphism
$$MUP \longrightarrow E_{(k,\mathbb{G})}$$
as well as an $\mathbb{E}_\infty$-ring homomorphism
$$\Sigma^{\infty}_+ BU[\beta^{-1}] \longrightarrow E_{(k,G)}.$$
\end{thm}

\begin{rmk} Applying the theorem in the case of the multiplicative formal group law over $\mathbb{F}_2$ yields the $MUP$ orientation of $KU_2^{\wedge}$.
\end{rmk}

\begin{rmk} We believe that slight variants of our arguments work just as well as odd primes $p>2$.  We work at the prime $2$ for simplicity.
\end{rmk}

A number of open questions are raised by our work here.


\begin{qst} Is there an $\mathbb{E}_\infty$-ring homomorphism $MU \longrightarrow \Sigma^{\infty}_+ BU[\beta^{-1}]$?  Is there an `exotic' $\mathbb{E}_\infty$-ring structure on periodic integral homology which admits an $\mathbb{E}_\infty$-ring map from $\Sigma^{\infty}_+ BU[\beta^{-1}]$?
\end{qst}

\begin{rmk} Tyler Lawson has emphasized to us that there are yet other $\mathbb{E}_\infty$-ring spectra that might naturally be called periodic complex cobordism, such as the Tate spectrum $MU^{tS^1}$.  It would be enlightening to have a catalogue of various $\mathbb{E}_\infty$ `forms' of periodic complex bordism, as well as forms of periodic integral homology such as $H\mathbb{Z}^{tS^1}$.
\end{rmk}

\begin{qst}
In the sense of the previous remark, which forms of periodic complex bordism orient which forms of periodic integral homology.  Is there a form of periodic integral homology which is $\mathbb{E}_\infty$ oriented by $\Sigma^{\infty}_+ BU[\beta^{-1}]$?
\end{qst}

\begin{qst}
In addition to the famous open question of which (height $>1$) Morava $E$-theories are $\mathbb{E}_\infty$-oriented by $MU$, which Morava $E$-theories are $\mathbb{E}_\infty$ oriented by which forms of periodic complex bordism?
\end{qst}