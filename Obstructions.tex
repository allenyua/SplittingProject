
Let $3< n\leq \infty$ be an integer.  We will now show that the $\mathbb{A}_\infty$ splitting $$\Sigma^{\infty}_+ \Omega SU(n) \simeq ???$$ cannot be promoted to an $\mathbb{E}_2$-splitting before smashing with $MU$.  

Suppose that such a splitting existed.  By Remark \ref{rmk:maptosq0}, we would obtain an $\mathbb{E}_2$-ring homomorphism $\Sigma^{\infty}_+ \Omega SU(n) \rightarrow \Sigma^{\infty}_+ \mathbb{CP}^{n-1}$, where $\Sigma^{\infty}_+ \mathbb{CP}^{n-1}$ is given the square-zero multiplication.  Furthermore, the precomposition with the inclusion $\Sigma^{\infty}_+ \mathbb{CP}^{n-1} \longrightarrow \Sigma^{\infty}_+ \Omega SU(n)$ must yield the identity map.  In particular, the map sends the generator of $\pi_2(\Sigma^{\infty}\CP^{n-1})$ to the generator of $\pi_2(\Sigma^{\infty}\Omega SU(n)).$  



Recall now that there is an adjunction \cite{MQRT}
$$\Sigma^{\infty}_+:\textbf{Double Loop Spaces}  \xrightleftharpoons{\quad} \Alg_{\E_2}(\Sp) :GL_1.$$

Using this, we may form the adjoint $\E_2$ map
$$\Omega SU(n) \rightarrow GL_1(\Sigma^{\infty}_+ \mathbb{CP}^{n-1}).$$

The right hand side is identified as an $\E_2$ algebra by Proposition \ref{prop:sq0units}.  In particular, we obtain an $\E_2$ composite $$\phi: \Omega SU(n) \to GL_1(\Sigma^{\infty}_+ \mathbb{CP}^{n-1}) \simeq  GL_1(S^0) \times Q\CP^{n-1} \to Q\CP^{n-1}$$ which has the additional property that it is an isomorphism on $\pi_2$.

We now show that such a map $\phi$ cannot exist due to the operations that exist in the homotopy of an $\E_2$ algebra.  

\begin{obs}Let $Y\in \Alg_{\E_2}(\cS)$, and suppose we are given a map $S^2\to Y$.  This extends to an $\E_2$ map $\Omega^2 S^4 \to Y.$  We may precompose with the map $S^5 \to \Omega^2 S^4$ adjoint to the Hopf map $S^7\to S^4$.  This procedure determines a natural operation $$\nu^u: \pi_2(Y) \to \pi_5(Y)$$ in the homotopy of any $\E_2$-algebra in spaces.  
\end{obs}

\begin{rmk} \label{rmk:multnu1}
The notation is meant to hint at the fact that if $Y = \Omega^\infty X$ comes from a spectrum, then the operation $\nu^u$ is given by multiplication by the element $\nu \in \pi_3(\mathbb{S})^{\wedge}_2$ from the $2$-primary homotopy groups of the sphere spectrum.  Thus, $\nu^u$ is an unstable version of $\nu$ that is already seen in any $\E_2$ algebra in spaces.    
\end{rmk}

Finally, we show that $\phi$ cannot be compatible with $\nu^u$ on homotopy by directly computing $\nu^u$ on either side.

For $n>3$, observe that the natural map $\Omega SU(n) \to BU$ is an isomorphism in homology up to degree $7$.  This implies that $\pi_5(\Omega SU(n)) \simeq \pi_5(BU) \simeq 0$ because $BU$ is even.  Hence, $\nu^u$ is trivial on the generator of $\pi_2(\Omega SU(n)).$  

Similarly, the map $Q\CP^{n-1} \to Q\CP^\infty$ is an isomorphism on $\pi_5$ for $n>3$.  However, it was computed in \cite[Theorem II.8]{Liulevicius} that $\pi_5(\CP^{\infty})=\mathbb{Z}/2$ generated by $\nu$ times the degree $2$ generator.  Hence, by Remark \ref{rmk:multnu1}, if $\beta\in \pi_2(Q\CP^{n-1})$ denotes the generator, then $\nu^u(\beta)\in \pi_5(Q\CP^{n-1})$ is nontrivial.  This implies that there can be no $\E_2$ map $\phi$ which induces an isomorphism on $\pi_2$ and concludes the proof.  

\begin{rmk}%say this better
Taking the limit as $n\to\infty$, we obtain the statement that the map $BU\to Q\CP^{\infty}$ implementing the splitting principle does not lift to an $\E_2$ map.  This map is well-studied: among other places, it appears as the first connecting map in the Weiss tower for the functor $V\mapsto BU(V)$.  As such, it can be seen as a ``$BU$-analog'' to the Kahn-Priddy map.  
\end{rmk}

%maybe remark that BU -> QCP^\infty not being E_2 is still true after taking \Sigma^\infty_+, and remark that this is like a version of the splitting principle. perhaps also remark about stuff being known about this map, its place in the weiss tower, etc.


**************below is the new version that I am currently writing*************

Let $3< n\leq \infty$ be an integer.  We will now show that the $\mathbb{A}_\infty$ splitting $$\Sigma^{\infty}_+ \Omega SU(n) \simeq gr(\Sigma^{\infty}_+ \{ F_{n,k} \})$$ cannot be promoted to an $\mathbb{E}_2$-splitting before smashing with $MU$.  The proof is via a power operation computation.  In particular, the $\mathbb{A}_\infty$ splitting map takes the stabilization of the bottom cell $\beta_l : S^2 \to  \CP^{n-1} \to \Omega SU(n)$ on the left-hand side to the stabilization of the bottom cell $\beta_r: S^2 \to F_{n,1} \simeq \CP^{n-1}$ on the right-hand side.  We construct a power operation $\nu^{s}$ and show that $\nu^{s}(\Sigma^{\infty} \beta_l) \neq \nu^s(\Sigma^{\infty} \beta_r).$  

\begin{obs}Let $Y\in \Alg_{\E_2}(\cS)$, and suppose we are given a map $ S^2\to Y$.  This extends to an $\E_2$ map $\Omega^2 S^4 \to Y.$  We may precompose with the map $h: S^5 \to \Omega^2 S^4$ adjoint to the Hopf map $S^7\to S^4$.  This procedure determines a natural operation $$\nu^u: \pi_2(Y) \to \pi_5(Y)$$ in the homotopy of any $\E_2$-algebra in spaces.  

Correspondingly, for any $X\in \Alg_{\E_2}{\Sp}$, a class in $\pi_2(X)$ determines an $\E_2$ map $\Sigma^{\infty}_+ \Omega^2 S^4 \to X$.  The above map $h$ then determines an operation $\nu^s :\pi_2(X) \to \pi_5(X)$ via precomposition.  This has the property that for $Y\in \Alg_{\E_2}(\cS)$ and $\beta \in \pi_2(Y)$, we have $\nu^s(\Sigma^{\infty} \beta) = \Sigma^{\infty} \nu^u (\beta).$
\end{obs}

\begin{rmk} \label{rmk:multnu}
The notation is meant to hint at the fact that if $Y = \Omega^\infty X$ comes from a spectrum, then the operation $\nu^u$ is given by multiplication by the element $\nu \in \pi_3(\mathbb{S})^{\wedge}_2$ from the $2$-primary homotopy groups of the sphere spectrum.  Thus, $\nu^u$ is an unstable version of $\nu$ that is already seen in any $\E_2$ algebra in spaces.    
\end{rmk}

We now compute the operation $\nu^s$ in the two cases.  
\begin{enumerate}
\item For $n>3$, observe that the natural map $\Omega SU(n) \to BU$ is an isomorphism in homology up to degree $7$.  This implies that $\pi_5(\Omega SU(n)) \simeq \pi_5(BU) \simeq 0$ because $BU$ is even.  Consequently, $\nu^u(\beta) = 0$ and so $\nu^s(\Sigma^{\infty} \beta) = 0.$  

\item rest goes here

\end{enumerate}

