
Let $[n]$ denote the linearly ordered set of integers $0\leq i\leq n$.  Define $\Fil_n = \text{Fun}([n], \Sp^{\J})$ and $\Cofil_n = \text{Fun}([n]^{\text{op}},\Sp^{\J})$.  These categories admit functors to $\Sp^{\J}$ by taking colimit and limit, respectively.  Let $\C_n = \Fil_n \times_{\Sp^{\J}} \Cofil_n.$  Finally, let $\Gr_n = \text{Fun}([n]^{\text{ds}}, \Sp^{\J})$ where $[n]^{\text{ds}}$ denotes the underlying discrete category.  We have the following lemma:

\begin{lem}
For all integers $n>0$, there is a fully faithful functor $i_n:\Gr_{n+1} \to \C_n.$  
\end{lem}
\begin{proof}
An element of $\C_n$ is given by a sequence of functors 
\begin{center}
$X_0 \longrightarrow X_1 \longrightarrow \cdots \longrightarrow X_n \simeq Y_n \longrightarrow \cdots \longrightarrow Y_1 \longrightarrow Y_0$ 
\end{center}
where the middle 
\end{proof}

We may then take inverse limits to get a category $\C_\infty = \Fil(\Sp^{\J}) \times_{\Sp^{\J}} \Cofil(\Sp^{\J})$ and a functor $i: \Gr \to \C_\infty$. 

\begin{cor}
The functor $i$ is fully faithful.
\end{cor}
\begin{proof}
%This amounts to checking that taking inverse limits retains fully faithfulness.  this is obvious, thanks to arpon
\end{proof}

at the end, restrict connectivity so that it's monoidal