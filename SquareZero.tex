
We will now discuss square zero extensions in our framework.  For this, it will be convenient to work with the category $\Gr_u$ of \emph{unital} graded spectra in the strong sense that the unit map induces an equivalence in grading 0.  
Note that there is a fully faithful functor $T:\Sp \to \Gr_u$ which sends a spectrum $A$ to the graded spectrum $$S^0, A, *, *, \cdots.$$  Its essential image is the full subcategory $i: \Gr^{\leq 1}_u \to \Gr_u$ consisting of unital graded spectra $X$ such that $X_k$ is contractible for $k>1$.  In this section, we analyze graded spectra in this subcategory $\Gr^{\leq 1}_u$. Our goal is to show any such graded spectrum admits an essentially unique $\E_n$-algebra structure for any $0\leq n\leq \infty.$  This goal is realized in Proposition \ref{prop:sq0unique}.  


The inclusion $i$ fits into an adjunction
$$L^{\leq 1}:  \Gr_u \xrightleftharpoons{\quad} \Gr_u^{\leq 1} : i$$ where the left adjoint $L^{\leq 1}$ can be thought of as truncating above grading 1.  The localization $L^{\leq 1}$ is visibly compatible with the monoidal structure in the sense that for any $f:X\to Y$ in $\Gr_u$ such that $L^{\leq 1}f$ is an equivalence and any $Z\in \Gr_u$, the natural map $L^{\leq 1} (X\wedge Z) \to L^{\leq 1}(Y\wedge Z)$ is an equivalence.  We are now in the situation of Proposition 2.2.1.9 of \cite{HA}, and so we may conclude that $\Gr_u^{\leq 1}$ inherits a symmetric monoidal structure such that $L^{\leq 1}$ is symmetric monoidal and the inclusion $i$ is lax monoidal.  This monoidal structure can be described explicitly by the formula $$X \otimes_{\Gr_u^{\leq 1}} Y = L^{\leq 1}(X \otimes_{\Gr_u} Y).$$

We may then apply Remark 7.3.2.13 of \cite{HA} to obtain an adjunction at the level of algebras for any integer $0\leq n\leq \infty$:
$$L^{\leq 1}_{\text{alg}}: \Alg_{\E_n}( \Gr_u)  \xrightleftharpoons{\quad} \Alg_{\E_n}(\Gr_u^{\leq 1}) : i_{\text{alg}}.$$

Since the counit $Li \to \text{id}$ before lifting to algebras is an equivalence, we have that the counit $L^{\leq 1}_{\text{alg}}  i_{\text{alg}} \to \text{id}$ is also an equivalence.  This implies in particular that $i_{\text{alg}}$ is fully faithful.  We are now in position to prove the main proposition of this section:

\begin{prop}\label{prop:sq0unique}
Let $0\leq n\leq \infty$ be an integer.  Then, there is a sequence of equivalences of categories $$\Sp \xrightarrow{\bar{T}} \Gr_u^{\leq 1} \longrightarrow \Alg_{\E_n}(\Gr_u^{\leq 1}) \longrightarrow  \Alg_{\E_n}( \Gr_u) \times_{\Gr_u} \Gr^{\leq 1}_u $$ where the first functor $\bar{T}$ is obtained by restricting the codomain of the functor $T:\Sp \to \Gr_u.$  In particular, for any $X\in \Gr_u^{\leq 1}$, the graded spectrum $iX\in \Gr_u$ has an essentially unique $\E_n$-algebra structure.  
\end{prop}
\begin{proof}
The third arrow is defined by $i_{\text{alg}}$, and is an equivalence because $i_{\text{alg}}$ is fully faithful, so it remains to consider the first two arrows.  

We have already seen that the functor $\bar{T}: \Sp \to \Gr^{\leq 1}_u$ is an equivalence of categories.  However, it may be promoted to a symmetric monoidal equivalence when $\Sp$ is given the cocartesian monoidal structure - that is, the monoidal structure defined by $\vee$, the coproduct.  This monoidal structure has a very special property: by Proposition 2.4.3.9 of \cite{HA}, there is for each $n$ an equivalence $\Sp \simeq \Alg^{\vee}_{\E_n}(\Sp)$, where the superscript $\vee$ indicates that we are considering algebras under the wedge.  Informally, this says that any $Y\in \Sp$ admits an essentially unique $\E_n$-algebra structure under the coproduct.  It follows that the same holds for any $X\in \Gr^{\leq 1}_u$, and so there is an equivalence $\Gr^{\leq 1}_u \to \Alg_{\E_n}(\Gr^{\leq 1}_u)$, as desired.    
\end{proof}


\begin{term}
Let $0\leq n\leq \infty$ be an integer.  By taking composing with the colimit functor, Proposition \ref{prop:sq0unique} provides a functor $$\omega_n: \Sp \to \Alg_{\E_n}(\Sp)$$ which we will refer to as the square zero extension.  It sends a spectrum $X$ to a ring with underlying spectrum $S^0\vee X$.  We will call any $\E_n$-algebra structure produced via Proposition \ref{prop:sq0unique} or $\omega_n$ a \emph{square zero} $\E_n$ structure.  
\end{term}

\begin{rmk} \label{rmk:maptosq0}
For any $X\in \Alg_{\E_n}(\Gr_u)$, we have a map $X\to i_{alg}L^{\leq 1}_{alg}X$ of $\E_n$-algebras.  Taking colimits, we obtain a map $\colim X \to \colim i_{alg}L^{\leq 1}_{alg}X$ of $\E_n$ ring spectra.
 We may summarize this informally by saying that any $\E_n$-split ring spectrum $X$ has an $\E_n$ map to the square zero extension determined by its degree one component $X_1$.  
 \end{rmk}

We will need to understand structured maps into square zero extensions.  This amounts to understanding the space of units.  In classical algebra, given a commutative ring $A$ and an $A$-module $M$, the group of units of the square zero extension are given by the formula $$(A\oplus M)^{\times} \simeq A^{\times} \times M.$$ A similar formula holds in our context for suspension spectra of connected spaces.  

\begin{prop}\label{prop:sq0units}
Let $0\leq n\leq \infty$ be an integer and $X\in \cS$ a connected space.  
 There is a canonical equivalence $$GL_1(\omega_n (\Sigma^{\infty} X)) \simeq GL_1(S^0) \times QX$$ of $\E_n$-algebras in spaces, where $QX$ is our notation for $\Omega^{\infty}\Sigma^\infty X$.  
\end{prop}
\begin{proof}
The functors $\omega_n$ are compatible under restriction, so it suffices to prove the statement for $n=\infty.$  For this case, we will show that there is a splitting $$gl_1(\omega_\infty(\Sigma^{\infty}X)) \simeq gl_1(S^0) \vee \Sigma^\infty X$$ of spectra, where $gl_1$ denotes the spectrum of units of an $\E_\infty$-ring introduced in \cite{MQRT}.  We first look at what happens on homotopy.  Recall that for any $\E_\infty$ ring spectrum $Y$, we have the formula $$\pi_*(gl_1(Y)) \simeq (\pi_*(Y))^\times$$ where on the right hand side, we consider $\pi_*(Y)$ as a graded ring.  In our case, this yields an identification $$\pi_*(gl_1(\omega_\infty (\Sigma^{\infty}X))) \simeq (\pi_*(S^0) \oplus \pi_*(\Sigma^{\infty}X))^\times \simeq \pi_*(S^0)^{\times} \times \pi_*(\Sigma^\infty X)$$
where we have used that on homotopy, $\omega_\infty (\Sigma^\infty X)$ is a square zero extension.  To conclude the proof, it suffices to show that the two factors on the right hand side can be realized by maps of spectra.  

The first factor is realized simply by $gl_1$ of the unit map $S^0 \to \omega_\infty (\Sigma^\infty X).$  In fact, it is not difficult to see directly that this map is split.  
%First, note that there is a tautological retraction sequence $$*\longrightarrow \Sigma^\infty X \longrightarrow *$$ of spectra, where $*\in \Sp$ denotes the zero object.  Applying $gl_1\circ \omega_\infty$ allows us to conclude that $gl_1(S^0) = gl_1(\omega_\infty(*))$ splits off of $gl_1(\omega_\infty(\Sigma^\infty X))$ in spectra.  In particular, this determines a map $$a:gl_1(S^0) \to gl_1(\omega_\infty (\Sigma^\infty X)).$$

For the second factor, observe that since $\omega_\infty (\Sigma^\infty X)$ is an $\E_\infty$-ring, it receives a canonical $\E_\infty$ map $$\Sigma^{\infty}_+ QX \longrightarrow \omega_\infty (\Sigma^\infty X)$$ from the free $\E_\infty$ ring on $\Sigma^\infty X$ which extends the canonical map of spectra $\Sigma^\infty X \to \omega_\infty (\Sigma^\infty X).$  Now, note that there is an adjunction \cite{MQRT} $$\Sigma^{\infty}_+\Omega^\infty : \Sp \xrightleftharpoons{\quad} \Alg_{\E_\infty}(\Sp) : gl_1$$ under which the above map may be identified with a map $$b: \Sigma^\infty X \to gl_1 (\omega_\infty (\Sigma^{\infty} X))$$ of spectra.  \textbf{NEED TO SAY A TINY BIT MORE}
%how to compute this on htpy...more or less obv but maybe should explain


Finally, we may take the map $a\vee b: gl_1(S^0)\vee \Sigma^\infty X \to gl_1(\omega_\infty (\Sigma^{\infty} X))$ and the above comments show that it is an equivalence, as desired.  








%recall the symmetric monoidal equivalence $\bar{T}:\Sp \to \Gr^{\leq 1}_u$ of Proposition \ref{prop:sq0unique}, where $\Sp$ is given a monoidal structure under $\vee$.  By the proof of Proposition \ref{prop:sq0unique}, there is a canonical retraction sequence of $\E_\infty$ objects in $\Sp$ (under $\vee$) $$* \longrightarrrow X \longrightarrow *.$$  Taking the associated square $0$ spectra



%for general reasons we know gl_1 S^0 splits off gl_1 S^0 v X
%by the \Sigma^\infty \Omega^\infty // gl_1 adjunction and the fact that \Sigma^\infty Q is free E_\infty, you get a map from \Sigma^\infty X to gl_1 S^0 v X that's nontrivial enough on homotopy...so then you just add em up and you're done

\end{proof}