The main result of \cite{Arone} shows that the Mitchell-Richter filtration on $\Omega SU(n)$ (and more generally, for loops on a Stiefel manifold) stably splits.  In this section, we show how to produce $\mathbb{A}_\infty$ stable splittings of $\Omega SU(n)$.  %and stiefel mflds?? 

%segue

Arone's key insight is that this filtration has extra structure: it is a particular value of a \emph{functor} which has a natural filtration.  Weiss calculus then allows him to exploit this structure by producing splitting maps.  To produce $\mathbb{A}_\infty$ splittings, it is no longer sufficient to simply produce splitting maps backward; instead we need the criterion of Theorem \ref{thm:SplitMachine}.  It requires two inputs: an $\mathbb{A}_{\infty}$ filtered spectrum and an $\mathbb{A}_{\infty}$ cofiltered spectrum.  The first of these is given by Theorem \ref{thm:AooFil} and the latter of these by Corollary \ref{cor:aronemonoidal}.  %they match because they're both loops SU(n). 
