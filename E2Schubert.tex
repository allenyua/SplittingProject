
In the next section we will define the Mitchell--Richter Bott filtration on $\Omega SU(n)$ and prove that it makes $\Sigma^{\infty}_+ \Omega SU(n)$ into a filtered $\mathbb{A}_\infty$-ring spectrum.  The filtration, especially as a multiplicative object, is most naturally described in the language of algebraic geometry.  Our aim in this section will be to recall the complex points of these algebro-geometric objects, and extract from both the algebro-geometric and topological literature all of the basic facts about these complex points that we will later need.  None of the ideas in this section are original: the objects we study are due to Beilinson and Drinfeld \textbf{CITE}, and their translation into algebraic topology is due to Jacob Lurie  \cite[\S 5.5]{HA}.

\subsection{The Schubert filtration on the Affine Grassmannian}

Fix for the moment a smooth, reductive, affine algebraic group $G$ over $\mathbb{C}$.  Through the rest of the paper, we will be interested only in the cases $G=\mathbb{G}_m,SL_n,$ or $GL_n$, and so the reader may safely restrict his attention to those cases for concreteness.  As we will explain in detail, an algebro-geometric model for $\Omega G(\mathbb{C})$ is the \textit{affine Grassmannian} $Gr_G$.  The $\mathbb{E}_2$-algebra structure present on $\Omega G(\mathbb{C})$ is encoded by a more elaborate object, the \textit{Beilinson--Drinfeld Grassmannian}.  A good general reference for both of these objects is \cite{Zhu}, whose presentation we will more or less follow below.

We use $D$ to denote the formal disk $\text{Spec}(\mathbb{C}[[t]])$ and $D^*$ to denote the punctured disk $\text{Spec}(\mathbb{C}((t)))$.  For $R$ a $\mathbb{C}$-algebra, we use $D_R$ to denote $\text{Spec}(\mathbb{C}[[t]] \hat{\otimes} R)$ and $D^*_R$ to denote $\text{Spec}(\mathbb{C}((t)) \hat{\otimes} R)$.  We refer to a point in the space $G(D^*)=G(\mathbb{C}((t)))$ as an algebraic free (i.e., unbased) loop in $G$.  Such points are exactly automorphisms of the trivial $G$-torsor $\mathcal{E}^0$ over $D^*$.

\begin{dfn}
The \textit{affine Grassmannian} $Gr_G$ of $G$ is the Ind-scheme with functor of points
$$Gr_G(R) = \{(\mathcal{E},\beta)\},\text{ where}$$
$\mathcal{E}$ is a $G$-torsor over $D_{R}$ and $\beta:\mathcal{E}|_{D^*_{R}} \cong \mathcal{E}^0_{D^*_{R}}$ is a trivialization over $D^*_{R}$.
\end{dfn}

The complex points $Gr_G(\mathbb{C})$ are a model for the topological space $\Omega G$ CITE.  These complex points are given by the homogeneous space CITE
$$G(\mathbb{C}((t)))/G(\mathbb{C}[[t]]),$$
which up to homotopy is the quotient of the free loop space on $G$ by the action of $G$.

The above functor of points is not representable by a scheme, but it is a filtered colimit of schemes $Gr_{G,\le \mu}$.  We will define below at least the complex points $Gr_{G, \le \mu}(\mathbb{C})$ of these schemes, which are specific topological subspaces of $Gr_G(\mathbb{C})$.  First, it is necessary to introduce a bit more notation.

Let $T$ denote a maximal torus inside $G$.  We use $\mathbb{X}^{\bullet}$ to denote the lattice of weights $\Hom(T,\mathbb{G}_m)$, and $\mathbb{X}_{\bullet}$ to denote the dual lattice of coweights.  Inside $\mathbb{X}^{\bullet}$ is the set $\Phi$ of roots.  We fix a particular Borel subgroup $B \subset G$, determining a choice of positive roots $\Phi^+ \subset \Phi$ and a semi-group of dominant coweights $\mathbb{X}^+_\bullet \subset \mathbb{X}_\bullet$.  There is a natural bijection
$$\mathbb{X}_{\bullet}^+ \cong G(\mathbb{C}[[t]])\backslash G(\mathbb{C}((t)))/G(\mathbb{C}[[t]])$$
of dominant coweights with the above double cosets.  Each coweight $\mu \in \Hom(\mathbb{G}_m,T)$ defines via the inclusion of $T$ into $G$ an element $t^{\mu}$ in $G(\mathbb{C}((t)))$.  There is a double-coset decomposition of the algebraic free loop space
$$
G(\mathbb{C}((t))) \cong \coprod_{\mu \in \mathbb{X}_{\bullet}^+} G(\mathbb{C}[[t]]) t^{\mu} G(\mathbb{C}[[t]]).
$$

Projecting onto the affine Grassmannian, one learns that the $G(\mathbb{C}[[t]])$-orbits of $Gr_G$ are indexed by $\mu \in \mathbb{X}_{\bullet}^+$.
We will use $Gr_{G,\le \mu}$ to denote the \textit{closure} of the orbit corresponding to $\mu$.  The closure $Gr_{G, \le \mu_1}$ contains $Gr_{G, \le \mu_2}$ if and only if $\mu_1-\mu_2$ is a sum of dominant coroots.

\begin{dfn} \label{def:coweightfiltered}
An $\mathbb{X}_{\bullet}^+$-filtered topological space $K$ is a poset of topological subspaces of $K$, indexed by $\mathbb{X}_{\bullet}^+$.
\end{dfn}

An example is of course given by $Gr_{G}(\mathbb{C})$, filtered by the $Gr_{G,\le \mu}(\mathbb{C})$.  We will refer to this filtration on $Gr_{G}(\mathbb{C})$ as the \textit{Schubert filtration}.

\begin{rmk} \label{rem:monodromy}
Suppose that $\gamma$ is a principal $G$-bundle on $\mathbb{A}^1$, and that $p$ is the origin in $\mathbb{A}^1$.  Then restriction of $\gamma$ to an infinitesimal neighborhood of $p$ gives a principal $G$-bundle on the formal disk $D$.  If one is then further given a trivialization of $\gamma$ away from $p$, there is then an associated point $x \in Gr_{G}(\mathbb{C})$.  This point $x$ lives in a certain $G(\mathbb{C}[[t]])$-orbit, and thus has well-defined \textit{monodromy} $\mu \in \mathbb{X}_{\bullet}^+$.

Now, if $p$ is \textit{any} point in $\mathbb{A}^1$, then a formal neighborhood of $p$ is isomorphic but not equal to the formal disk $D$.  If given a trivialization of $\gamma$ away from $p$, this means that one does not get a well-defined point $x \in Gr_{G}(\mathbb{C})$, but one \textit{does} obtain a well-defined $G(\mathbb{C}[[t]])$-orbit inside of $Gr_{G}(\mathbb{C})$.  In other words, the trivialization of $\gamma$ away from $p$ has well-defined monodromy $\mu \in \mathbb{X}_{\bullet}^+$.  Notice, in fact, that to define $\mu$ one needs not a trivialization on all of $\mathbb{A}^1 \backslash \{ p \}$, but only a trivialization defined away from $p$ in a neighborhood of $p$.
\end{rmk}



\begin{exm} \label{sl2example}
Suppose $G=SL_2(\mathbb{C})$ with its usual maximal torus.  A coweight $\mu \in \mathbb{X}_\bullet$ consists of a pair $(a,b)$ of integers with $a+b=0$.  We choose a Borel so that a coweight is dominant if $a \ge b$.  The conjugation action of $SL_2(\mathbb{C})$ on $\Omega SL_2(\mathbb{C})$ has one orbit for each pair $(a,-a)$ with $a \ge 0$.  The orbit corresponding to $(a,-a)$ contains the loop $\mathbb{G}_m \rightarrow \Omega SL_2(\mathbb{C})$ given by
$$
t \mapsto \left( \begin{array}{cc} t^a & 0 \\ 0 & t^{-a}  \end{array} \right).
$$
The closure of the $(a,-a)$ orbit contains the $(b,-b)$ orbit if and only if $b \le a$.  To topologists, $\Omega SL_2(\mathbb{C}) \simeq \Omega \Sigma S^2$ is recognizable as the free $\mathbb{A}_\infty$-algebra on the pointed space $S^2$.  In particular, $Gr_{SL_2}(\mathbb{C})$ is naturally equipped with the James filtration by word length.  The closure of the $(a,-a)$ orbit, denoted $Gr_{SL_2,\le (a,-a)}(\mathbb{C})$, turns out to be the $(2a)$th component of the James filtration.

Thus, the Schubert filtration is strictly coarser than the James filtration.  In particular, the $S^2$ that appears as the first James filtered piece of $\Omega SL_2(\mathbb{C})$ is not closed under the $SL_2(\mathbb{C})$ conjugation action.  Only the collection of words of length $2$ or less is closed under the $SL_2(\mathbb{C})$ action.

In this paper, we are interested in a filtration called the \emph{Bott filtration} of $\Omega SL_n(\mathbb{C}).$  For $\Omega SL_2(\mathbb{C}),$ the Bott filtration corresponds to the James filtration on $\Omega S^3$, and in particular is \textbf{not} given by the Schubert filtration on $Gr_{SL_2}(\mathbb{C})$.  However, in Section CITE, we will explain how to obtain the Bott filtration from the Schubert filtration, while in the remainder of this section we will explain how the Schubert filtration interacts with the $\mathbb{E}_2$-algebra structure on $\Omega SL_2(\mathbb{C})$.
\end{exm}

\subsection{$\mathbb{E}_2$-algebras via disk algebras}

The $\mathbb{E}_2$-algebra structure on $\Omega G$ is elegantly encoded in algebraic geometry through the notion of the Beilinson--Drinfeld Grassmannian.  In this section we explain this structure and why it gives $\mathbb{E}_2$ algebras in topology.  We start by explaining the picture in topology following \cite[\S 5.5]{HA}.

Let $X$ be a topological space.  We first construct a space $\mathrm{Ran}(X)$ called the \emph{Ran space} of $X$ whose points are non-empty finite subsets of $X$.  Note that $X$ determines a contravariant functor from the category $\Fin_{\text{surj}}$ of finite sets and surjections to the category $\mathrm{Top}$ of topological spaces carrying a finite set $S$ to $X^S$ and a surjection $S \twoheadrightarrow T$ to the natural diagonal map $X^T \to X^S.$

\begin{dfn}
Let $\Fin^{\leq n}_{\mathrm{surj}}$ denote the category of finite sets of cardinality at most $n$.  Define the topological spaces $$\Ran^{\leq n}(X) := \colim_{S\in \Fin^{\leq n}_{\mathrm{surj}}} X^S$$ and $$\Ran(X) := \colim_{n\to \infty} \Ran^{\leq n}(X)$$ where the colimits are taken in the 1-category of topological spaces.  
\end{dfn}

\begin{rmk}
There is another topology on the Ran space, given for instance in \cite[\S 5.5]{HA}.  For our purposes, it will not matter which topology we use, so we will use the colimit topology throughout.  
\end{rmk}

Classically, given a symmetric monoidal category $\C$, one thinks of an $\mathbb{E}_2$-algebra $A$ as an object with multiplications parametrized by embeddings of disjoint unions of disks.  We can think of this as giving for each point in the plane $x\in \mathbb{C}$ a copy $A_x$ of $A$.  On an infinitesimal scale, the embeddings of disks can be thought of as the collision of points in the plane.  The Ran space precisely tracks this collision data, and it is through this that the Ran space gives another way of thinking about an $\mathbb{E}_2$-algebra.  More precisely, given an $\mathbb{E}_2$-algebra $A$ and a finite set of points in the plane $\{x_i\}_{i\in I} \in \Ran(\mathbb{C})$, one can form the tensor product $\bigotimes_{x\in X} A_x.$  These tensor products turn out to be stalks of a cosheaf on $\Ran(\mathbb{C})$; the functoriality in the Ran space expresses the idea of points colliding determining the multiplicative structure.  

This story can be made precise in the language of $\infty$-operads as in \cite[\S 5.4.5]{HA}.  There is a colored operad $\mathrm{Disk}(\mathbb{C})$ whose colors are disks (open subspaces of $\mathbb{C}$ which are homeomorphic to $\mathbb{R}^2$).  The set of operations from $(D_1, D_2, \cdots, D_n)$ to $D$ is the singleton if the $D_i$ are disjoint and contained in $D$, and empty otherwise.  From this colored operad, we may obtain an $\infty$-operad $\N(\mathrm{Disk}(\mathbb{C}))^{\otimes}$ whose algebras we will refer to as $\N(\mathrm{Disk}(\mathbb{C}))$-algebras.  A $\N(\mathrm{Disk}(\mathbb{C}))$ algebra $A$ is the data of an object $A(D)\in\C$ for each disk $D\subset \mathbb{C}$ and coherent maps $A(D_1)\otimes A(D_2)\otimes \cdots \otimes A(D_n) \to A(D)$ for inclusions $D_1\coprod \cdots \coprod D_n \to D$ of disks.  %We will produce a $N(\mathrm{Disk}(\mathbb{C}))$ algebra by 

The following theorem relates these to $\mathbb{E}_2$-algebras\footnote{For technical reasons, we will work with the non-unital (abbreviated `$\mathrm{nu}$') variants of $\mathbb{E}_2$ and $\mathrm{Disk}(\mathbb{C})$-algebras from now on.}:

\begin{thm}[Proposition 5.4.5.15, \cite{HA}]\label{thm:lcdisk}
There is a fully faithful functor $\Alg^{\mathrm{nu}}_{\mathbb{E}_2}(\C) \to \Alg^{\mathrm{nu}}_{\N(\mathrm{Disk}(\mathbb{C}))}(\C)$ with essential image the full subcategory of $\N(\mathrm{Disk}(\mathbb{C}))_{\mathrm{nu}}$-algebras $A$ which are \emph{locally constant} in the sense that for every embedding $D\hookrightarrow D'$ of disks, the natural map $A(D)\to A(D')$ is an equivalence.  
\end{thm}

We will make our $\mathbb{E}_2$-algebras by constructing a $\N(\mathrm{Disk}(\mathbb{C}))_{\mathrm{nu}}$-algebra and checking the condition of Theorem \ref{thm:lcdisk}.  

\begin{comment}
The Ran viewpoint is captured by an $\infty$-operad $\mathrm{Fact}(\mathbb{C})^{\otimes}$ whose algebras $A$ are the data of an object $A(U)\in\C$ functorially for every open set $U\subset \mathrm{Ran}(\mathbb{C})$ together with structure of \emph{factorization} maps $A(U)\otimes A(V) \to A(U\times V)$ whenever $U$ and $V$ are independent subsets of $\mathrm{Ran}(\mathbb{C})$ (and similar data for more opens).\footnote{The reader unfamiliar with the Ran space can keep in mind the example when $U=\mathrm{Ran}(D_1)$, $V=\mathrm{Ran}(D_2)$ for disjoint disks $D_1,D_2\subset \mathbb{C}$, in which case $U\times V = \mathrm{Ran}(D_1\coprod D_2).$}  %NOPE THIS IS WRONG!

%\begin{rmk}Given a space $f: X \to \Ran(\mathbb{C})$, one can provide part of the data of
%We emphasize that these factorization maps are \emph{extra structure}.  somehow we should emphasize this when talking about the BDGr, since the referee asked about it

As alluded to earlier, the notions of $\mathrm{Fact}(\mathbb{C})$ algebra and $\mathrm{Disk}(\mathbb{C})_{\mathrm{nu}}$ algebra are closely related.  There is a map of $\infty$-operads $\mathrm{Disk}(\mathbb{C})_{\mathrm{nu}}^\otimes \to \mathrm{Fact}(\mathbb{C})^\otimes.$  Concretely, this means that one can extract from any $\mathrm{Fact}(\mathbb{C})$ algebra $A$ an underlying $\mathrm{Disk}(\mathbb{C})_{\mathrm{nu}}$ algebra $\overline{A}$ by the formula $\overline{A}(D) = A(\mathrm{Ran}(D)).$  The following theorem of Lurie describes the relationship more precisely:

\begin{thm}[Theorem 5.5.4.10, \cite{HA}]
Let $\C$ be a symmetric monoidal $\infty$-category which admits all colimits and such that the tensor product preserves colimits separately in each variable.  Then operadic left Kan extension along the functor $\mathrm{Disk}(\mathbb{C})_{\mathrm{nu}}^\otimes \to \mathrm{Fact}(\mathbb{C})^\otimes$ defines a fully faithful functor $F: \Alg_{\mathrm{Disk}(\mathbb{C})_{\mathrm{nu}}}(\C) \to \Alg_{\mathrm{Fact}(\mathbb{C})}(\C)$.  Moreover, $F$ restricts to an equivalence between the full subcategory of locally constant $\mathrm{Disk}(\mathbb{C})_{\mathrm{nu}}$ algebras and the full subcategory of $\mathrm{Fact}(\mathbb{C})$ algebras $A$ which are constructible cosheaves and for which the factorization maps $A(U)\otimes A(V) \to A(U\times V)$ are equivalences.  
\end{thm}

\begin{rmk}
In fact, this theorem will be logically unnecessary for what follows and we include it only for completeness.  As such, we omit the definition of the term \emph{constructible cosheaf}.  
\end{rmk}

We summarize the preceding discussion by drawing the following corollary of Theorem~\ref{thm:lcdisk}:
\begin{cor} \label{cor:E2crit}
Let $\C$ be a symmetric monoidal $\infty$-category and $A \in \C$ be a $\mathrm{Fact}(\mathbb{C})$ algebra.  Suppose that $A$ has the property that for any inclusion $D\subset D' \subset \mathbb{C}$ of disks, the natural map $A(\mathrm{Ran}(D))\to A(\mathrm{Ran}(D'))$ is an equivalence.  Then, for any disk $D\subset \mathbb{C}$, $A(\mathrm{Ran}(D))$ inherits the structure of a non-unital $\mathbb{E}_2$-algebra in $\C$.  
\end{cor}
\end{comment}


\subsection{The Beilinson-Drinfeld Grassmannian}
The structure described in the previous section arises naturally in algebraic geometry in the form of the Beilinson-Drinfeld Grassmannian.  Our goal in this section is to set up this object and describe the $\mathbb{E}_2$ algebras it produces via Theorem \ref{thm:lcdisk}.

\begin{dfn}
The \textit{Ran space} $\text{Ran}_{\mathbb{A}^1}$ is the presheaf that assigns to every $\mathbb{C}$-algebra $R$ the set of non-empty finite subsets of $\text{Spec}(R) \times \mathbb{A}^1$ over $\text{Spec}(R)$.   The Beilinson--Drinfeld Grassmannian is the presheaf $Gr_{G,\text{Ran}}$ that assigns to each $\mathbb{C}$-algebra $R$ the set of triplets $(x,\mathcal{E},\beta)$, where $x \in \text{Ran}_{\mathbb{A}^1}(R)$, $\mathcal{E}$ is a $G$-torsor on $\mathbb{A}^1 \times \text{Spec}(R)$, and $\beta$ is a trivialization of $\mathcal{E}$ away from the graph of $x$ in $\text{Spec}(R) \times \mathbb{A}^1$.
\end{dfn}

The Beilinson--Drinfeld Grassmannian is naturally fibered over the Ran space.  Again, we shall be primarily interested in complex points.  Let us denote the map on complex points by $$\Gra_{G,\Ran}(\mathbb{C}) \xrightarrow{p} \Ran(\mathbb{C})$$ and for an open set $U\in \mathbb{C}$, let $\Gra_{G,\Ran}(U):= p^{-1}(\Ran(U)).$   As explained above, a point $x$ in $\text{Ran}(\mathbb{C})$ consists of a non-empty finite subset $I \subset \mathbb{C}$ of points in $\mathbb{C}$.  The fiber of the Beilinson--Drinfeld Grassmannian $Gr_{G,\text{Ran}}(\mathbb{C})$ over $x$ is the moduli of $G$-bundles on $\mathbb{A}^1$ equipped with a trivialization away from the points in $I$.  This fiber is non-canonically isomorphic to the product of $|I|$ copies of affine Grassmannian $Gr_G(\mathbb{C})$.  

Via the machinery in the previous section (as we will explain in more detail later), $Gr_{G,\text{Ran}}(\mathbb{C})$ describes the $\mathbb{E}_2$-algebra structure on the affine Grassmannian.  However, we are interested in the compatibility of this $\mathbb{E}_2$ structure with certain filtrations on $Gr_G(\mathbb{C})$.  We must therefore exhibit a filtration on the whole Beilinson-Drinfeld Grassmannian.  In particular, we filter $Gr_{G,\text{Ran}}(\mathbb{C})$ by spaces $Gr_{G,\text{Ran},\le \mu}$, allowing us to view $Gr_{G,\text{Ran}}(\mathbb{C})$ as an $\mathbb{X}_{\bullet}^+$-filtered topological space in the sense of Definition \ref{def:coweightfiltered}.  

To do so, consider an arbitrary point $p \in \text{Gr}_{G,\text{Ran}}(\mathbb{C})$, consisting of a $G$-bundle $\gamma$ on $\mathbb{A}^1$ trivialized away from a collection of points $I=(i_1,i_2,...,i_\ell) \subset \mathbb{C}$.  For each point $i_k \in \mathbb{C}$, the restriction of $\gamma$ to a formal neighborhood of $i_k$ has well-defined monodromy $\mu_k$, in the sense of Remark \ref{rem:monodromy}.  We say that $p$ lives in $\text{Gr}_{G,\text{Ran},\le \mu}(\mathbb{C})$ if and only if
$$\sum_{i=1}^{k} \mu_k \le \mu.$$  This defines $\Gra_{G,\Ran}(\mathbb{C})$ as a $\mathbb{X}^+_{\bullet}$-filtered topological space.

We will ultimately be concerned with certain filtered spectra that one can extract from $\Gra_{G}(\mathbb{C})$ and its coweight filtration.  Given a dominant coweight $\mu \in \mathbb{X}^+_{\bullet}$, one obtains a filtered spectrum $ \{ \Sigma^{\infty}_+ \Gra_{G,\leq k\mu}(\mathbb{C})\}_{k\in \mathbb{Z}_{\geq 0}}$. In the case when the sublattice of $\mathbb{X}^+_{\bullet}$ generated by $\mu$ is cofinal, the filtration is exhaustive.  We will now choose and fix such a dominant coweight $\mu$.  Our goal for the remainder of the section is to prove the following fact:

\begin{prop}\label{prop:schubertE2}
The filtered spectrum $\{\Sigma^{\infty}_+ \Gra_{G, k \mu} \}_{k\in \mathbb{Z}_{\geq 0}}$ naturally admits the structure of an $\mathbb{E}_2$-algebra in filtered spectra.
\end{prop}

\begin{cnstr}\label{cnstr:diskalg}
We proceed by defining a $\mathrm{Disk}(\mathbb{C})_{\mathrm{nu}}$ algebra $\bar{A}$ valued in filtered topological spaces.  This will arise from the map $p:\Gra_{G,\Ran}(\mathbb{C}) \to \Ran(\mathbb{C})$ constructed above.  For a disk $D\subset \mathbb{C}$, we define $\bar{A}(D) := \Gra_{G,\Ran}(D).$  Explicitly, the points of $\bar{A}(D)$ are given by the data of a pair $(I, \mathcal{F}_I)$ where $I\subset D$ is a nonempty finite subset and $\mathcal{F}_I$ is a principal $G$-bundle with a trivialization away from $I$.  Let $D_1 \coprod D_2\coprod \cdots \coprod D_n \hookrightarrow D$ be the nonempty inclusion of disjoint disks into a larger disk $D$.  To define a $\mathrm{Disk}(\mathbb{C})_{\mathrm{nu}}$ algebra, we need to exhibit maps $\bar{A}(D_1)\times \cdots \times \bar{A}(D_n) \to \bar{A}(D)$ satisfying the obvious coherences.  %This is the key extra structure that the Beilinson-Drinfeld Grassmannian possesses
This map sends the point determined by $(I_1, \mathcal{F}_1), \cdots, (I_n, \mathcal{F}_n)$ to the point $(I, \mathcal{F})$ where $I= I_1 \coprod \cdots \coprod I_n$ and $\mathcal{F}$ is obtained by gluing the $\mathcal{F}_i$ along the trivializations away from $I$.  
\end{cnstr}
\begin{rmk}
More generally, the Beilinson-Drinfeld Grassmannian yields the structure of a \emph{factorizable cosheaf} on the Ran space.  The key structure present is that of \emph{factorization}, which roughly says that the Grassmannian over two disjoint subsets $U,V\subset \mathbb{C}$ is equivalent to the product of the Grassmannians over the two sets separately.  One sees this is also the structure that is giving us the disk algebras in Construction \ref{cnstr:diskalg}.  However, we have chosen to work directly with the underlying $\mathrm{Disk}(\mathbb{C})$-algebra and avoid directly discussing factorization algebras; the interested reader can refer to \cite[\S 5.5.4.10]{HA} for a comparison of these concepts.  
\end{rmk}

Passing to $\infty$-categories and stabilizing, we obtain a $\N(\mathrm{Disk}(\mathbb{C}))_{\mathrm{nu}}$-algebra $A$ valued in filtered spectra whose value on a disk $D$ is $A(D) = \Sigma^{\infty}_+ \bar{A}(D) = \Sigma^{\infty}_+ \Gra_{G,\Ran}(D)$.  The following theorem is the essential non-formal input to producing an $\mathbb{E}_2$-algebra.  

\begin{prop}\label{prop:diskequiv}
Suppose that $G$ is reductive and let $D\subset \mathbb{C}$ be a disk and $\{x \} \hookrightarrow D$ be a point.  Then, the natural map $\Gra_{G,\Ran}(\{x\}) \to \Gra_{G,\Ran}(D)$ induces an equivalence of filtered spectra.  
\end{prop}

The theorem will come down to the following key fact:

\begin{lem} \label{lem:grproper}
Suppose that $G$ is reductive.  For each finite set $I$ and each $\mu \in \mathbb{X}_{\bullet}^+$, define $P$ so that the following square is a pullback in the category of topological spaces:
$$
\begin{tikzcd}
P \arrow{r} \arrow{d}{f} & Gr_{G,\text{Ran},\le \mu}(\mathbb{C}) \arrow{d} \\
\mathbb{C}^I \arrow{r} & \text{Ran}_{\mathbb{A}^1}(\mathbb{C}).
\end{tikzcd}
$$
Then the map $f$ is a proper map of topological spaces.
\end{lem}

The authors understand the above fact to be folklore in the geometric representation theory literature.  A reference is BLAH.  A sketch of a proof in the algebro-geometric context can be found in BLAH.  \textbf{Figure out a better way to sell this}


We also need the following point-set lemma:

\begin{lem}\label{lem:colimproper}
Let $I$ be a finite indexing category and $\{X_i\}_{i\in I}$ be a diagram of topological spaces indexed by $I$.  Let $X = \colim_{i\in I} X_i$ and suppose $f: Y \to X$ is a map of topological spaces with the property that for all $i\in I$, the natural map $f_i: Y\times_X X_i \to X_i$ is proper.  Then the map $f$ is proper.
\end{lem}
\begin{proof}
****write some shit or not?****
\end{proof}

For any $X\subset \mathbb{C}$, $\mathrm{Ran}^{\leq n}(X)$ is a finite colimit of spaces of the form $X^J$ for a finite set $J$.  Combining this with Lemma \ref{lem:grproper}, we obtain the following corollary:

\begin{cor}\label{cor:ranproper}
Let $X\subset \mathbb{C}$.  Then, the restriction of the Beilinson-Drinfeld Grassmannian $\Gra(X)_{G,\bullet}^{\leq n} \to \mathrm{Ran}^{\leq n}(X)$ is proper.
\end{cor}

\begin{proof}[Proof of Proposition \ref{prop:diskequiv}]
In order to show an equivalence of filtered spectra, we would like to show that for each $k\in \mathbb{Z}_{\geq 0}$, the natural map $\Gra_{G,\leq k \mu}(\{ x\}) \to \Gra_{G,\leq k\mu}(D)$ induces an equivalence of spectra.  It suffices to show that this is true upon restricting to $\Ran^{\leq n}$ for each $n$, because $\Gra_{G, \leq k \mu} = \colim_n \Gra^{\leq n}_{G,\leq k\mu}$ is a filtered colimit of closed inclusions.  For ease of notation, fix $n,k\in \mathbb{Z}_{\geq 0}$ and let $\Gra$ stand in for $\Gra^{\leq n}_{G,\leq k \mu}$ for the remainder of the proof.  

We have a pullback square of topological spaces %is p some sort of fibration?
\begin{equation*}
\begin{tikzcd}
\Gra (\{ x\}) \arrow[r,"i"]\arrow[d,"p_x"] & \Gra (D) \arrow[d,"p"]\\
\{x \} = \Ran^{\leq n}(\{ x\} )\arrow[r,"j"] & \Ran^{\leq n}(D)
\end{tikzcd} \end{equation*}
where $p$ and $p_x$ are proper by Corollary \ref{cor:ranproper}.  We show that the upper map $i$ induces an isomorphism on cohomology.  %why is that enough? can we show Gr(D) is simply connected? seems fine if you know p is a fibration; well we only need an equivalence of spectra

Applying the proper base change theorem to the constant sheaf $\underline{\mathbb{Z}}$ on $\Gra(D)$, we find that there is a natural equivalence $j^*p_*(\underline{\mathbb{Z}}) \xrightarrow{\sim} (p_x)_{*}i^*(\underline{\mathbb{Z}}).$  The right-hand side is $C^*(\Gra(\{x\}) ;\mathbb{Z})$, the integral cohomology of the fiber at $x$.  On the other hand, the left-hand side is by definition $$\colim_{x\in U} C^*(p^{-1}(U);\mathbb{Z})$$ where the colimit is taken over $U\subset \Ran^{\leq n}(D)$ containing $x$.  We may compute this colimit by extracting a cofinal sequence of opens $$x \subset \cdots \subset \Ran^{\leq n}(D_k) \subset \cdots \subset \Ran^{\leq n}(D_0)$$ where $D_i$ are disks such that $D_0=D$.  The transition maps in this system induce equivalences because there is a deformation retraction of $D_k$ onto $D_{k+1}$.  Therefore, we conclude that $$\colim_{x\in U} C^*(p^{-1}(U);\mathbb{Z}) \simeq C^*(\Gra(D); \mathbb{Z})$$ and so the map $i$ induces a cohomology isomorphism as desired.  
\end{proof} 

Proposition \ref{prop:diskequiv} simultaneously checks that our $\N (\mathrm{Disk}(\mathbb{C}))_{\mathrm{nu}}$-algebra $A$ satisfies the condition of Theorem \ref{thm:lcdisk} and identifies the underlying spectrum of the resulting non-unital $\mathbb{E}_2$-algebra as the fiber of the Beilinson-Drinfeld Grassmannian over a point of the Ran space.  This is precisely the filtered spectrum $\{ \Sigma^{\infty}_+ \Gra_{G,\leq k\mu}\}_{k \in \mathbb{Z}_{\geq 0}}$ determined by the affine Grassmannian and choice of coweight $\mu$.  

***insert some stuff about unitality***

This completes the proof of Proposition \ref{prop:schubertE2}.  
