
In the next section we will define the Mitchell--Richter Bott filtration on $\Omega SU(n)$.  This filtration is most naturally described in the language of algebraic geometry, and our aim in this section is to recall the relevant algebraic geometry and draw from it the topological consequences we will need.  None of the ideas in this section are original: the objects we study are due to Beilinson and Drinfeld \textbf{CITE}, and their translation into algebraic topology is due to Jacob Lurie \textbf{CITE}.  

Fix for the moment a smooth, reductive, affine algebraic group $G$ over $\mathbb{C}$.  Through the rest of the paper, we will be interested only in the cases $G=\mathbb{G}_m,SL_n,$ or $GL_n$, and so the reader may safely restrict his attention to those cases for concreteness.  As we will explain in detail, an algebro-geometric model for $\Omega G(\mathbb{C})$ is the \textit{affine Grassmannian} $Gr_G$.  The $\mathbb{E}_2$-algebra structure present on $\Omega G(\mathbb{C})$ is encoded by a more elaborate object, the \textit{Beilinson--Drinfeld Grassmannian}.  A good general reference for both of these objects is \cite{Zhu}, whose presentation we will more or less follow below.

We use $D$ to denote the formal disk $\text{Spec}(\mathbb{C}[[t]])$ and $D^*$ to denote the punctured disk $\text{Spec}(\mathbb{C}((t)))$.  For $R$ a $\mathbb{C}$-algebra, we use $D_R$ to denote $\text{Spec}(\mathbb{C}[[t]] \hat{\otimes} R)$ and $D^*_R$ to denote $\text{Spec}(\mathbb{C}((t)) \hat{\otimes} R)$.  We refer to a point in the space $G(D^*)=G(\mathbb{C}((t)))$ as an algebraic free (i.e., unbased) loop in $G$.  Such points are exactly automorphism of the trivial $G$-torsor $\mathcal{E}^0$ over $D^*$.

\begin{dfn}
The \textit{affine Grassmannian} $Gr_G$ of $G$ is the Ind-scheme with functor of points
$$Gr_G(R) = \{(\mathcal{E},\beta)\},\text{ where}$$
$\mathcal{E}$ is a $G$-torsor over $D_{R}$ and $\beta:\mathcal{E}|_{D^*_{R}} \cong \mathcal{E}^0_{D^*_{R}}$ is a trivialization over $D^*_{R}$.
\end{dfn}

The complex points $Gr_G(\mathbb{C})$ are a model for the topological space $\Omega G$ CITE.  These complex points are given by the homogeneous space CITE
$$G(\mathbb{C}((t)))/G(\mathbb{C}[[t]]),$$
which up to homotopy is the quotient of the free loop space on $G$ by the action of $G$.

The above functor of points is not representable by a scheme, but it is a filtered colimit of schemes $Gr_{G,\le \mu}$.  We will define below at least the complex points $Gr_{G, \le \mu}(\mathbb{C})$ of these schemes, which are specific topological subspaces of $Gr_G(\mathbb{C})$.  First, it is necessary to introduce a bit more notation.

Let $T$ denote a maximal torus inside $G$.  We use $\mathbb{X}^{\bullet}$ to denote the lattice of weights $\Hom(T,\mathbb{G}_m)$, and $\mathbb{X}_{\bullet}$ to denote the dual lattice of coweights.  Inside $\mathbb{X}^{\bullet}$ is the set $\Phi$ of roots.  We fix a particular Borel subgroup $B \subset G$, determining a choice of positive roots $\Phi^+ \subset \Phi$ and a semi-group of dominant coweights $\mathbb{X}^+_\bullet \subset \mathbb{X}_\bullet$.  There is a natural bijection
$$\mathbb{X}_{\bullet}^+ \cong G(\mathbb{C}[[t]])\backslash G(\mathbb{C}((t)))/G(\mathbb{C}[[t]])$$
of dominant coweights with the above double cosets.  Each coweight $\mu \in \Hom(\mathbb{G}_m,T)$ defines via the inclusion of $T$ into $G$ an element $t^{\mu}$ in $G(\mathbb{C}((t)))$.  There is a double-coset decomposition of the algebraic free loop space
$$
G(\mathbb{C}((t))) \cong \coprod_{\mu \in \mathbb{X}_{\bullet}^+} G(\mathbb{C}[[t]]) t^{\mu} G(\mathbb{C}[[t]]).
$$
Projecting onto the affine Grassmannian, one learns that the $G(\mathbb{C}[[t]])$-orbits of $Gr_G$ are indexed by $\mu \in \mathbb{X}_{\bullet}^+$.  We will use $Gr_{G,\le \mu}$ to denote the \textit{closure} of the orbit corresponding to $\mu$.  The closure $Gr_{G, \le \mu_1}$ contains $Gr_{G, \le \mu_2}$ if and only if $\mu_1-\mu_2$ is a sum of dominant coroots.  We call $\{Gr_{G,\le \mu}|\mu \in \mathbb{X}^+_{\bullet}\}$ the \textit{Schubert filtration} of $Gr_G$. 

\begin{exm} \label{sl2example}
Suppose $G=SL_2(\mathbb{C})$ with its usual maximal torus.  A coweight $\mu \in \mathbb{X}_\bullet$ consists of a pair $(a,b)$ of integers with $a+b=0$.  We choose a Borel so that a coweight is dominant if $a \ge b$.  The conjugation action of $SL_2(\mathbb{C})$ on $\Omega SL_2(\mathbb{C})$ has one orbit for each pair $(a,-a)$ with $a \ge 0$.  The orbit corresponding to $(a,-a)$ contains the loop $\mathbb{G}_m \rightarrow \Omega SL_2(\mathbb{C})$ given by
$$
t \mapsto \left( \begin{array}{cc} t^a & 0 \\ 0 & t^{-a}  \end{array} \right).
$$
The closure of the $(a,-a)$ orbit contains the $(b,-b)$ orbit if and only if $b \le a$.  To topologists, $\Omega SL_2(\mathbb{C}) \simeq \Omega \Sigma S^2$ is recognizable as the free $\mathbb{A}_\infty$-algebra on the pointed space $S^2$.  In particular, $Gr_{SL_2}(\mathbb{C})$ is naturally equipped with the James filtration by word length.  The closure of the $(a,-a)$ orbit, denoted $Gr_{SL_2,\le (a,-a)}(\mathbb{C})$, turns out to be the $(2a)$th component of the James filtration.

Thus, the Schubert filtration is strictly coarser than the James filtration.  In particular, the $S^2$ that appears as the first James filtered piece of $\Omega SL_2(\mathbb{C})$ is not closed under the $SL_2(\mathbb{C})$ conjugation action.  Only the collection of words of length $2$ or less is closed under the $SL_2(\mathbb{C})$ action.

The Bott filtration on $\Omega SL_2(\mathbb{C})$ corresponds to the James filtration on $\Omega S^3$, and in particular is \textbf{not} given by the Schubert filtration on $Gr_{SL_2}(\mathbb{C})$.  In Section CITE, we will explain how to obtain the Bott filtration from the Schubert filtration, while in the remainder of this section we will explain how the Schubert filtration interacts with the $\mathbb{E}_2$-algebra structure on $\Omega SL_2(\mathbb{C})$.
\end{exm}

The $\mathbb{E}_2$-algebra structure on $\Omega G$ is elegantly encoded in algebraic geometry through the notion of the Beilinson--Drinfeld Grassmannian:

\begin{dfn}
The \textit{Ran space} $\text{Ran}_{\mathbb{A}^1}$ is the presheaf that assigns to every $\mathbb{C}$-algebra $R$ the set of non-empty finite subsets of $\text{Spec}(R) \times \mathbb{A}^1$ over $\text{Spec}(R)$.   The Beilinson--Drinfeld Grassmannian is the presheaf $Gr_{G,\text{Ran}}$ that assigns to each $\mathbb{C}$-algebra $R$ the set of triplets $(x,\mathcal{E},\beta)$, where $x \in \text{Ran}(\mathbb{A}^1)(R)$, $\mathcal{E}$ is a $G$-torsor on $\mathbb{A}^1 \times \text{Spec}(R)$, and $\beta$ is a trivialization of $\mathcal{E}$ away from the graph of $x$ in $\text{Spec}(R) \times \mathbb{A}^1$.
\end{dfn}

The Beilinson--Drinfeld Grassmannian is naturally fibered over the Ran space.  Again, we shall be primarily interested in complex points.  A point $x$ in $\text{Ran}_{\mathbb{A}^1}(\mathbb{C})$ consists of a non-empty finite subset $I \subset \mathbb{C}$ of points in $\mathbb{C}$.  The fiber of the Beilinson--Drinfeld Grassmannian $Gr_{G,\text{Ran}}(\mathbb{C})$ over $x$ is the moduli of $G$-bundles on $\mathbb{A}^1$ equipped with a trivialization away from the points in $I$.  This fiber is non-canonically isomorphic to the product of $|I|$ copies of $Gr_G(\mathbb{C})$.

As we will explain, the $\mathbb{E}_2$ multiplication on $Gr_G(\mathbb{C})$ is encoded by degeneration of fibers in $Gr_{G,\text{Ran}}(\mathbb{C})$ as points collide in $\mathbb{A}^1$.  First, we will explain how to filter $Gr_{G,\text{Ran}}(\mathbb{C})$ by compact spaces, just as one filters $Gr_{G}(\mathbb{C})$ by the $Gr_{G, \le \mu}(\mathbb{C})$.

A point in .


