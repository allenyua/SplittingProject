\documentclass[reqno, oneside]{amsart}
\usepackage{etex}
\usepackage{hyperref}

%\usepackage[export]{adjustbox}
%\usepackage[dvips]{pict2e}
%\usepackage{amsmath,amsthm}
\usepackage{comment}
\usepackage{amsfonts, fullpage, fancyhdr, mathtools, qtree, amsmath, tipa, amssymb, hyperref, url, amsthm, subfigure, xy, tikz-cd, verbatim}

\usepackage[color=cyan!40]{todonotes}

\usepackage{appendix}

\usepackage[a4paper]{geometry}


%----------------------------------------------------------------------%

%\renewcommand{\appendixname}{Example}
%\swapnumbers

\theoremstyle{definition}
\newtheorem{nul}{}[section]
\newtheorem{dfn}[nul]{Definition}
\newtheorem{axm}[nul]{Axiom}
\newtheorem{rmk}[nul]{Remark}
\newtheorem{term}[nul]{Terminology}
\newtheorem{cnstr}[nul]{Construction}
\newtheorem{ntn}[nul]{Notation}
\newtheorem{exm}[nul]{Example}
\newtheorem{obs}[nul]{Observation}
\newtheorem{ctrexm}[nul]{Counterexample}
\newtheorem{rec}[nul]{Recollection}
\newtheorem{exr}[nul]{Exercise}
\newtheorem{wrn}[nul]{Warning}
\newtheorem*{dfn*}{Definition}
\newtheorem*{axm*}{Axiom}
\newtheorem*{ntn*}{Notation}
\newtheorem*{exm*}{Example}
\newtheorem*{exr*}{Exercise}
\newtheorem*{int*}{Intuition}
\newtheorem*{qst*}{Question}


\theoremstyle{plain}
\newtheorem{sch}[nul]{Scholium}
\newtheorem{claim}[nul]{Claim}
\newtheorem{thm}[nul]{Theorem}
\newtheorem{prop}[nul]{Proposition}
\newtheorem{lem}[nul]{Lemma}
\newtheorem{var}[nul]{Variant}
\newtheorem{sublem}{Lemma}[nul]
\newtheorem{por}[nul]{Porism}
\newtheorem{cnj}[nul]{Conjecture}
\newtheorem{cor}{Corollary}[nul]
\newtheorem*{thm*}{Theorem}
\newtheorem*{prop*}{Proposition}
\newtheorem*{cor*}{Corollary}
\newtheorem*{lem*}{Lemma}
\newtheorem*{cnj*}{Conjecture}


%----------------------------------------------------------------------%

\DeclareMathOperator{\Aut}{\text{Aut}}
\DeclareMathOperator{\Tr}{\text{Tr}}
\DeclareMathOperator{\Res}{\text{Res}}
\DeclareMathOperator{\im}{\text{im}}
\DeclareMathOperator*{\colim}{\text{colim}}
\DeclareMathOperator{\Map}{\text{Map}}
\DeclareMathOperator{\cofiber}{\text{cofiber}}
\DeclareMathOperator{\fiber}{\text{fiber}}
\DeclareMathOperator{\Hom}{\text{Hom}}
\DeclareMathOperator{\Skel}{\text{Skel}}
\DeclareMathOperator*{\hocolim}{\text{hocolim}}
\DeclareMathOperator*{\holim}{\text{holim}}
\DeclareMathOperator{\smsh}{\wedge}



\DeclareMathOperator{\C}{\mathcal{C}}
\DeclareMathOperator{\D}{\mathcal{D}}
\DeclareMathOperator{\CP}{\mathbb{CP}}
\DeclareMathOperator{\Z}{\mathbb{Z}}
\DeclareMathOperator{\E}{\mathbb{E}}
\DeclareMathOperator{\mE}{\mathcal{E}}
\DeclareMathOperator{\Q}{\mathbb{Q}}
\DeclareMathOperator{\m}{\mathfrak{m}}
\DeclareMathOperator{\G}{\mathbb{G}}
\DeclareMathOperator{\F}{\mathbb{F}}
\DeclareMathOperator{\cG}{\mathcal{G}}
\DeclareMathOperator{\cF}{\mathcal{F}}
\DeclareMathOperator{\cS}{\mathcal{S}}
\DeclareMathOperator{\Ring}{\textbf{Ring}}
\DeclareMathOperator{\Aff}{\textbf{Aff}}
\DeclareMathOperator{\Spec}{\text{Spec}}
\DeclareMathOperator{\Poly}{\text{Poly}}
\DeclareMathOperator{\Ext}{\text{Ext}}
\DeclareMathOperator{\nil}{\text{nil}}
\DeclareMathOperator{\fin}{\text{fin}}
\DeclareMathOperator{\Gr}{\textbf{Gr}}
\DeclareMathOperator{\Fil}{\textbf{Fil}}
\DeclareMathOperator{\Fun}{\text{Fun}}
\DeclareMathOperator{\Alg}{\text{Alg}}
\DeclareMathOperator{\Sp}{\text{Sp}}
\DeclareMathOperator{\Spaces}{\mathcal{S}}
\DeclareMathOperator{\J}{\mathcal{J}}
\DeclareMathOperator{\Cofil}{\textbf{Cofil}}

%----------------------------------------------------------------------%

\usepackage{tikz}
\usetikzlibrary{matrix,arrows,decorations}
\usepackage{tikz-cd}

\usepackage{adjustbox}

%----------------------------------------------------------------------%

\hyphenation{Mack-ey mon-oid-al Wald-hau-sen}

%----------------------------------------------------------------------%
%----------------------------------------------------------------------%


\begin{document}

\title{Multiplicative Structure in the Stable Splitting of $\Omega SU(n)$}
\author{Jeremy Hahn and Allen Yuan}

\begin{abstract}
The space of based loops in $SU(n)$, also known as the affine Grassmannian of $SL_n(\mathbb{C})$, admits an $\mathbb{E}_2$ or fusion product.  Work of Mitchell and Richter proves that this based loop space stably splits as an infinite wedge sum.  We prove that the Mitchell--Richter splitting is $\mathbb{A}_\infty$, but not $\mathbb{E}_2$.  Furthermore, we show that the splitting becomes $\mathbb{E}_2$ after base-change to complex cobordism.  Our proofs involve on the one hand an analysis of the multiplicative properties of Weiss calculus, and on the other a use of Beilinson--Drinfeld Grassmannians to verify a conjecture of Mahowald and Richter.
\end{abstract}



%----------------------------------------------------------------------%
%----------------------------------------------------------------------%


\setcounter{tocdepth}{1}
\maketitle

\tableofcontents


%----------------------------------------------------------------------%

\section{Introduction}

We study the homotopy type of the affine Grassmannian of $SL_n(\mathbb{C})$, which is equivalent to the space $\Omega SU(n)$ of based loops in $SU(n)$.  There are essentially two multiplications on this homotopy type, one arising from the composition of loops and the other from the group multiplication on $SL_n(\mathbb{C})$.  Together, these two multiplications interact to give $\Omega SU(n)$ the structure of an $\mathbb{E}_2$ or fusion algebra.  In geometric representation theory, this structure is witnessed by the existence of the Beilinson--Drinfeld Grassmannian.

Using either of the above (homotopy equivalent) products, it is possible to make $H_*(\Omega SU(n);\mathbb{Z})$ into a graded ring.  To describe this ring, let us first name some of its elements.  For each one-dimensional subspace $V \subset \mathbb{C}^n$, there is a loop $\lambda_V:S^1 \rightarrow U(n)$ given by the formula
$$\lambda_V(z)=\left( \begin{array}{cc} z & 0 \\ 0 & I \end{array} \right),$$
with the matrix presented in terms of the decomposition $\mathbb{C}^n \cong V \oplus V^{\perp}$.  Fixing a particular line $W \subset \mathbb{C}^n$, the construction $V \mapsto \lambda_W^{-1} \cdot \lambda_V$ defines a well-known map
$$\mathbb{CP}^{n-1} \rightarrow \Omega SU(n).$$
Let $b_1,b_2,...,b_{n-1}$, $|b_i|=2i$, denote the images in $H_*(\Omega SU(n);\mathbb{Z})$ of the generators of $H_*(\mathbb{CP}^{n-1};\mathbb{Z})$.  It is a result of Bott \cite{Bott} that
$$H_*(\Omega SU(n);\mathbb{Z}) \cong \mathbb{Z}[b_1,b_2,\cdots],$$
with the latter denoting the polynomial algebra on the classes $b_i$.

Notice, in particular, that $H_*(\Omega SU(n);\mathbb{Z})$ is naturally a \textit{bigraded} ring, one grading being given by $*$ and the other by assigning each $b_i$ degree $1$.  Mahowald observed that the action of the Steenrod algebra on $H_*(\Omega SU(n);\mathbb{F}_2)$ preserves this second degree, and he conjectured a geometric splitting to be responsible.  Indeed, it was eventually proven by Mitchell and Richter \cite[Theorem 2.1]{CrabbMitchell}, that the suspension spectrum $\Sigma^{\infty} \Omega SU(n)$ splits as an infinite wedge sum:
$$\Sigma^{\infty}_+ \Omega SU(n) \simeq \mathbb{S} \vee \Sigma^{\infty} \mathbb{CP}^{n-1} \vee \cdots.$$

In order to prove this splitting, Mitchell \cite{MitchellSU(n)} (and, independently, Segal \cite{Segal}) first constructed a filtration of the space $\Omega SU(n)$.  Following Mitchell, we name this the \textit{Bott filtration} of $\Omega SU(n)$.  The first filtered piece is given by the above map $\mathbb{CP}^{n-1} \rightarrow \Omega SU(n)$, and the theorem of Mitchell and Richter is that the filtration stably splits.  The construction of the Bott filtration is somewhat involved, and we review it in Section \ref{sec:MRFil}--it is a subfiltration of the Bruhat ordering on (closures of) Iwahori orbits.

In Section \ref{sec:FilGra}, we review the symmetric monoidal structures on the ($\infty$)-categories of filtered and graded spectra.  This allows us to properly state our first main theorem, proven in Section \ref{sec:MRFil}:

\begin{thm} \label{thm:BottIsAoo}
The suspension of the Bott filtration 
$$\mathbb{S} \longrightarrow \Sigma_+^{\infty} \mathbb{CP}^{n-1} \simeq \Sigma_+^{\infty} F_{n,1} \longrightarrow \Sigma_+^{\infty} F_{n,2} \longrightarrow \cdots \longrightarrow \Sigma^{\infty}_+ \Omega SU(n).$$
is an $\mathbb{A}_\infty$-algebra object in filtered spectra.
\end{thm}

\begin{rmk}
The Bott filtration is multiplicative before suspension, but for technical reasons we prefer to phrase our results in terms of filtered spectra instead of filtered spaces.
\end{rmk}

\begin{qst}
Is the Bott filtration an $\mathbb{E}_2$ filtration?  We do not know the answer--for some thoughts about the problem, see Remark \ref{rmk:E2fil}.
\end{qst}

The proof of Theorem \ref{thm:BottIsAoo} is fairly straightforward, once given access to the sophisticated machinery behind the Beilinson--Drinfeld Grassmannian.  For example, we will explain in Section \ref{sec:MRFil} that this machinery immediately dispenses with a conjecture of Mahowald and Richter \cite{MahowaldRichter}.  Nonetheless, there are some subtleties involved, and it is these subtleties that prevent us from determining if the Bott filtration is $\mathbb{E}_2$.  The problem is readily visible in the case $n=\infty$:

\begin{exm}
The limiting case of the Bott filtration of $\Omega SU(n)$ as $n$ tends to $\infty$ is the filtration
$$* \longrightarrow BU(1) \longrightarrow BU(2) \longrightarrow BU(3) \longrightarrow \cdots \longrightarrow BU \simeq \Omega SU.$$
It is easy to see that $\coprod BU(n)$ is a graded $\mathbb{E}_2$-algebra in spaces (in fact, it is a graded $\mathbb{E}_\infty$-algebra, being the nerve of the category of vector spaces).  However, the filtered object is much more subtle.  For example, the squares
$$
\begin{tikzcd}
BU(i) \times BU(j) \arrow{d} \arrow{r} & BU(i) \times BU(j+1) \arrow{d} \\
BU(i+1) \times BU(j) \arrow{r} & BU(i+1) \times BU(j+1)
\end{tikzcd}
$$
do not commute on the nose, but only up to non-canonical homotopy.
\end{exm}

The rest of the paper is concerned with the stable splitting of this Bott filtration.  Our main results are as follows:

\begin{thm} \label{thm:MainAoo}
As an $\mathbb{A}_\infty$-algebra object in filtered spectra, the Bott filtration of $\Sigma^{\infty}_+ \Omega SU(n)$ is equivalent to its associated graded.
\end{thm}

\begin{cor}
For any homology theory $E$, $E_*(\Omega SU(n))$ is a bigraded ring.  One grading is given by $*$, and the other by the associated graded of the Bott filtration.
\end{cor}

\begin{thm} \label{thm:MainObstruction}
Suppose $n \ge 4$.  If the Bott filtration of $\Sigma^{\infty}_+ \Omega SU(n)$ may be made into an $\mathbb{E}_2$-algebra object in filtered spectra, then it is \textbf{not} equivalent to its $\mathbb{E}_2$ associated graded.
\end{thm}

\begin{thm} \label{thm:MainMUE2}
Let $MU$ denote the $\mathbb{E}_\infty$-ring spectrum of complex bordism, and let $\text{gr}(\Sigma^{\infty}_+\{F_{n,k}\})$ denote the associated graded of the Bott filtration of $\Sigma^{\infty}_+ \Omega SU(n)$.  Then
\begin{enumerate}
\item There exists a graded $\mathbb{E}_2$-algebra structure on the graded spectrum $\text{gr}(\Sigma^{\infty}_+ \{F_{n,k}\})$ that extends the canonical graded $\mathbb{A}_\infty$-algebra structure.

\item For any $\mathbb{E}_2$-algebra structure on the underlying (ungraded) $\mathbb{A}_\infty$-ring $\text{gr}(\Sigma^{\infty}_+\{F_{n,k}\})$, there is an equivalence of $\mathbb{E}_2$-$MU$-algebras
$$MU \smsh \Sigma^{\infty}_+ \Omega SU(n) \simeq MU \smsh \text{gr}(\Sigma^{\infty}_+\{F_{n,k}\}).$$
\end{enumerate}
\end{thm}

\begin{rmk}
There exist exotic $\mathbb{E}_2$-algebra structures on $gr(\Sigma^{\infty}_+ \{F_{n,k}\})$ before smashing with $MU$.  For example, as $n$ tends to infinity we recover the Snaith splitting \cite{SnaithBook}
$$\Sigma^{\infty}_+ BU \simeq \bigvee_n MU(n),$$
where $MU(n)$ is the Thom spectrum of the canonical bundle over $BU(n)$.  The $\mathbb{E}_2$-ring structure arising from the Thom spectrum construction applied to
$$\coprod BU(n) \stackrel{J}{\longrightarrow} Pic(\mathbb{S})$$
does not agree with the $\mathbb{E}_2$-ring structure on $\Sigma^{\infty}_+ BU$ that arises from the double loop space structure on $BU$.
\textbf{Allen are we sure these don't agree?}
\end{rmk}

The final result above, regarding $MU$-module spectra, can be seen as a once-looped analogue of work of Kitchloo \cite{Kitchloo}.   Kitchloo studied a splitting, due to Miller \cite{MillerSplitting}, of $\Sigma^{\infty}_+ SU(n)$.  His theorem is that, \textit{for complex-oriented $E$}, the corresponding direct sum decomposition of $E_*(SU(n))$ is multiplicative.

Our proof of Theorem \ref{thm:MainMUE2} is by obstruction theory.  We show in Section \ref{sec:MUE2} that all obstructions to an $\mathbb{E}_2$-equivalence vanish.  On the other hand, we prove Theorem \ref{thm:MainObstruction} by explicitly calculating a non-zero obstruction in Section \ref{sec:Obstruction}.

It remains to discuss Theorem \ref{thm:MainAoo}, the $\mathbb{A}_\infty$ splitting.

\textbf{SOMETHING ABOUT STIEFEL MANIFOLDS}

Open questions regarding natural extensions of our work:

\textbf{What is the structure of the equivariant splitting?}

\textbf{What is the proper motivic analogue of our result?}



%%Might be worth making sure that at least in the statements of the theorems/results, we use "infty-category" instead of "category

%acknowledge
%1. Jacob, Mike
%2. arone? akhil, denis?, dyang?, justin? 

%Notations, todo?
%\Sp for spectra, S for spaces? should also say that by default, these are given \smash, and \times.


\section{Filtered and Graded Ring Spectra} \label{sec:FilGra}


It will be important for us to have a precise language for discussing filtered and graded spectra, what it means to be split, what it means to take associated graded, and the multiplicative aspects of these constructions. Here we review a framework from \cite{LurieRot} for studying graded and filtered objects.  The reader is referred to \cite{LurieRot} for a more thorough treatment and all proofs.  

%Let $\C$ be a symmetric monoidal stable $\infty$-category which admits filtered colimits.  
\subsection{First definitions}
Let $\D$ be an $\infty$-category which we will regard as the diagram category.  Our filtered objects will be valued in the functor category $\Sp^{\D}.$  This will be no more difficult than just ordinary spectra because limits, colimits, and smash products will be considered pointwise; in any case, we will refer to objects of $\Sp^{\D}$ as functors or simply as spectra.  

Denote by $\Z_{\geq 0}$ the poset of non-negative integers thought of as an ordinary category where $\Hom(a,b)$ is a singleton if $a\leq b$, and empty otherwise.  Denote by $\Z_{\geq 0}^{ds}$ the corresponding discrete category.  We will implicitly take nerves to obtain $\infty$-categories which will serve as the indexing sets for filtered and graded spectra.  The reader is warned that our numbering conventions are opposite the ones in \cite{LurieRot}.

\begin{dfn} 
Let $\Gr(\Sp^{\D})$ denote the functor category $\Fun(\Z_{\geq 0}^{ds}, \Sp^{\D}).$  We shall refer to $\Gr(\Sp^{\D})$ as the category of graded objects in $\Sp^{\D}$.  Its objects can be thought of as sequences $X_0, X_1,X_2,\cdots \in \Sp^{\D}$.
\end{dfn}

\begin{dfn} 
Let $\Fil(\Sp^{\D})$ denote the functor category $\Fun(\Z_{\geq 0}, \Sp^{\D}).$  We shall refer to $\Fil(\Sp^{\D})$ as the category of filtered objects in $\Sp^{\D}$.  Its objects can be thought of as sequences $Y_0\to Y_1\to Y_2 \to \cdots \in \Sp^{\D}$ filtering $\colim_i Y_i$.  
\end{dfn}


The obvious map $\Z_{\geq 0}^{ds} \to \Z_{\geq 0}$ induces a restriction functor $\text{res}: \Fil(\Sp^{\D}) \to \Gr(\Sp^{\D})$ which can be thought of as forgetting the maps in the filtered object.  The restriction fits into an adjunction  
$$I:\Gr(\Sp^{\D}) \xrightleftharpoons{\quad} \Fil(\Sp^{\D}) : \text{res}$$
where the left adjoint $I: \Gr(\Sp^{\D}) \to \Fil(\Sp^{\D})$ is given by left Kan extension.  The functor $I$ can be described explicitly as taking a graded object $X_0,X_1,X_2,\cdots$ to the filtered object $$I(X_0, X_1, \cdots) = (X_0\to X_0\oplus X_1\to X_0 \oplus X_1\oplus X_2\to \cdots).$$   

Inverse to this, there is an associated graded functor $\text{gr }: \Fil(\Sp^{\D}) \to \Gr(\Sp^{\D})$ such that the composite $\text{gr }\circ I : \Gr(\Sp^{\D}) \to \Gr(\Sp^{\D})$ is an equivalence.   This can be thought of pointwise by the formula $$\text{gr}(X_0\to X_1\to X_2\to \cdots) = X_0, X_1/X_0, X_2/X_1, \cdots.$$

As the names suggest, one may recover from a filtered or graded functor the underlying object.  For filtered objects, this is a functor $$\colim : \Fil(\Sp^{\D}) \to \Sp^{\D}$$ given by Kan extending along $\Z_{\geq 0}\to *.$   It can be thought of as taking the colimit.  For graded objects, the underlying object is simply the direct sum of all the graded piece. We will systematically abuse notation by conflating a graded spectrum with its underlying spectrum; however, when there is any ambiguity, we will specify whether we are referring to the graded object or the underlying.

%, and will be written $$\bigvee : \Gr(\Sp^{\D}) \to \Sp^{\D}.$$



\subsection{Monoidal structures}\label{sect:monoidal}
We now begin studying the monoidal structures on graded and filtered spectra.  We confine ourselves to a basic discussion here, leaving a more technical discussion for Appendix \ref{app:day}.

%\cite[Corollary 2.3.9]{LurieRot}
By \cite{Glasman} or \cite[Example 2.2.6.17]{HA}, the categories $\Gr(\Sp)$ and $\Fil(\Sp)$ may be given symmetric monoidal structures via the Day convolution.  Then, via the identifications $\Gr(\Sp^{\D}) = \Gr(\Sp)^{\D}$ and $\Fil(\Sp^{\D}) = \Fil(\Sp)^{\D}$, the categories $\Gr(\Sp^{\D})$ and $\Fil(\Sp^{\D})$ may be given symmetric monoidal structures pointwise on $\D$.  In both cases, we denote the resulting operation by $\otimes$. Explicitly, the filtered tensor product $$\left(X_0 \longrightarrow X_1 \longrightarrow X_2 \longrightarrow \cdots \right) \otimes \left(Y_0 \longrightarrow Y_1 \longrightarrow Y_2 \longrightarrow \cdots \right)$$
of two filtered spectra is computed as

\begin{center}
$X_0 \smsh Y_0 \longrightarrow \colim $
\adjustbox{scale=0.7} 
{$ \left(\begin{tikzcd} X_0 \smsh Y_1 \\  X_0 \smsh Y_0 \arrow{u} \arrow{r} & X_1 \smsh Y_0 \end{tikzcd} \right) $} 
$\longrightarrow \colim$
\adjustbox{scale=0.7} {$ \left( \begin{tikzcd} X_0 \smsh Y_2 \\ X_0 \smsh Y_1 \arrow{r} \arrow{u} & X_1 \smsh Y_1  \\ X_0 \smsh Y_0 \arrow{r} \arrow{u} & X_1 \smsh Y_0 \arrow{u} \arrow{r} & X_2 \smsh Y_0 \end{tikzcd} \right) $}
$\longrightarrow \cdots.$
\end{center}

For graded spectra, the analogous formula is:

$$(A_0,A_1,A_2,\cdots) \otimes (B_0,B_1,B_2,\cdots) \simeq \left( A_0 \smsh B_0, (A_1 \smsh B_0) \vee (A_0 \smsh B_1), \cdots, \bigvee_{i+j=n} A_i \smsh B_j, \cdots \right).$$


The unit $\mathbb{S}^{gr}_{\D}$ of $\otimes$ in $\Gr(\Sp^{\D})$ is the constant diagram at $S^0$ in degree 0 and $*$ otherwise; the unit $\mathbb{S}^{fil}_{\D}$ in $\Fil(\Sp^{\D})$ is $I\mathbb{S}^{gr}_{\D}.$  We may then talk about $\E_n$-algebras in $\Gr(\Sp^{\D})$ and $\Fil(\Sp^{\D})$.  


The functors $I$ and $\text{gr}$ can be given symmetric monoidal structures such that the composite $\text{gr}\circ I : \Gr(\Sp^{\D}) \to \Gr(\Sp^{\D})$ is a symmetric monoidal equivalence by \cite[Proposition 3.2.1]{LurieRot}.  It follows in particular that they extend to functors between the categories of $\E_n$-algebras in $\Gr(\Sp^{\D})$ and $\Fil(\Sp^{\D})$.  Thus, given an $\E_n$-algebra $Y$ in filtered spectra, we obtain a canonical $\E_n$ structure on its associated graded $\text{gr}(Y).$  Conversely, given $X\in \Alg_{\E_n}(\Gr(\Sp^{\D}))$, we obtain $IX\in \Alg_{\E_n}(\Fil(\Sp^{\D})).$  

\begin{dfn}
A filtered $\mathbb{E}_n$-algebra $X\in \Alg_{\E_n}(\Fil(\Sp^{\D}))$ is called \emph{$\E_n$-split} if there exists some $Y \in \Alg_{\E_n}(\Gr(\Sp^{\D}))$ with an equivalence $X \simeq IY$ in $\Alg_{\E_n}(\Fil(\Sp^{\D}))$.  
\end{dfn}

Given an $\E_n$-split filtered spectrum $X$, we can recover the underlying graded spectrum by taking the associated graded.  

\begin{exm}\label{exm:snaith}
Let $X\in \cS$ be connected and $n>0$.  The Snaith splitting can be seen as giving $\Sigma^{\infty}_+ \Omega^n \Sigma^n X$ the structure of a split $\E_n$ filtered space.  To see this, first note that there is a commutative square of $\infty$-categories:
$$
\begin{tikzcd}
\Gr(\Sp) \arrow[d]\arrow[r,"F_{\E_n}"]&  \Alg_{\E_n}(\Gr(\Sp)) \arrow[d]\\
\Sp \arrow[r,"F_{\E_n}"] & \Alg_{\E_n}(\Sp) 
\end{tikzcd}
$$
where the downward arrows forget the grading and the rightward arrows take free algebras.  This is essentially because the formula for the free graded $\E_n$ algebra on a graded spectrum is the same as in the ungraded case.  Now consider $\Sigma^{\infty}X[1]$, the graded space with $\Sigma^{\infty} X$ in degree 1.  We then get a map by freeness $$\bigvee F_{\E_n} \Sigma^{\infty} X[1] \simeq  F_{\E_n} \Sigma^{\infty} X \to \Sigma^{\infty}_+ \Omega^n \Sigma^n X$$ which is an equivalence by the Snaith splitting.  As a result, the filtered spectrum $I(F_{\E_n} \Sigma^{\infty} X[1])$ exhibits $\Sigma^{\infty}_+ \Omega^n \Sigma^n X$ as an $\E_n$-split $\E_n$ filtered spectrum.  
\end{exm}

In this paper, we will be interested in when a given $\E_n$ filtered spectrum is $\E_n$-split.  Disregarding the multiplicative structure, a filtered spectrum $$X_0\longrightarrow X_1 \longrightarrow X_2 \longrightarrow \cdots ,$$ will split if and only if there are maps going the other way: $$X_0 \longleftarrow X_1 \longleftarrow X_2 \longleftarrow \cdots,$$ with the property that the relevant composites are equivalences.   To systematically talk about these backwards maps, we need the following definition:

\begin{dfn} Let $\Cofil(\Sp^{\D})$ denote the functor category $\Fun(\Z_{\geq 0}^{op}, \Sp^{\D}).$  We shall refer to $\Cofil(\Sp^{\D})$ as the category of cofiltered objects in $\Sp^{\D}$.  Its objects can be thought of as towers of functors $Y_0\leftarrow Y_1\leftarrow Y_2 \leftarrow \cdots \in \Sp^{\D}$.
\end{dfn}

In Appendix \ref{app:day}, we show how to give $\Cofil(\Sp^{\D})$ the structure of a symmetric monoidal $\infty$-category.  One might then correctly surmise that producing a multiplicatively structured splitting involves the multiplicativity of this opposite filtration.  In particular, we have the following criterion, which we prove in Appendix \ref{app:SplittingMachine} (we now switch from $\Sp^{\D}$ to $\Sp$ for ease of notation):

\begin{thm}\label{thm:SplitMachine}%maybe this is an iff
Let $X\in \Alg_{\E_n}(\Fil(\Sp))$ be an $\E_n$ filtered spectrum.  Suppose there exists an $\E_n$ cofiltered spectrum $Y\in \Alg_{\E_n}(\Cofil(\Sp))$ with the following two properties:
\begin{enumerate}
\item There is an equivalence $\mathrm{colim } X \simeq \lim Y$ of $\E_n$-algebras in spectra.
\item The resulting natural maps $X_i \to Y_i$ are equivalences.  
\end{enumerate}
Then, the filtered spectrum $X$ is $\E_n$-split.
\end{thm}




As a final remark, we note that the category of filtered spectra may be recovered as a module category inside the category of graded spectra.  Specifically, consider $A=\Sigma^{\infty}_+ \mathbb{Z}^{ds}_{\ge 0}$, the suspension of the nerve of the symmetric monoidal category $\mathbb{Z}^{ds}_{\ge 0}$, as an $\mathbb{E}_\infty$-algebra in $\Gr(\Sp)$.  The spectrum underlying $A$ is an infinite wedge of copies of $\mathbb{S}^0$.  Then the following may be shown by the argument in \cite[Proposition 3.1.6]{LurieRot}:

\begin{lem} \label{lem:FilAsGrMod}
There is an equivalence of symmetric monoidal categories
$$\Fil(\Sp) \stackrel{\simeq}{\longrightarrow} \textbf{Mod}_{A}(\Gr(\Sp)),$$
given by the forgetful functor.
\end{lem}




%%%Maybe it's worth making a remark about what these mean in terms of power operations, or say, the spectral sequence associated to the filtered object; maybe there's some sort of power operation on the $E_2$ page and you can say something about it...






\section{The Bott filtration on \texorpdfstring{$\Omega SU(n)$}{OmegaSU(n)}} \label{sec:MRFil}

In this section we recall and study the Bott filtration \cite{MitchellLoopGroup} on $\Omega SU(n)$.  Our main result is that the Bott filtration is at least $\mathbb{A}_\infty$, meaning in particular that its suspension is an $\mathbb{A}_\infty$-filtered spectrum in the sense of Section \ref{sec:FilGra}.

In the previous Section \ref{sec:Schubert} we learned very general techniques to construct $\mathbb{E}_2$-filtrations.  To see why these techniques do not directly produce the Bott filtration, it is helpful to recall the very instructive Example \ref{sl2example}.  Let us briefly summarize our previous discussion of that example:

\begin{rmk}
Consider $Gr_{SL_2}(\mathbb{C}) \simeq \Omega SU(2) \simeq \Omega S^3$.  This has a natural James filtration
$$* \longrightarrow J_1(S^2) \longrightarrow J_2(S^2) \longrightarrow \cdots \longrightarrow \Omega S^3,$$
which happens to be a special case of the Bott filtration we will define below.
The discussion of Section \ref{sec:Schubert} allows us to prove that the coarsened filtration
$$* \longrightarrow J_2(S^2) \longrightarrow J_4(S^2) \longrightarrow \cdots \longrightarrow \Omega S^3$$
has suspension a filtered $\mathbb{E}_2$-algebra.  However, we would like to understand the James filtration, rather than its coarsening, and prove that it is at least an $\mathbb{A}_\infty$-filtration.
\end{rmk}

We will follow Segal \cite{Segal} and access the Bott filtration on $Gr_{SL_n}(\mathbb{C})$ in a somewhat indirect manner, by considering not $Gr_{SL_n}(\mathbb{C})$ but $Gr_{GL_n}(\mathbb{C})$.

We consider $GL_n(\mathbb{C})$ with its usual maximal torus, and choose a Borel such that dominant coweights $\mu$, in bijection with integer sequences $(a_1,a_2,\cdots,a_n)$ such that $a_1 \ge a_2 \ge \cdots \ge a_n$, are given by maps
$$t \stackrel{\mu}{\mapsto} \left( \begin{array}{cccc} t^{a_1} & 0 & \cdots & 0 \\ 0 & t^{a_2} & \cdots & 0 \\ \vdots & \vdots & \ddots & \vdots \\ 0 & 0 & \cdots & t^{a_n} \end{array} \right).$$

\begin{dfn}
Consider the affine Grassmannian $Gr_{GL_n}(\mathbb{C})$.  We denote by $F_{n,k}$ the subset of $Gr_{GL_n}$ that is the closure of the $GL_n(\mathbb{C}[[t]])$ orbit containing:
$$t \mapsto \left( \begin{array}{cccc} t^k & 0 & \cdots & 0 \\ 0 & 1 & \cdots & 0 \\ \vdots & \vdots & \ddots & \vdots \\ 0 & 0 & \cdots & 1 \end{array} \right).$$
Using the language of Section \ref{sec:Schubert}, define for each $k\geq 0$
$$F_{n,k}  :=  \Gra_{GL_n, \leq (k,0,0,\cdots)}(\mathbb{C}).$$
\end{dfn}

The following lemma, affirming a conjecture of Mahowald and Richter, is then an immediate corollary of Theorem \ref{thm:schubertE2} and Remark \ref{rmk:choosespectra}:

\begin{lem} [Conjecture of Mahowald--Richter \cite{MahowaldRichter}] 
The inclusion 
$$\coprod_k F_{n,k} \subset \Omega GL_n(\mathbb{C})$$
may be made into a map of $\mathbb{E}_2$-algebras.  The suspension $$\Sigma^{\infty}_+ \coprod_k F_{n,k}$$ is a graded $\mathbb{E}_2$-algebra in the sense of Section \ref{sec:FilGra}.
\end{lem}

As explained by Segal \cite{Segal}, the coproduct $\coprod_k F_{n,k}$ may be viewed as the subspace of loops in $U(n)$ `of positive winding number.'  The $k$th piece $F_{n,k}$ consists of loops of winding number exactly $k$, and the group completion of $\coprod_k F_{n,k}$ is $\Omega U(n)$.

\begin{exm}
For any $n$, $F_{n,1}$ is equivalent to $\mathbb{CP}^{n-1}$.  The space $F_{2,k}$ is non-canonically homeomorphic to the $k$th stage of the James filtration of $\Omega S^3$, consisting of all words of length $\le k$.
\end{exm}

It is not at all obvious from the above construction that there should exist maps $F_{n,k} \rightarrow F_{n,k+1}$.  To make such maps requires some way of identifying the various connected components of $\Omega U(n)$, each of which is individually equivalent to $\Omega SU(n)$.  Following Segal \cite[pg. 3--4]{Segal}, one makes this identification by multiplying by powers of 
$$\lambda = \left( \begin{array}{cccc} t & 0 & \cdots & 0 \\ 0 & 1 & \cdots & 0 \\ \vdots & \vdots & \ddots & \vdots \\ 0 & 0 & \cdots & 1 \end{array} \right).$$
In other words, there is a map from the space of loops of winding number $k$ to loops of winding number $0$ given by multiplication by $\lambda^{-k}$.

\begin{dfn}
The Bott filtration on $\Omega SL_n(\mathbb{C})$ is the filtration with $k$th piece given by $\lambda^{-k} F_{n,k}$.  We will refer to the associated filtered spectrum 
$$\mathbb{S} \rightarrow \Sigma^{\infty} \lambda^{-1} F_{n,1} \simeq \Sigma^{\infty} \mathbb{CP}^{n-1} \rightarrow \Sigma^{\infty} \lambda^{-2} F_{n,2} \rightarrow \cdots$$
by $\Sigma^{\infty}_+ \{F_{n,k}\}$.
\end{dfn}

The above constructions make $\Sigma^{\infty}_+ \{F_{n,k}\}$ into a filtered spectrum whose underlying graded spectrum is $\mathbb{E}_2$.  We will now discuss the problem of making the filtered spectrum itself $\mathbb{E}_2$, or at least $\mathbb{A}_\infty$.  For this, recall from Lemma \ref{lem:FilAsGrMod} that there is a graded $\mathbb{E}_\infty$ ring $A=\Sigma^{\infty}_+ \mathbb{Z}^{ds}_{\ge 0}$ so that filtered spectra may be described as $A$-modules in graded spectra.  We now discuss the following theorem, which implies Theorem \ref{thm:BottIsAoo} from the Introduction, and which again follows easily from the machinery of Beilinson--Drinfeld Grassmannians:

\begin{thm} \label{thm:AooFil}
There is a map of $\mathbb{E}_2$-algebra objects in graded spectra
$$A \simeq \Sigma^{\infty}_+ \mathbb{Z}^{ds}_{\ge 0} \longrightarrow \Sigma^{\infty}_+ \coprod_k F_{n,k}.$$
In particular, $\Sigma^{\infty}_+ \coprod_k F_{n,k}$ is an $\mathbb{A}_\infty$-algebra in $A$-modules, and so $\Sigma^{\infty}_+ \{F_{n,k}\}$ is a filtered $\mathbb{A}_\infty$-algebra.
\end{thm}

\begin{rmk}
The $\mathbb{E}_2$-algebra map $A \rightarrow \Sigma^{\infty}_+ \coprod_k F_{n,k}$ sits in a commutative diagram of $\mathbb{E}_2$-algebras
$$
\begin{tikzcd}
A \arrow{d} \arrow{r} & \Sigma^{\infty}_+ \coprod_k F_{n,k} \arrow{d} \\
\Sigma^{\infty}_+ \mathbb{Z} \arrow{r} & \Sigma^{\infty}_+ \Omega U(n).
\end{tikzcd}
$$
The map $\Sigma^{\infty}_+ \mathbb{Z} \rightarrow \Sigma^{\infty}_+ \Omega U(n)$ may be described as the suspension of the natural map
$$\Omega^2(BU(1) \rightarrow BU(n)).$$
\end{rmk}

\begin{rmk} \label{rmk:E2fil}
The fact that there is an $\mathbb{E}_2$-algebra map $A \rightarrow \Sigma^{\infty}_+ \coprod_k F_{n,k}$ is stronger than the fact that $\Sigma^{\infty}_+ \{F_{n,k}\}$ is $\mathbb{A}_\infty$ filtered, but it is weaker than the claim that $\Sigma^{\infty}_+ \{F_{n,k}\}$ is $\mathbb{E}_2$ filtered.  We do not know if the Bott filtration is $\mathbb{E}_2$ or not, but would be very interested to learn the answer.

The machinery of Beilinson--Drinfeld Grassmannians proves that the coarsened filtration consisting of every $n$th piece of the Bott filtration (i.e. $\Sigma^{\infty}_+ \{F_{n,nk}\}$) is an $\mathbb{E}_2$-filtration.  The question is equivalent to the production of an $\mathbb{E}_3$-algebra map from $A$ to the $\mathbb{E}_3$-center of the $\mathbb{E}_2$-algebra $\Sigma^{\infty}_+ \coprod \{F_{n,k}\}$. 
After group completion, this would in particular imply the existence of an $\mathbb{E}_3$-algebra map
$$\mathbb{Z} \longrightarrow (\Omega U(n))^{hU(n)}.$$
We do not know whether even this last map exists.
\end{rmk}


\begin{proof}[Proof of Theorem \ref{thm:AooFil}]
Consider $Gr_{\mathbb{G}_m}$, the affine Grassmannian for the multiplicative group.  This is a model for $\Omega S^1$ and so has $\mathbb{Z}$ many contractible connected components.  Choosing a dominant coweight corresponding to a loop of winding number $1$ identifies a copy of $\mathbb{Z}^{ds}_{\ge 0}$ inside of $Gr_{\mathbb{G}_m}$.  The Beilinson--Drinfeld Grassmannian for the group $G=\mathbb{G}_m$ then describes $\Sigma^{\infty}_+ \mathbb{Z}^{ds}_{\ge 0}$ as a sub-$\mathbb{E}_2$-algebra of $\Sigma^{\infty}_+ Gr_{\mathbb{G}_m}$.

Now, the map of groups $\mathbb{G}_m \rightarrow GL_n(\mathbb{C})$ given by the dominant coweight $(1,0,\cdots,0)$ induces a map of Beilinson--Drinfeld Grassmannians.  Applying Remark \ref{rmk:bdgrfunct}, we obtain the desired map of graded $\mathbb{E}_2$-algebras.
\end{proof}

We end this section by sketching what is named Construction \ref{cnstr:IntroGr} in the Introduction:

\begin{cnstr} \label{cnstr:E2GrConstruction}
The graded $\mathbb{A}_\infty$-algebra $gr(\Sigma^{\infty}_+ \{F_{n,k}\})$ may be equipped with the structure of a graded $\mathbb{E}_2$-algebra.
\end{cnstr}

\begin{proof}[Proof sketch]
As explained above, the $\mathbb{E}_2$-algebra in spaces $\coprod F_{n,k}$ receives a natural $\mathbb{E}_2$-map from $\mathbb{Z}^{ds}_{\ge 0}$.  We may thus view $\coprod F_{n,k}$ as an $\mathbb{E}_2$-algebra over $\mathbb{Z}^{ds}_{\ge 0}$ (by, e.g., the straightening and unstraightening correspondence).  There is a diagram of $\mathbb{E}_2$-algebras:
$$
\begin{tikzcd}
\coprod_k F_{n,k} \arrow{d} \arrow{r} & \Omega U(n) \arrow{r} & \Omega U \simeq BU \times \mathbb{Z} \arrow{r}{J} & Pic(\mathbb{S}) \subset \Sp \\
\mathbb{Z}^{ds}_{\ge 0}.
\end{tikzcd}
$$
In \cite[1.7]{Segal}, it is proven that the colimit of the functor $\coprod F_{n,k} \longrightarrow \Sp$ is equivalent (as a spectrum) to $gr(\Sigma^{\infty}_+ \{F_{n,k}\})$.  Note that this colimit is more classically described as the Thom spectrum of the map $\coprod F_{n,k} \longrightarrow BU \times \mathbb{Z}$.

One may also compute this colimit by first making a left Kan extension along the map $$\coprod F_{n,k} \rightarrow \mathbb{Z}^{ds}_{\ge 0},$$ and then taking the coproduct of the images of the resulting map $$\mathbb{Z}^{ds}_{\ge 0} \rightarrow \Sp.$$  Taking an operadic left Kan extension as in \cite[3.1.2]{HA}, one learns that the left Kan extension $\mathbb{Z}^{ds}_{\ge 0} \longrightarrow \Sp$ is lax $\mathbb{E}_2$-monoidal.  The properties of Day convolution (explained in, e.g., Appendix \ref{app:day}) then imply that the Thom spectrum is naturally an $\mathbb{E}_2$-graded spectrum.

To see that the underlying $\mathbb{A}_\infty$-graded spectrum agrees with the associated graded of the Bott filtration, notice that the zero-section of the Thom construction is a map of graded $\mathbb{E}_2$-algebras.  This zero-section is, on the $k$th graded piece, a model for the map $$\Sigma^{\infty} F_{n,k} \longrightarrow \Sigma^{\infty} F_{n,k}/F_{n,k-1}.$$
The sequence of graded $\mathbb{E}_2$-algebra maps
$$\Sigma^{\infty}_+ \mathbb{Z}^{ds}_{\ge 0} \longrightarrow \Sigma^{\infty}_+ \coprod_k F_{n,k} \longrightarrow \text{Thom}\left( \coprod_k F_{n,k} \right)$$
then implies the result.
\end{proof} 

\section{A General Splitting Machine}\label{sec:SplittingMachine}


Let $[n]$ denote the linearly ordered set of integers $0\leq i\leq n$.  Define $\Fil_n = \text{Fun}([n], \Sp^{\J})$ and $\Cofil_n = \text{Fun}([n]^{\text{op}},\Sp^{\J})$.  These categories admit functors to $\Sp^{\J}$ by taking colimit and limit, respectively.  Let $\C_n = \Fil_n \times_{\Sp^{\J}} \Cofil_n.$  Finally, let $\Gr_n = \text{Fun}([n]^{\text{ds}}, \Sp^{\J})$ where $[n]^{\text{ds}}$ denotes the underlying discrete category.  We have the following lemma:

\begin{lem}
For all integers $n>0$, there is a fully faithful functor $i_n:\Gr_{n+1} \to \C_n.$  
\end{lem}
\begin{proof}
An element of $\C_n$ is given by a sequence of functors 
\begin{center}
$X_0 \longrightarrow X_1 \longrightarrow \cdots \longrightarrow X_n \simeq Y_n \longrightarrow \cdots \longrightarrow Y_1 \longrightarrow Y_0$ 
\end{center}
where the middle 
\end{proof}

We may then take inverse limits to get a category $\C_\infty = \Fil(\Sp^{\J}) \times_{\Sp^{\J}} \Cofil(\Sp^{\J})$ and a functor $i: \Gr \to \C_\infty$. 

\begin{cor}
The functor $i$ is fully faithful.
\end{cor}
\begin{proof}
%This amounts to checking that taking inverse limits retains fully faithfulness.  this is obvious, thanks to arpon
\end{proof}

at the end, restrict connectivity so that it's monoidal

\section{Multiplicative Aspects of Weiss Calculus}



In this section, we will briefly review notions of Weiss calculus to set notation and then prove a statement about its multiplicative properties.  The reader is referred to \cite{Weiss} for proofs and additional details.  We note that the discussion there is in the case of real vector spaces, but the results work just the same in the complex case.  We shall also work in the language of $\infty$-categories rather than topological categories, and Remark \ref{rmk:infinityweiss} justifies this passage.  

Let $\J$ be the $\infty$-category which is the nerve of the topological category whose objects are finite dimensional complex vector spaces equipped with a Hermitian inner product and whose morphisms are spaces of linear isometries.  

The theory of Weiss calculus studies functors out of $\J$ in a way analogous to Goodwillie calculus, by understanding successive ``polynomial approximations'' to these functors.  Here, we will discuss only the stable setting where we apply the theory to the functor category $\Sp^{\J}$. The central definition is:

\begin{dfn}\label{dfn:polyfun}
A functor $F\in \Sp^{\J}$ is polynomial of degree at most $n$ if the natural map $$F(V) \to \lim_U F(U\oplus V)$$ is an equivalence, where the limit is indexed over the $\infty$-category of nonzero subspaces $U\subset \mathbb{C}^{n+1}.$
\end{dfn}

As in Goodwillie calculus, the inclusion of the full subcategory $\Poly^{\leq n}(\Sp^{\J}) \subset \Sp^{\J}$ of functors which are polynomial of degree at most $n$ admits a left adjoint $$P_n: \Sp^{\J} \xrightleftharpoons{\quad} \Poly^{\leq n}(\Sp^{\J}): j_n.$$ %center this...
 The unit $\eta_n$ of this adjunction provides for each $F\in \Sp^{\J}$ a natural transformation $F \to P_nF$ which we will refer to as the \emph{degree $n$ polynomial approximation} of $F$. 

\begin{rmk}\label{rmk:infinityweiss}
This universal property was not explicitly stated in \cite{Weiss}, but it follows formally from Weiss's results as follows: the functor $P_n$ and the transformation $\eta_n$ can be defined explicitly as in \cite{Weiss} by iteratively applying the functor $\tau_n: \Sp^{\J} \to \Sp^{\J}$ defined by the formula $$\tau_n F(V) = \lim_U F(U\oplus V)$$ with the limit indexed as in Definition \ref{dfn:polyfun}.   The facts required of the functors $P_n$ in the proof of Theorem 6.1.1.10 in \cite{HA} are precisely the content of Theorem 6.3 of \cite{Weiss}.  
\end{rmk}

Given this universal property, Proposition 5.4 of \cite{Weiss} ensures the existence of a natural Taylor tower $$F \longrightarrow \cdots \longrightarrow P_{n} F \xrightarrow{p_{n-1}} P_{n-1} F \longrightarrow \cdots \longrightarrow P_0F$$ living under any functor $F\in \Sp^{\J}.$  The fiber $D_n F$ of $p_{n-1}$ has the special property that it is polynomial of degree at most $n$ and $P_{n-1} D_n F \simeq 0$.  Such a functor is called \emph{$n$-homogeneous}; such functors are completely classified by the following theorem:

\begin{thm}[{{\cite[Theorem 7.3]{Weiss}}}]
Let $F\in \Sp^{\J}$.  Then $F$ is an $n$-homogeneous functor if and only if there exists a spectrum $\Theta$ with an action of the unitary group $U(n)$ such that $$F(V) = (\Theta \wedge S^{nV})_{hU(n)}.$$
\end{thm}


The observation of Goodwillie, as exploited by \cite{Arone}, is that this provides a canonical way to split certain functorial filtrations whose successive quotients are homogeneous.  More precisely, we have the following theorem:

\begin{thm}[\cite{Arone}]\label{thm:aronesplit}
Suppose $F \in \Sp^{\J}$ is a functor together with an increasing filtration $$0 = F^{(0)} \longrightarrow F^{(1)}\longrightarrow F^{(2)} \longrightarrow  \cdots F$$ by functors $F^{(i)}\in \Sp^{\J}$ with the property that the successive quotients $F^{(n)}/F^{(n-1)}$ are $n$-homogeneous for all integers $n>0$.  Then, each functor $F^{(n)}$ is polynomial of degree at most $n$ and each composite $F^{(n-1)} \longrightarrow F^{(n)} \xrightarrow{\eta_{n-1}} P_{n-1} F^{(n)}$ is an equivalence.
\end{thm}
%we need to at some point deal with the fact that Weiss doesn't actually deal with homog. functors to spectra

In \cite{Arone}, this result is applied to the functor $F\in \Sp^{\J}$ defined by the formula $$F_V(W) = \Sigma^{\infty}_+ \Omega \J(V,V\oplus W)$$ where $V\in \J$ is a fixed finite dimensional complex vector space.  Arone provides a filtration $F^{(0)}_V(W) \subset F^{(1)}_V(W) \subset \cdots F_V(W)$ which is functorial in both $V$ and $W$, and which satisfies the constraints of Theorem \ref{thm:aronesplit} for fixed $V$.  This provides a stable splitting of the space $\Omega \J(V,V\oplus W).$  Letting $W=\mathbb{C}$ and $V=\mathbb{C}^{n-1}$, we obtain splittings of the loop groups $\Omega SU(n)$, and for higher dimension $W$, this provides splittings of the loop spaces of Stiefel manifolds.  


In order to upgrade the results of \cite{Arone} to structured multiplicative splittings, we must understand the multiplicative properties of the polynomial approximation functors.  More precisely, for a functor $F\in \Sp^{\J}$, we aim to understand the Taylor tower of $F\wedge F$ in terms of the tower for $F.$  The results in this section are likely known to experts, but the authors were not able to locate it in the literature.  They thank Jacob Lurie for suggesting that Proposition \ref{prop:weissmonoidal} is true.  

The idea is to consider all the polynomial approximations at once.  To do so, we first set some additional notation:

\begin{dfn} 
Let $\Cofil(\Sp^{\J})$ denote the functor category $\Fun(\Z_{\geq 0}^{op}, \Sp^{\J}).$  We shall refer to $\Cofil(\Sp^{\J})$ as the category of cofiltered objects in $\Sp^{\J}$.  Its objects can be thought of as towers of functors $Y_0\leftarrow Y_1\leftarrow Y_2 \leftarrow \cdots \in \Sp^{\J}$.
\end{dfn}%ok maybe I'll want cofil^+ or some garbage....

The category $\Cofil(\Sp^{\J})$ is the natural target for the Weiss tower.  The following construction makes this precise:

\begin{cnstr}\label{cnstr:tower}
We now construct a functor $$\text{Tow}: \Sp^{\J} \to \Cofil(\Sp^{\J})$$ with the property that it sends a functor $F\in \Sp^{\J}$ to its Taylor tower $$\text{Tow}(F) = P_0F \longleftarrow P_1F \longleftarrow P_2F \longleftarrow \cdots.$$


Recall that the $P_n$ functors are given as left adjoints of the fully faithful inclusions $\Poly^{\leq n}(\Sp^{\J}) \subset \Sp^{\J}$.  We proceed by telling a parametrized version of this story that includes all $n$ simultaneously.  The proper framework for such a story is the formalism of \emph{relative adjunctions}; these are developed in the $\infty$-categorical context in \cite{HA}, Section 7.3.2.  

Consider the category $\Sp^{\J}\times \Z^{op}_{\geq 0}$ together with the full subcategory $(\Sp^{\J}\times \Z^{op}_{\geq 0})_{\text{poly}} \subset \Sp^{\J}\times \Z^{op}_{\geq 0}$ on the pairs $(F, [n])$ such that $F\in \Poly^{\leq n}(\Sp^{\J}).$  Via projection, these fit into a diagram
$$
\begin{tikzcd}\label{dia:reladj}
\Sp^{\J}\times \Z^{op}_{\geq 0} \arrow[rd,"q"]& &(\Sp^{\J}\times \Z^{op}_{\geq 0})_{\text{poly}} \arrow[ld,"p"]\arrow[ll,"i"]  \\
& \Z^{op}_{\geq 0}
\end{tikzcd}
$$
 This will be relevant to us because the category of sections of $q$ are precisely $\Cofil(\Sp^{\J}).$  The sections of $p$ can be thought of those cofiltered functors such that the $n$th piece is polynomial of degree at most $n$.  We will denote this category of sections of $p$ by $\Cofil(\Sp^{\J})_{\text{poly}}.$ 

On the fibers over an integer $[n] \in \Z^{op}_{\geq 0}$, we see the inclusion $\Sp^{\J} \leftarrow \Poly^{\leq n}(\Sp^{\J}).$  It is in this sense that the current picture is a parametrized version of the ordinary polynomial approximations.  We now claim that $i$ admits a left adjoint $P^{\text{total}}: \Sp^{\J}\times \Z^{op}_{\geq 0} \to (\Sp^{\J}\times \Z^{op}_{\geq 0})_{\text{poly}}$ \emph{relative} to $\Z^{op}_{\geq 0}.$    The strategy is to use Proposition 7.3.2.6 of \cite{HA}, which tells us that we need to check the following three statements:
\begin{enumerate}
\item The functors $p$ and $q$ are locally Cartesian categorical fibrations.
\item For each $[n]\in \Z^{op}_{\geq 0}$, the functor on fibers $i|_{p^{-1}[n]}:p^{-1}[n] \to q^{-1}[n]$ admits a right adjoint.  
\item The functor $i$ carries locally $p$-Cartesian morphisms of $(\Sp^{\J}\times \Z^{op}_{\geq 0})_{\text{poly}}$ to locally $q$-Cartesian morphisms of $\Sp^{\J}\times \Z^{op}_{\geq 0}$.
\end{enumerate}

Condition (2) is clear from the existence of polynomial approximations in Weiss calculus.  To see conditions (1) and (3), we first note that $q$ is in fact a Cartesian fibration because it is a projection from a product.  Moreover, the $q$-Cartesian morphisms are precisely those morphisms which are equivalences on the $\Sp^{\J}$ coordinate.  We now observe that for any pair $(F, [m]) \in \Sp^{\J}\times \Z^{op}_{\geq 0}$ such that $F\in \Poly^{\leq m}(\Sp^{\J})$ and morphism $\sigma :[n]\to [m]$, any $q$-Cartesian edge lying over $\sigma$ with target $(F, [m])$ is also in the full subcategory $(\Sp^{\J}\times \Z^{op}_{\geq 0})_{\text{poly}}.$  This implies that $p$ is also a Cartesian fibration and that the inclusion $i$ carries $p$-Cartesian edges to $q$-Cartesian edges.  Since any Cartesian fibration is a categorical fibration (\cite[Proposition 3.3.1.7]{HTT}), conditions (1) and (3) are verified.  

We now wish to look at the adjunction at the level of sections of $q$ and $p$.  Considering functors from $\Z_{\geq 0}^{op}$ into Diagram \ref{dia:reladj}, we obtain a new diagram 
$$
\begin{tikzcd}
\Fun(\Z^{op}_{\geq 0},\Sp^{\J}\times \Z^{op}_{\geq 0}) \arrow[rr, bend left=10,"P^{\text{total}}_*"] \arrow[rd,"q_*"]& &\Fun(\Z^{op}_{\geq 0},(\Sp^{\J}\times \Z^{op}_{\geq 0})_{\text{poly}}) \arrow[ld,"p_*"]\arrow[ll,"i_*"]  \\
& \Fun(\Z^{op}_{\geq 0}, \Z^{op}_{\geq 0})
\end{tikzcd}
$$
which exhibits $P^{\text{total}}$ as a left adjoint of $i_*$ relative to $\Fun(\Z_{\geq 0}^{op},\Z_{\geq 0}^{op}).$  Proposition 7.3.2.5 of \cite{HA} ensures that there is an adjunction at the level of fibers above $\text{id}\in \Fun(\Z_{\geq 0}^{op},\Z_{\geq 0}^{op})$: $$\text{Tow}^*: \Cofil(\Sp^{\J}) \xrightleftharpoons{\quad} \Cofil(\Sp^{\J})_{\text{poly}}:j .$$  
Finally, observe that the unique functor $r: \Z^{op}_{\geq 0} \to *$ induces an adjunction $$r^*: \Sp^{\J} \xrightleftharpoons{\quad} \Cofil(\Sp^{\J}): \lim$$ where $r^*$ is the constant functor and $\lim$ is the same as right Kan extension along $r$.  We now compose these adjunctions, denoting $\text{Tow} = \text{Tow}^*\circ r^*$ to obtain: $$\text{Tow}: \Sp^{\J} \xrightleftharpoons{\quad}  \Cofil(\Sp^{\J})_{\text{poly}}: \lim.$$

It remains to check that $\text{Tow}$ actually recovers the Weiss tower.  TBD%Tow is the right functor...what does that mean?  It certainly does the right thing on objects... might want the morphisms to be the natural one.  Certainly it can't be anything else, but seems like it might be a hassle to check this.  I guess maybe we only care about the individual morphisms...so that's doable but perhaps long :(


This concludes the construction of $\text{Tow}.$ 
\end{cnstr}


The next task is to understand the monoidal structure on $\text{Tow}$.  The idea is that we would like to express $\text{Tow}(F\wedge F)$ in terms of $\text{Tow}(F)$ and a ``Day convolution'' monoidal structure on $\Cofil(\Sp^{\J}).$  However, there is trouble defining the convolution as in Section \ref{sect:monoidal} because smash product does not preserve \emph{limits} of spectra in each variable separately.  The situation becomes better if one restricts to the category $\Sp_{\fin}$ of \emph{finite} spectra.  The full subcategory $\Fun(\Z_{\geq 0}, \Sp_{\fin}^{\J})\subset \Cofil(\Sp^{\J})$ is closed under the convolution product defined in Section \ref{sect:monoidal}, and therefore inherits a symmetric monoidal structure.  By Spanier-Whitehead duality, this induces a symmetric monoidal structure on $\Fun(\Z_{\geq 0},(\Sp_{\fin}^{\J})^{op}),$ which in turn induces a symmetric monoidal structure on its opposite, $\Cofil(\Sp^{\J}_{\fin}).$  This can be described explicitly as sending %insert some pictures!
%ok this monoidal structure is wrong potentially...

DO NOT TRUST THE FOLLOWING SECTION RIGHT NOW


One would now hope that Tow restricts to a symmetric monoidal functor when we replace $\Sp^{\J}$ by $\Sp^{\J}_{\fin}.$  However, we would like to understand functors which do not necessarily take values in finite spectra.  Furthermore, it is not even clear that the derivatives of a functor $\J \to \Sp$ factoring through $\Sp_{\fin}$ will still factor through $\Sp_{\fin}.$  We may simply restrict to this situation.  

\begin{dfn} \label{dfn:stronglyfinite}

Let $\overline{\Sp}^{\J}$ denote the full subcategory of functors $F\in \Sp^{\J}$ such that the functors $P_nF$ factor through $\Sp_{\fin}^{\J}$ for all $n$.  Additionally, let $\overline{\Sp}^{\J}_{\fin}$ denote the full subcategory of $F\in \overline{\Sp}^{\J}$ for which $F$ itself factors through $\Sp^{\J}_{\fin}.$  %We will refer to objects of $\overline{\Sp}^{\J}_{\fin}$ as \emph{strongly finite} functors.  
\end{dfn}

\begin{prop}\label{prop:weissmonoidal}
The Weiss tower functor $\text{Tow}$ restricts to a symmetric monoidal functor $$\text{Tow}_{\text{fin}}: \overline{\Sp}^{\J} \to \Cofil(\overline{\Sp}^{\J}_{\fin}).$$
\end{prop}
\begin{proof}
Combining Construction \ref{cnstr:tower} with Definition \ref{dfn:stronglyfinite}, we obtain the diagram

$$
\begin{tikzcd}
\Sp^{\J} \rar[harpoon, yshift=1pt] &\lar[harpoon, yshift=-1pt] \Cofil(\Sp^{\J})&\Cofil(\Sp^{\J})_{\text{poly}} \arrow[l,hook, "j"]  \\
 \overline{\Sp}^{\J} \arrow[u, hook] &\Cofil(\overline{\Sp}^{\J}) \arrow[l, "\lim"] \arrow[u,hook] &  \Cofil(\overline{\Sp}^{\J}_{\fin})_{\text{poly}} \arrow[l,hook, "\overline{j}"]  \arrow[u,hook]\\
 \end{tikzcd}%diagram needs to be fixed
$$
where the hooked arrows denote inclusions of full subcategories.  Recall that $j$ admits a left adjoint $\text{Tow}_*$ and $\lim$ admits a left adjoint $r^*$.  It is clear that these left adjoints restrict to functors $\overline{\text{Tow}}_*$ and $r^*$ on the bottom row.  Consequently, setting $\text{Tow}_{\fin} = \overline{\text{Tow}}_* \circ r^*$, we obtain an adjunction $$\text{Tow}_{\fin}: \overline{\Sp}^{\J} \xrightleftharpoons{\quad} \Cofil(\overline{\Sp}^{\J}_{\fin})_{\text{poly}}: \overline{j}\circ \lim.$$ %need to be centered

We now show that there are natural symmetric monoidal structures on both of these categories such that the right adjoint $\overline{j}\circ \lim$ has a symmetric monoidal structure.  This is because it admits a different factorization $$\Cofil(\overline{\Sp}^{\J}_{\fin})_{\text{poly}} \xrightarrow{\overline{j}'} \Cofil(\overline{\Sp}^{\J}_{\fin}) \xrightarrow{\lim} \overline{\Sp}^{\J}.$$ The category $\Cofil(\overline{\Sp}^{\J}_{\fin})$ inherits a symmetric monoidal structure from $\Cofil(\Sp^{\J}_{\fin})$ as a full subcategory which is closed under the tensor product.  As such, the functor $\lim$ is symmetric monoidal because the the tensor product commutes with limits of finite spectra separately in each variable.  %might need to restrict connectivities for this to actually work...


bserve that a finite limit of functors which are polynomial of degree at most $n$ is itself polynomial of degree at most $n$.  This implies that the full subcategory $\overline{j}'$ is closed under the symmetric monoidal structure.  It follow that the left adjoint $\overline{\text{Tow}}_*$ naturally admits the structure of a oplax symmetric monoidal functor.  Moreover, it is clear that the constant functor $\overline{\Sp^{\J}_{\fin}}\to  \Cofil(\overline{\Sp}^{\J}_{\fin})$ is symmetric monoidal, and so precomposing by it yields an oplax monoidal functor $\text{Tow}_{\fin} :\overline{\Sp}^{\J}_{\fin} \to \Cofil(\overline{\Sp}^{\J}_{\fin})_{\text{poly}}.$

Concretely, the oplax structure can be described on the $n$th filtered piece as follows: since $\text{Tow}_{\fin}(F)\in  \Cofil(\overline{\Sp}^{\J}_{\fin})_{\text{poly}}$, the filtered piece $(\text{Tow}_{\fin}(F) \otimes \text{Tow}_{\fin}(F))_n$ is polynomial of degree at most $n$.  It follows that the natural map from $F\wedge F$ factors through a map $\varphi_n: P_n(F\wedge F)\to (\text{Tow}_{\fin}(F) \otimes \text{Tow}_{\fin}(F))_n$.  To see that $\text{Tow}_{\fin}$ is a symmetric monoidal functor, it suffices to show that each $\varphi_n$ is an equivalence.   TODO
%need to check the oplax structure map is an equivalence; do the cubes thing...

\end{proof}

\begin{rmk}
The functor $F(W) = \Sigma^{\infty}_+\Omega \J(V,V+W)$ described after Theorem \ref{thm:aronesplit} does not take values in finite spectra (note that the V has been suppressed in the notation).  However, it is filtered by functors $F^{(n)}(W)$ which do take values in finite complexes for all $n<\infty $.  Moreover, $P_nF(W) = F^{(n)}(W).$  
%to get non-finite things, we take limits in the resulting equivalence; i.e., each P_nF is a monoid, and I can take a filtered colimit of these, and I know the things preserve filtered colimits and whatnot so I'm okay.  

\end{rmk}

\begin{rmk}
Proposition \ref{prop:weissmonoidal} is written in the language of Weiss calculus as that is the present application, but the proof works equally well in Goodwillie calculus.  
\end{rmk}


\section{Stable \texorpdfstring{$\mathbb{A}_\infty$}{Aoo} Splittings} \label{sec:AooSplit}

In this brief section, we assemble results proved above in order to obtain what is labeled Theorem \ref{thm:MainAoo} in the Introduction:

\begin{thm} \label{thm:MainAooInText}
As an $\mathbb{A}_\infty$-algebra object in filtered spectra, the Bott filtration of $\Sigma^{\infty}_+ \Omega SU(n)$ is equivalent to its associated graded.
\end{thm}

\begin{proof}
The construction of the stable Bott filtration as an $\mathbb{A}_\infty$-filtered spectrum is our Theorem \ref{thm:AooFil}.  According to our Theorem \ref{thm:SplitMachine}, to complete the proof of Theorem \ref{thm:MainAooInText} it suffices to produce an $\mathbb{A}_\infty$-cofiltered spectrum with limit $\Sigma^{\infty}_+ \Omega SU(n)$ and with the property that certain composites are equivalences.

As explained in the Introduction, we follow Arone \cite{Arone} in producing the desired cofiltered spectrum by means of Weiss calculus.  Consider, in the notation of Section \ref{sec:MultWeiss} and particularly Example \ref{ex:aronefunctor}, the functor $$F_V:\mathcal{J} \longrightarrow \textbf{Spectra}$$ given by $F_V(W) = \Sigma^{\infty}_+ \Omega \mathcal{J}(V, V \oplus W)$.  Corollary \ref{cor:aronemonoidal} implies that the Taylor tower of $F_V$, applied to $W$, is an $\mathbb{A}_\infty$-cofiltered spectrum with limit $\Sigma^{\infty}_+ \Omega \mathcal{J}(V,V \oplus W)$.

Specializing to the case $V=\mathbb{C}^{n-1}$, $W=\mathbb{C}$, it is straightforward to see that $\mathcal{J}(\mathbb{C}^{n-1},\mathbb{C}^{n-1} \oplus \mathbb{C})$ is equivalent to $SU(n)$.  Roughly speaking, this is because any embedding of $\mathbb{C}^{n-1}$ into $\mathbb{C}^n$ may be extended in a unique way to an automorphism of $\mathbb{C}^n$ that is unitary of determinant one.  Thus, applying Corollary \ref{cor:aronemonoidal} in the case $V=\mathbb{C}^{n-1}$, $W=\mathbb{C}$ gives an $\mathbb{A}_\infty$-cofiltered spectrum with the desired limit.  To complete the proof of Theorem \ref{thm:MainAooInText} it suffices then to check that certain composites are equivalences.  In fact, one of the main results of \cite{Arone} is that those composites are equivalences (see the proof of \cite[Theorem 1.2]{Arone}).
\end{proof}

\begin{rmk}
Mitchell and Richter constructed \cite{CrabbBarcelona} a filtration not just of $$\Omega SU(n) \simeq \Omega \mathcal{J}(\mathbb{C}^{n-1},\mathbb{C}^n),$$ but also of $\Omega \mathcal{J}(V,V\oplus W)$ for a general $V$ and $W$.  Arone showed in \cite[Theorem 1.2]{Arone} that this Mitchell--Richter filtration always stably splits, and Corollary \ref{cor:aronemonoidal} provides an $\mathbb{A}_\infty$-cofiltered spectrum inducing this splitting.  We do not know, however, whether the Mitchell--Richter filtration is always $\mathbb{A}_\infty$-split. 
\end{rmk}

\section{\texorpdfstring{$\mathbb{E}_2$}{E2} Splittings in Complex Cobordism} \label{sec:MUE2}


In this brief section, we remark that the $\mathbb{A}_\infty$ splitting $$\Sigma^{\infty}_+ \Omega SU(n) \simeq gr(\Sigma^{\infty}_+ \{F_{n,k}\})$$ becomes $\mathbb{E}_2$ after base-change to complex bordism.  More precisely, we show that there is an equivalence of $\mathbb{E}_2$-$MU$-algebras
$$MU \smsh \Sigma^{\infty}_+ \Omega SU(n) \simeq MU \smsh gr(\Sigma^{\infty}_+ \{F_{n,k}\}).$$

Though it is true that this is an equivalence of graded $\mathbb{E}_2$-$MU$-algebras, for simplicity we do not prove that here.  We content ourselves with an argument that the underlying $\mathbb{E}_2$-$MU$-algebras are equivalent, and leave to the reader the straightforward modifications necessary to prove the graded statement.

The $\mathbb{A}_\infty$-$MU$-algebra equivalence obtained by base-change from the results of Section \ref{sec:AooSplit} is realized by a map of $\mathbb{A}_\infty$-$\mathbb{S}$-algebras
\begin{equation} \label{SplittingMap}
\Sigma_+^{\infty} \Omega SU(n) \longrightarrow MU \smsh gr(\Sigma^{\infty}_+ \{F_{n,k}\}).
\end{equation}

Our task is to show that (\ref{SplittingMap}) may be refined to a morphism of $\mathbb{E}_2$-ring spectra.  We do so by obstruction theory--the key fact powering our proof is that 
$$MU_{2*+1}\left(\Omega SU(n)\right) \cong 0.$$
This classical vanishing result may be proven via Atiyah--Hirzerburch spectral sequence, using the even cell-decomposition of $Gr_{SL_n}(\mathbb{C})$ via Iwahori orbits.  Inspired by \cite{ChadwickMandell}, we prove the following general result:

\begin{thm}
Suppose that $R$ is an $\mathbb{E}_2$-ring spectrum with no homotopy groups in odd degrees.  Then any homotopy commutative ring homomorphism
$$\Sigma^{\infty}_+ \Omega SU(n) \rightarrow R$$
lifts to a morphism of $\mathbb{E}_2$-ring spectra.  Moreover, any chosen $\mathbb{A}_\infty$ lift may be extended to an $\mathbb{E}_2$ lift.
\end{thm}

\begin{proof} 
By taking connective covers, one learns that any ring homomorphism
$$\Sigma^{\infty}_+ \Omega SU(n) \rightarrow R$$
must factor through the natural $\mathbb{E}_2$-algebra map $\tau_{\ge 0} R \rightarrow R$.  Thus, without loss of generality we will assume that $R$ is $(-1)$-connected.

It is clear that the composite ring homomorphism
$$\Sigma^{\infty}_+ \Omega SU(n) \longrightarrow R \longrightarrow \tau_{\le 0} R \simeq H\pi_0(R)$$
may be lifted to an $\mathbb{E}_2$-ring homomorphism factoring through $\tau_{\le 0} \Sigma^{\infty}_+ \Omega SU(n) \simeq H\mathbb{Z}$.   Suppose now for $q>0$ that we have chosen an $\mathbb{E}_2$-ring homomorphism 
$$\Sigma^{\infty}_+ \Omega SU(n) \longrightarrow \tau_{\le q-1} R$$
We will show that there is no obstruction to the existence of a further $\mathbb{E}_2$-lift $$\Sigma^{\infty}_+ \Omega SU(n) \longrightarrow \tau_{\le q} R,$$
and that one may be chosen lifting any specified $\mathbb{A}_\infty$ map $\Sigma^{\infty}_+ \Omega SU(n) \rightarrow \tau_{\le q} R$.

According to \cite[Theorem $4.1$]{ChadwickMandell}, there is a diagram of principal fibrations
$$
\begin{tikzcd}
\mathbb{E}_2\text{-Ring}(\Sigma^{\infty}_+ \Omega SU(n), \tau_{\le q} R) \arrow{r} \arrow{d} & \mathbb{A}_\infty\text{-Ring}(\Sigma^{\infty}_+ \Omega SU(n), \tau_{\le q} R) \arrow{d} \\
\mathbb{E}_2\text{-Ring}(\Sigma^{\infty}_+ \Omega SU(n), \tau_{\le q-1} R) \arrow{r} \arrow{d} & \mathbb{A}_\infty\text{-Ring}(\Sigma^{\infty}_+ \Omega SU(n), \tau_{\le q-1} R) \arrow{d} \\
\cS_*(BSU(n),K(\pi_q R,q+3)) \arrow{r} & \cS_*(SU(n),K(\pi_q R,q+2))
\end{tikzcd}
$$
For $q$ odd, $\tau_{\le q-1} R \simeq \tau_{\le q} R$, so there is no obstruction.  Let us therefore assume that $q$ is even.

Since the cohomology of $BSU(n)$ is even-concentrated with coefficients in any abelian group, we have that $\pi_0 \cS_*(BSU(n),K(\pi_q R,q+3)) \cong H^{q+3}(BSU(n);\pi_q R)$ is zero.  It follows then that the given class $$x \in \pi_0 \mathbb{E}_2\text{-Ring}(\Sigma^{\infty}_+ \Omega SU(n), \tau_{\le q-1} R)$$ admits some lift $$\widetilde{x} \in \mathbb{E}_2\text{-Ring}(\Sigma^{\infty}_+ \Omega SU(n), \tau_{\le q} R).$$  We may need to modify $\widetilde{x}$ to match our chosen $\mathbb{A}_\infty$-ring homomorphism.  This is always possible so long as the map
$$\pi_1(\cS_*(BSU(n),K(\pi_q R,q+3))) \longrightarrow \pi_1(\cS_*(SU(n),K(\pi_q R,q+2)))$$
is surjective.  Said in other terms, this is just the map
$$H^{2q+2}(BSU(n);\pi_q R) \longrightarrow H^{2q+1}(SU(n);\pi_q R) \cong H^{2q+2}(\Sigma SU(n);\pi_q R)$$
induced by the natural map $\Sigma SU(n) \rightarrow BSU(n)$.  It is a classical fact that this map is surjective (it follows from a calculation with the bar spectral sequence, using the fact that the cohomology of $SU(n)$ is exterior).
\end{proof}

\section{Obstructions to a General \texorpdfstring{$\mathbb{E}_2$}{E2} Splitting} \label{sec:Obstruction}



Let $3< n\leq \infty$ be an integer.  The $\mathbb{A}_{\infty}$ filtered equivalence of Theorem \ref{thm:MainAoo} gives an equivalence of $\mathbb{A}_\infty$ ring spectra  $$\Sigma^{\infty}_+ \Omega SU(n) \simeq gr(\Sigma^{\infty}_+ \{ F_{n,k} \}).$$  The right-hand side is the associated graded of the stable Bott filtration $\Sigma^{\infty}_+ \{ F_{n,k} \}$, which we showed is $\mathbb{A}_\infty$ in Theorem \ref{thm:BottIsAoo}, but which is not known to be $\E_2$ (see Question \ref{qst:BottE2}).

In this section, we show that the graded spectrum on the right-hand side cannot be given a graded $\E_2$ structure which makes the above equivalence $\E_2$ on underlying ring spectra.  This proves Theorem \ref{thm:MainObstruction}, and in particular says that even if the Bott filtration is $\E_2$, it will not be $\E_2$-split before smashing with $MU$. 

The proof is via a power operation computation.  In particular, the $\mathbb{A}_\infty$ splitting map takes the stabilization of the bottom cell $\beta_l : S^2 \to  \CP^{n-1} \to \Omega SU(n)$ on the left-hand side to the stabilization of the bottom cell $\beta_r: S^2 \to F_{n,1} \simeq \CP^{n-1}$ on the right-hand side.  We construct a power operation $\nu^{s}$ and show that $\nu^{s}(\Sigma^{\infty} \beta_l) \neq \nu^s(\Sigma^{\infty} \beta_r).$  The obstruction will be $2$-primary, so we will implicitly complete at $2$ for the remainder of the section.  

\begin{obs}Let $Y\in \Alg_{\E_2}(\cS)$, and suppose we are given a map $S^2\to Y$.  This extends to an $\E_2$ map $\Omega^2 S^4 \to Y.$  We may precompose with the map $h: S^5 \to \Omega^2 S^4$ adjoint to the Hopf map $S^7\to S^4$.  This procedure determines a natural operation $$\nu^u: \pi_2(Y) \to \pi_5(Y)$$ in the homotopy of any $\E_2$-algebra in spaces.  

Correspondingly, for any $X\in \Alg_{\E_2}(\Sp)$, a class in $\pi_2(X)$ determines an $\E_2$ map $\Sigma^{\infty}_+ \Omega^2 S^4 \to X$.  The above map $h$ then determines an operation $\nu^s :\pi_2(X) \to \pi_5(X)$ via precomposition.  This has the property that for $Y\in \Alg_{\E_2}(\cS)$ and $\beta \in \pi_2(Y)$, we have $\nu^s(\Sigma^{\infty} \beta) = \Sigma^{\infty} \nu^u (\beta).$
\end{obs}

\begin{rmk} \label{rmk:multnu}
The notation is meant to hint at the fact that if $Y = \Omega^\infty X$ comes from a spectrum, then the operation $\nu^u$ is given by multiplication by the element $\nu \in \pi_3(\mathbb{S})^{\wedge}_2$ from the $2$-primary homotopy groups of the sphere spectrum.  Thus, $\nu^u$ is an unstable version of $\nu$ that is already seen in any $\E_2$ algebra in spaces.    %indeed, any "unstable power operation" is simply a multiplication of this form
\end{rmk}

We now compute the operation $\nu^s$ on $\Sigma^{\infty} \beta_l$ and $\Sigma^{\infty} \beta_r$.  
\begin{enumerate}
\item For $n>3$, observe that the natural map $\Omega SU(n) \to BU$ is an isomorphism in homology up to degree $7$.  This implies that $\pi_5(\Omega SU(n)) \simeq \pi_5(BU) \simeq 0$ because $BU$ is even.  Consequently, $\nu^u(\beta_l) = 0$ and so $\nu^s(\Sigma^{\infty} \beta_l) = 0.$  

\item For $\beta_r$, we use the assumption that $gr(\Sigma^{\infty}_+ \{ F_{n,k} \})$ is an $\E_2$ graded spectrum.  The map $\beta_r: S^2 \to F_{n,1}$ extends to an $\E_2$ map of underlying $\mathbb{E}_2$-algebras $$\Sigma^{\infty}_+ \Omega^2 S^4 \to  gr(\Sigma^{\infty}_+ \{ F_{n,k} \}).$$  Since $\Sigma^{\infty} \beta_r$ hits the degree 1 piece, we may lift this to an $\E_2$ map of graded spectra $$F_{\E_2}(\Sigma^\infty S^2[1]) \to gr(\Sigma^{\infty}_+ \{ F_{n,k} \})$$ from the free graded $\E_2$ algebra on $\Sigma^{\infty} S^2$ in degree 1 (see Example \ref{exm:snaith}).
\end{enumerate}

We aim to show that $\nu^s(\Sigma^\infty \beta_r)$ is nonzero.  From the graded statement, it suffices to see that its component in grading $1$ is nonzero; it is given by the composite $$\Sigma^{\infty} S^5 \to \Sigma^{\infty}_+ \Omega^2 S^4 \to \Sigma^{\infty} S^2 \xrightarrow{\Sigma^{\infty} \beta_r} \Sigma^{\infty} F_{n,1} =\Sigma^{\infty} \CP^{n-1}$$ where the middle map is given by projection onto the first graded piece (it's the map from the Snaith splitting).  It is easy to see that the first composite $\Sigma^{\infty} S^5 \to \Sigma^{\infty} S^2$ is simply $\nu \in \pi_3(\mathbb{S})^{\wedge}_2.$  Therefore, the whole composite is given by the product $\nu\cdot (\Sigma^{\infty} \beta_r).$  
However, it was computed in \cite[Theorem II.8]{Liulevicius} that $\pi_5(\Sigma^{\infty}\CP^{\infty})=\mathbb{Z}/2$ generated by $\nu \cdot (\Sigma^{\infty}\beta_r).$  Moreover, the natural map $\Sigma^{\infty}\CP^{n-1} \to \Sigma^{\infty}\CP^\infty$ is an isomorphism on $\pi_5$ for $n>3$.  We conclude that $\nu \cdot (\Sigma^{\infty}\beta_r )\neq 0$ and thus $\nu^s(\Sigma^{\infty} \beta_r) \neq 0$.  This contradicts the existence of an $\E_2$ splitting.   


\begin{rmk}
Taking the limit as $n\to\infty$, we see from the above computations that the $\E_2$ power operations on the bottom cells of $BU$ and $Q\CP^{\infty}$ do not agree.  The bottom of the Weiss tower for the functor $V \mapsto BU(V)$ gives a well-known loop map $s:BU \to Q\CP^{\infty}$, implementing the splitting principle.  The obstruction of this section recovers the classical fact that $s$ is not a double loop map.
\end{rmk}



\appendix

\section{Further Properties of Day Convolution}


\subsection{Monoidal structures, II}\label{sect:monoidal2}
Here we discuss some additional constructions and results that we will need for the more technical parts of this paper.  %at some point, warn that we'll have to use infinity operads

The monoidal structures on our categories will arise from Day convolution.  This was studied for $\infty$-categories by Glasman \cite{Glasman} and Lurie \cite{LurieRot, HA} at varying levels of generality.  We will find it convenient to use the formulation from Section 2.2.6 of \cite{HA}.  

\begin{thm}[\cite{HA}, Example 2.2.6.9]
Let $\C$ and $\D$ be symmetric monoidal $\infty$-categories.  Then there is an $\infty$-operad $\Fun(\C, \D)^{\otimes} $ with the following properties:
\begin{enumerate}
\item The underlying $\infty$-category of $\Fun(\C,\D)^{\otimes}$ is the functor category $\Fun(\C, \D)$.
\item The $\infty$-category $\Alg_{\E_\infty}(\Fun(\C, \D)^{\otimes})$ of $\E_\infty$ algebras in $\Fun(\C,\D)^{\otimes}$ is equivalent to the category of lax symmetric monoidal functors from $\C$ to $\D$.  

\end{enumerate}
\end{thm}

In order for the $\infty$-operad $\Fun(\C,\D)^{\otimes}$ to actually be a symmetric monoidal $\infty$-category, one needs to make additional assumptions.  

\begin{prop}[\cite{HA}, Proposition 2.2.6.16]\label{prop:dayconvsmc}
Let $\C$ and $\D$ be symmetric monoidal $\infty$-categories.  Suppose that $\kappa$ is an uncountable regular cardinal such that:
\begin{enumerate}
\item $\C$ is essentially $\kappa$-small.
\item $\D$ admits $\kappa$-small colimits.
\item The tensor product on $\D$ preserves $\kappa$-small colimits separately in each variable.  
\end{enumerate}
Then $\Fun(\C,\D)^{\otimes}$ is a symmetric monoidal $\infty$-category.  
\end{prop}

Recall that the Day convolution is defined classically via left Kan extension.  Assumptions (1) and (2) ensure that the relevant Kan extensions exist.  Assumption (3) then ensures that the multiplication is associative by allowing the colimits taken in the formula for left Kan extension to commute with the tensor product.  

As stated before, Proposition \ref{prop:dayconvsmc} is sufficient to construct symmetric monoidal $\infty$-categories $\Fil(\Sp)$ and $\Gr(\Sp)$.  However, we wish to understand the interaction of the Weiss calculus with multiplicative structure; there, the filtrations go the other way.

\begin{dfn}Let $\D$ be an $\infty$-category.  Let $\Cofil(\Sp^{\D})$ denote the functor category $\Fun(\Z_{\geq 0}^{op}, \Sp^{\D}).$  We shall refer to $\Cofil(\Sp^{\D})$ as the category of cofiltered objects in $\Sp^{\D}$.  Its objects can be thought of as towers of functors $Y_0\leftarrow Y_1\leftarrow Y_2 \leftarrow \cdots \in \Sp^{\D}$.
\end{dfn}

We would like to make $\Cofil(\Sp)$ a symmetric monoidal $\infty$-category by putting the Day convolution on its opposite, $\Fun(\Z_{\geq 0}, \Sp^{op}).$  However, the smash product of spectra does not preserve small colimits separately in each variable.  Nevertheless, it does preserve \emph{finite} colimits separately in each variable.  In fact, these are the only colimits that are needed in the case at hand and so we have the following variant of Proposition \ref{prop:dayconvsmc}:

\begin{var}\label{var:day}
Let $\C$ and $\D$ be symmetric monoidal $\infty$-categories.  Suppose that:
\begin{enumerate}
\item Let $I$ be a finite set and consider the multiplication map $\Pi_{i\in I} \C \to \C$.  For every $C\in \C$, the slice category $\Pi_{i\in I}\C \times_{\C} \C_{/C}$ is finite.  %should also assume some other slices are finite
\item $\D$ admits finite colimits. 
\item The tensor product on $\D$ preserves finite colimits separately in each variable.  
\end{enumerate}
Then $\Fun(\C, \D)^{\otimes}$ is a symmetric monoidal $\infty$-category.  
\end{var}
\begin{proof}
This follows directly from the same arguments as Proposition \ref{prop:dayconvsmc}.  In \cite[Corollary 2.2.6.14]{HA}, the assumptions are used to guarantee the existence of a left Kan extension; this again exists by assumptions (1) and (2) and \cite[Lemma 4.3.2.13]{HTT}.  Similarly, the proof of \cite[Proposition 2.2.6.16]{HA} only makes reference to commuting tensor products in $\D$ with finite colimits, which is ensured by assumption (3).  
\end{proof}


In Section \ref{sec:SplittingMachine}, we will need to consider not only the Day convolution monoidal structure on $\Fun(\C,\D)$ but its functoriality as $\C$ varies.  For instance, we would for symmetric monoidal functors $\C_1 \to \C_2$ to induce symmetric monoidal functors $\Fun(\C_1,\D) \to \Fun(\C_2,\D)$ via left Kan extension.  

We give a very close variant of \cite[Corollary 3.8]{Nikolaus} in our current framework:

\begin{prop}\label{prop:kanmonoidal}
Let $\C_1$, $\C_2$, and $\D$ be symmetric monoidal $\infty$-categories such that the pairs $(\C_1, \D)$ and $(\C_2, \D)$ satisfy the hypotheses of Proposition \ref{prop:dayconvsmc} or of Variant \ref{var:day}.  Let $f:\C_1 \to \C_2$ be a symmetric monoidal functor.  Then there is an adjunction 
$$ f_! : \Fun(\C_1, \D) \xrightleftharpoons{\quad} \Fun(\C_2, \D): f_* $$ %okay need to be careful here
where $f_*$ denotes restriction and $f_!$ denotes left Kan extension.  Moreover, the functor $f_*$ is lax symmetric monoidal and $f_!$ is symmetric monoidal.  
\end{prop}
\begin{proof}
The universal property of $\Fun(\C_1, \D)^{\otimes}$ immediately implies the existence of a map of $\infty$-operads $\Fun(\C_2,\D)^{\otimes} \to \Fun(\C_1,\D)^{\otimes}$, which makes $f_*$ a lax symmetric monoidal functor.  

Assumptions (1) and (2) of Proposition \ref{prop:dayconvsmc} guarantee that the adjunction exists at the level of $\infty$-categories.  The rest of the proof from \cite[Corollary 3.8]{Nikolaus} carries over verbatim.  
\end{proof}


%should check I wrote lax *symmetric* monoidal, not just lax monoidal everywhere


\section{Square Zero Algebras}


We will now discuss square zero extensions in our framework.  For this, it will be convenient to work with the category $\Gr_u$ of \emph{unital} graded spectra in the strong sense that the unit map induces an equivalence in grading 0.  
Note that there is a fully faithful functor $T:\Sp \to \Gr_u$ which sends a spectrum $A$ to the graded spectrum $$S^0, A, *, *, \cdots.$$  Its essential image is the full subcategory $i: \Gr^{\leq 1}_u \to \Gr_u$ consisting of unital graded spectra $X$ such that $X_k$ is contractible for $k>1$.  In this section, we analyze graded spectra in this subcategory $\Gr^{\leq 1}_u$. Our goal is to show any such graded spectrum admits an essentially unique $\E_n$-algebra structure for any $0\leq n\leq \infty.$  This goal is realized in Proposition \ref{prop:sq0unique}.  


The inclusion $i$ fits into an adjunction
$$L^{\leq 1}:  \Gr_u \xrightleftharpoons{\quad} \Gr_u^{\leq 1} : i$$ where the left adjoint $L^{\leq 1}$ can be thought of as truncating above grading 1.  The localization $L^{\leq 1}$ is visibly compatible with the monoidal structure in the sense that for any $f:X\to Y$ in $\Gr_u$ such that $L^{\leq 1}f$ is an equivalence and any $Z\in \Gr_u$, the natural map $L^{\leq 1} (X\wedge Z) \to L^{\leq 1}(Y\wedge Z)$ is an equivalence.  We are now in the situation of Proposition 2.2.1.9 of \cite{HA}, and so we may conclude that $\Gr_u^{\leq 1}$ inherits a symmetric monoidal structure such that $L^{\leq 1}$ is symmetric monoidal and the inclusion $i$ is lax monoidal.  This monoidal structure can be described explicitly by the formula $$X \otimes_{\Gr_u^{\leq 1}} Y = L^{\leq 1}(X \otimes_{\Gr_u} Y).$$

We may then apply Remark 7.3.2.13 of \cite{HA} to obtain an adjunction at the level of algebras for any integer $0\leq n\leq \infty$:
$$L^{\leq 1}_{\text{alg}}: \Alg_{\E_n}( \Gr_u)  \xrightleftharpoons{\quad} \Alg_{\E_n}(\Gr_u^{\leq 1}) : i_{\text{alg}}.$$

Since the counit $Li \to \text{id}$ before lifting to algebras is an equivalence, we have that the counit $L^{\leq 1}_{\text{alg}}  i_{\text{alg}} \to \text{id}$ is also an equivalence.  This implies in particular that $i_{\text{alg}}$ is fully faithful.  We are now in position to prove the main proposition of this section:

\begin{prop}\label{prop:sq0unique}
Let $0\leq n\leq \infty$ be an integer.  Then, there is a sequence of equivalences of categories $$\Sp \xrightarrow{\bar{T}} \Gr_u^{\leq 1} \longrightarrow \Alg_{\E_n}(\Gr_u^{\leq 1}) \longrightarrow  \Alg_{\E_n}( \Gr_u) \times_{\Gr_u} \Gr^{\leq 1}_u $$ where the first functor $\bar{T}$ is obtained by restricting the codomain of the functor $T:\Sp \to \Gr_u.$  In particular, for any $X\in \Gr_u^{\leq 1}$, the graded spectrum $iX\in \Gr_u$ has an essentially unique $\E_n$-algebra structure.  
\end{prop}
\begin{proof}
The third arrow is defined by $i_{\text{alg}}$, and is an equivalence because $i_{\text{alg}}$ is fully faithful, so it remains to consider the first two arrows.  

We have already seen that the functor $\bar{T}: \Sp \to \Gr^{\leq 1}_u$ is an equivalence of categories.  However, it may be promoted to a symmetric monoidal equivalence when $\Sp$ is given the cocartesian monoidal structure - that is, the monoidal structure defined by $\vee$, the coproduct.  This monoidal structure has a very special property: by Proposition 2.4.3.9 of \cite{HA}, there is for each $n$ an equivalence $\Sp \simeq \Alg^{\vee}_{\E_n}(\Sp)$, where the superscript $\vee$ indicates that we are considering algebras under the wedge.  Informally, this says that any $Y\in \Sp$ admits an essentially unique $\E_n$-algebra structure under the coproduct.  It follows that the same holds for any $X\in \Gr^{\leq 1}_u$, and so there is an equivalence $\Gr^{\leq 1}_u \to \Alg_{\E_n}(\Gr^{\leq 1}_u)$, as desired.    
\end{proof}


\begin{term}
Let $0\leq n\leq \infty$ be an integer.  By taking composing with the colimit functor, Proposition \ref{prop:sq0unique} provides a functor $$\omega_n: \Sp \to \Alg_{\E_n}(\Sp)$$ which we will refer to as the square zero extension.  It sends a spectrum $X$ to a ring with underlying spectrum $S^0\vee X$.  We will call any $\E_n$-algebra structure produced via Proposition \ref{prop:sq0unique} or $\omega_n$ a \emph{square zero} $\E_n$ structure.  
\end{term}

\begin{rmk} \label{rmk:maptosq0}
For any $X\in \Alg_{\E_n}(\Gr_u)$, we have a map $X\to i_{alg}L^{\leq 1}_{alg}X$ of $\E_n$-algebras.  Taking colimits, we obtain a map $\colim X \to \colim i_{alg}L^{\leq 1}_{alg}X$ of $\E_n$ ring spectra.
 We may summarize this informally by saying that any $\E_n$-split ring spectrum $X$ has an $\E_n$ map to the square zero extension determined by its degree one component $X_1$.  
 \end{rmk}

We will need to understand structured maps into square zero extensions.  This amounts to understanding the space of units.  In classical algebra, given a commutative ring $A$ and an $A$-module $M$, the group of units of the square zero extension are given by the formula $$(A\oplus M)^{\times} \simeq A^{\times} \times M.$$ A similar formula holds in our context for suspension spectra of connected spaces.  

\begin{prop}\label{prop:sq0units}
Let $0\leq n\leq \infty$ be an integer and $X\in \cS$ a connected space.  
 There is a canonical equivalence $$GL_1(\omega_n (\Sigma^{\infty} X)) \simeq GL_1(S^0) \times QX$$ of $\E_n$-algebras in spaces, where $QX$ is our notation for $\Omega^{\infty}\Sigma^\infty X$.  
\end{prop}
\begin{proof}
The functors $\omega_n$ are compatible under restriction, so it suffices to prove the statement for $n=\infty.$  For this case, we will show that there is a splitting $$gl_1(\omega_\infty(\Sigma^{\infty}X)) \simeq gl_1(S^0) \vee \Sigma^\infty X$$ of spectra, where $gl_1$ denotes the spectrum of units of an $\E_\infty$-ring introduced in \cite{MQRT}.  We first look at what happens on homotopy.  Recall that for any $\E_\infty$ ring spectrum $Y$, we have the formula $$\pi_*(gl_1(Y)) \simeq (\pi_*(Y))^\times$$ where on the right hand side, we consider $\pi_*(Y)$ as a graded ring.  In our case, this yields an identification $$\pi_*(gl_1(\omega_\infty (\Sigma^{\infty}X))) \simeq (\pi_*(S^0) \oplus \pi_*(\Sigma^{\infty}X))^\times \simeq \pi_*(S^0)^{\times} \times \pi_*(\Sigma^\infty X)$$
where we have used that on homotopy, $\omega_\infty (\Sigma^\infty X)$ is a square zero extension.  To conclude the proof, it suffices to show that the two factors on the right hand side can be realized by maps of spectra.  

The first factor is realized simply by $gl_1$ of the unit map $S^0 \to \omega_\infty (\Sigma^\infty X).$  In fact, it is not difficult to see directly that this map is split.  
%First, note that there is a tautological retraction sequence $$*\longrightarrow \Sigma^\infty X \longrightarrow *$$ of spectra, where $*\in \Sp$ denotes the zero object.  Applying $gl_1\circ \omega_\infty$ allows us to conclude that $gl_1(S^0) = gl_1(\omega_\infty(*))$ splits off of $gl_1(\omega_\infty(\Sigma^\infty X))$ in spectra.  In particular, this determines a map $$a:gl_1(S^0) \to gl_1(\omega_\infty (\Sigma^\infty X)).$$

For the second factor, observe that since $\omega_\infty (\Sigma^\infty X)$ is an $\E_\infty$-ring, it receives a canonical $\E_\infty$ map $$\Sigma^{\infty}_+ QX \longrightarrow \omega_\infty (\Sigma^\infty X)$$ from the free $\E_\infty$ ring on $\Sigma^\infty X$ which extends the canonical map of spectra $\Sigma^\infty X \to \omega_\infty (\Sigma^\infty X).$  Now, note that there is an adjunction \cite{MQRT} $$\Sigma^{\infty}_+\Omega^\infty : \Sp \xrightleftharpoons{\quad} \Alg_{\E_\infty}(\Sp) : gl_1$$ under which the above map may be identified with a map $$b: \Sigma^\infty X \to gl_1 (\omega_\infty (\Sigma^{\infty} X))$$ of spectra.  \textbf{NEED TO SAY A TINY BIT MORE}
%how to compute this on htpy...more or less obv but maybe should explain


Finally, we may take the map $a\vee b: gl_1(S^0)\vee \Sigma^\infty X \to gl_1(\omega_\infty (\Sigma^{\infty} X))$ and the above comments show that it is an equivalence, as desired.  








%recall the symmetric monoidal equivalence $\bar{T}:\Sp \to \Gr^{\leq 1}_u$ of Proposition \ref{prop:sq0unique}, where $\Sp$ is given a monoidal structure under $\vee$.  By the proof of Proposition \ref{prop:sq0unique}, there is a canonical retraction sequence of $\E_\infty$ objects in $\Sp$ (under $\vee$) $$* \longrightarrrow X \longrightarrow *.$$  Taking the associated square $0$ spectra



%for general reasons we know gl_1 S^0 splits off gl_1 S^0 v X
%by the \Sigma^\infty \Omega^\infty // gl_1 adjunction and the fact that \Sigma^\infty Q is free E_\infty, you get a map from \Sigma^\infty X to gl_1 S^0 v X that's nontrivial enough on homotopy...so then you just add em up and you're done

\end{proof}













\begin{comment}

We can add this in to a second version of the paper, or perhaps write a small second paper.

\section{Snaith's Construction of Periodic Complex Bordism} \label{sec:SnaithSplitting}


A classical theorem of Snaith \cite{SnaithOriginal} gives an equivalence of homotopy commutative ring spectra $$\Sigma^{\infty}_+ BU [\beta^{-1}] \simeq MUP.$$  The equivalence arises from considering the total $MU$-Chern class map $BU \to GL_1(MUP).$  It is known from \cite{SnaithNotMultiplicative} that the total Chern class in integral homology is not an infinite loop map.  It follows from the existence of an $\E_\infty$ map $MUP \to H\Z P$ from periodic complex bordism to periodic integral homology that Snaith's equivalence is not an equivalence of $\E_\infty$ ring spectra.  The following theorem refines this observation:


%should check if the obstruction in SnaithNotMultiplicative is also E_3...I would think it is
%should we also cite the totaro paper that does literally the same thing as SnaithNotMultiplicative?

\begin{thm}
The equivalence $\Sigma^{\infty}_+ BU [\beta^{-1}] \simeq MUP$ is $\mathbb{E}_2$ but not $\mathbb{E}_3$.
\end{thm}

\begin{proof}
Proof goes here
\end{proof}

Comment now about GepnerSnaith.
We should cite at some point here or the introduction all of \cite{SnaithNotMultiplicative},  \cite{GepnerSnaith}, and the Snaith book with the original splitting.


\section{Miscellaneous stuff here}

It would be nice to at some point deal with showing the associated graded $E_2$ structure of BU is the thom spectrum VMU(n).  I've directly pasted in some writing from a previous argument I claimed, but it definitely uses that $\coprod BU(n)$ is an $E_2$ algebra over $\Z _{\geq 0}$ which I never got straight an actual proof of.  

\begin{prop}The associated graded of $\Sigma^{\infty}_+BU$ is $E_2$ equivalent to the Thom spectrum $\bigvee MU(n).$
\end{prop}
\begin{proof}
Let $R = BU$ with its natural filtration, and let $R^{\oplus} = \coprod BU(n)$ with its natural filtration.  Let $M$ be the ($E_\infty$) filtered spectrum which is $MU(n)$ in degree $n$, and all maps are $0$.  In other words, $\bigvee MU(n)$ with its natural filtration is $I(res(M))$.  

We begin with a filtered $E_\infty$ map $z:R^\oplus \to I(res(M))$ coming from the zero section.  Then, $R^\oplus$ comes with the structure of an $E_2$ algebra over $\Z_{\geq 0}^{fil}$.  In fact, $I(res(M))$  has a trivial structure as an $E_\infty$-algebra over $\Z_{\geq 0}^{fil}$ via the augmentation $\Z_{\geq 0}^{fil}\to S^{0,fil} \to I(res(M))$.  We may then tensor $z$ along the augmentation to get a map of $E_2$ filtered spectra $z':R \to I(res(M))$.  

There is a canonical equivalence $I(res(M)) \otimes \mathbb{A} \simeq M$ because $M$ is in the image of $\mathbb{A} \otimes I(-)$ (that is, all the maps in the filtration of $M$ were zero).  As such, $M$ acquires a canonical structure as an $\mathbb{A}$ algebra such that the map $M \otimes \mathbb{A} \to M$ is a map of $E_2$ rings (in fact I think it's $E_\infty$?).  

Finally, we observe that we may tensor $z'$ with $\mathbb{A}$ and compose with the multiplication map to get an $E_2$ map $R\otimes A \to M \otimes A \to M$ which is the right thing up to homotopy, so it's an equivalence.  
\end{proof}

******  

What is $\Sigma^{\infty}_+ \Omega SU(n)[\beta^{-1}]$, by the way?  Is it related to a periodic version of the $X(n)$-filtration of $MU$??


\end{comment}

\bibliographystyle{alpha}
\bibliography{Bibliography}

\end{document}