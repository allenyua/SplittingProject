In this section, we determine the multiplicative properties of the Weiss calculus polynomial approximation functors \cite{Weiss}.  More precisely, for a functor $F$, we aim to understand the Taylor tower of $F\wedge F$ in terms of the tower for $F.$  The results in this section are likely known to experts, but the authors were not able to locate it in the literature.  They thank Jacob Lurie for suggesting that Theorem \ref{thm:weissmonoidal} is true.

\subsection{Review of Weiss calculus}
We briefly review notions of Weiss calculus to set notation.  The reader is referred to \cite{Weiss} for proofs and additional details.  We note that the discussion there is in the case of real vector spaces, but the results work just the same in the complex case.  We shall also work in the language of $\infty$-categories rather than topological categories, and Remark \ref{rmk:infinityweiss} justifies this passage.  

Let $\J$ be the $\infty$-category which is the nerve of the topological category whose objects are finite dimensional complex vector spaces equipped with a Hermitian inner product and whose morphisms are spaces of linear isometries.  

Weiss calculus studies functors out of $\J$ in a way analogous to Goodwillie calculus, by understanding successive ``polynomial approximations'' to these functors.  Here, we will discuss only the stable setting where we apply the theory to the functor category $\Sp^{\J}$. The central definition is:

\begin{dfn}\label{dfn:polyfun}
A functor $F\in \Sp^{\J}$ is polynomial of degree $n$ if the natural map $$F(V) \to \lim_U F(U\oplus V)$$ is an equivalence, where the limit is indexed over the $\infty$-category of nonzero subspaces $U\subset \mathbb{C}^{n+1}.$
\end{dfn}

As in Goodwillie calculus, the inclusion of the full subcategory $\Poly^{\leq n}(\Sp^{\J}) \subset \Sp^{\J}$ of functors which are polynomial of degree $n$ admits a left adjoint $$P_n: \Sp^{\J} \xrightleftharpoons{\quad} \Poly^{\leq n}(\Sp^{\J}): j_n.$$ 
 The unit $\eta_n$ of this adjunction provides for each $F\in \Sp^{\J}$ a natural transformation $F \to P_nF$ which we will refer to as the \emph{degree $n$ polynomial approximation} of $F$. 

\begin{rmk}\label{rmk:infinityweiss}
This universal property was not explicitly stated in \cite{Weiss}, but it follows formally from Weiss's results as follows: the functor $P_n$ and the transformation $\eta_n$ can be defined explicitly as in \cite{Weiss} by iteratively applying the functor $\tau_n: \Sp^{\J} \to \Sp^{\J}$ defined by the formula $$\tau_n F(V) = \lim_U F(U\oplus V)$$ with the limit indexed as in Definition \ref{dfn:polyfun}.   The facts required of the functors $P_n$ in the proof of Theorem 6.1.1.10 in \cite{HA} are precisely the content of Theorem 6.3 of \cite{Weiss}.  
\end{rmk}

Given this universal property, Proposition 5.4 of \cite{Weiss} ensures the existence of a natural Taylor tower $$F \longrightarrow \cdots \longrightarrow P_{n} F \xrightarrow{p_{n-1}} P_{n-1} F \longrightarrow \cdots \longrightarrow P_0F$$ living under any functor $F\in \Sp^{\J}.$  The fiber $D_n F$ of $p_{n-1}$ has the special property that it is polynomial of degree $n$ and $P_{n-1} D_n F \simeq 0$.  Such a functor is called \emph{$n$-homogeneous}; such functors are completely classified by the following theorem:

\begin{thm}[{{\cite[Theorem 7.3]{Weiss}}}]
Let $F\in \Sp^{\J}$.  Then $F$ is an $n$-homogeneous functor if and only if there exists a spectrum $\Theta$ with an action of the unitary group $U(n)$ such that $$F(V) = (\Theta \wedge S^{nV})_{hU(n)}.$$
\end{thm}


\subsection{The Taylor tower}

It is helpful, for our study of multiplicative properties, to package all polynomial approximations into a single object--the following construction makes this precise:
%To do so, we first set some additional notation:

%\begin{dfn} 
%Let $\Cofil(\Sp^{\J})$ denote the functor category $\Fun(\Z_{\geq 0}^{op}, \Sp^{\J}).$  We shall refer to $\Cofil(\Sp^{\J})$ as the category of cofiltered objects in $\Sp^{\J}$.  Its objects can be thought of as towers of functors $Y_0\leftarrow Y_1\leftarrow Y_2 \leftarrow \cdots \in \Sp^{\J}$.
%\end{dfn}
%ok maybe I'll want cofil^+ or some garbage....

%The category $\Cofil(\Sp^{\J})$ is the natural target for the Weiss tower.  The following construction makes this precise:

\begin{cnstr}\label{cnstr:tower}
We now construct a functor $$\mathrm{Tow}: \Sp^{\J} \to \Cofil(\Sp^{\J})$$ with the property that it sends a functor $F\in \Sp^{\J}$ to its Taylor tower $$\mathrm{Tow}(F) = P_0F \longleftarrow P_1F \longleftarrow P_2F \longleftarrow \cdots.$$


Recall that the $P_n$ functors are given as left adjoints of the fully faithful inclusions $$\Poly^{\leq n}(\Sp^{\J}) \subset \Sp^{\J}.$$  We proceed by telling a parametrized version of this story that includes all $n$ simultaneously.  The proper framework for such a story is the formalism of \emph{relative adjunctions}; these are developed in the $\infty$-categorical context in \cite[Section 7.3.2]{HA}.  

Consider the category $\Sp^{\J}\times \Z^{op}_{\geq 0}$ together with the full subcategory $$(\Sp^{\J}\times \Z^{op}_{\geq 0})_{\text{poly}} \subset \Sp^{\J}\times \Z^{op}_{\geq 0}$$ on the pairs $(F, [n])$ such that $F\in \Poly^{\leq n}(\Sp^{\J}).$  Via projection, these fit into a diagram
$$
\begin{tikzcd}\label{dia:reladj}
\Sp^{\J}\times \Z^{op}_{\geq 0} \arrow[rd,"q"]& &(\Sp^{\J}\times \Z^{op}_{\geq 0})_{\text{poly}} \arrow[ld,"p"]\arrow[ll,"i"]  \\
& \Z^{op}_{\geq 0}
\end{tikzcd}
$$
 This will be relevant to us because the category of sections of $q$ are precisely $\Cofil(\Sp^{\J}).$  The sections of $p$ can be thought of as those cofiltered functors such that the $n$th piece is polynomial of degree $n$.  We will denote this category of sections of $p$ by $\Cofil(\Sp^{\J})_{\text{poly}}.$ 

On the fibers over an integer $[n] \in \Z^{op}_{\geq 0}$, we see the inclusion $\Sp^{\J} \leftarrow \Poly^{\leq n}(\Sp^{\J}).$  It is in this sense that the current picture is a parametrized version of the ordinary polynomial approximations.  We now claim that $i$ admits a left adjoint $P^{\text{total}}: \Sp^{\J}\times \Z^{op}_{\geq 0} \to (\Sp^{\J}\times \Z^{op}_{\geq 0})_{\text{poly}}$ \emph{relative} to $\Z^{op}_{\geq 0}.$    The strategy is to use Proposition 7.3.2.6 of \cite{HA}, which tells us that we need to check the following three statements:
\begin{enumerate}
\item The functors $p$ and $q$ are locally Cartesian categorical fibrations.
\item For each $[n]\in \Z^{op}_{\geq 0}$, the functor on fibers $i|_{p^{-1}[n]}:p^{-1}[n] \to q^{-1}[n]$ admits a right adjoint.  
\item The functor $i$ carries locally $p$-Cartesian morphisms of $(\Sp^{\J}\times \Z^{op}_{\geq 0})_{\text{poly}}$ to locally $q$-Cartesian morphisms of $\Sp^{\J}\times \Z^{op}_{\geq 0}$.
\end{enumerate}

Condition (2) is clear from the existence of polynomial approximations in Weiss calculus.  To see conditions (1) and (3), we first note that $q$ is in fact a Cartesian fibration because it is a projection from a product.  Moreover, the $q$-Cartesian morphisms are precisely those morphisms which are equivalences on the $\Sp^{\J}$ coordinate.  
Now suppose we are given a pair $(F, [m]) \in \Sp^{\J}\times \Z^{op}_{\geq 0}$ such that $F\in \Poly^{\leq m}(\Sp^{\J})$ and morphism $\sigma :[n]\to [m]$.  Any $q$-Cartesian edge lying over $\sigma$ with target $(F, [m])$ has source equivalent to $(F, [n])$ and thus is also in the full subcategory $(\Sp^{\J}\times \Z^{op}_{\geq 0})_{\text{poly}}$ because $m\leq n$.   
%We now observe that for any pair $(F, [m]) \in \Sp^{\J}\times \Z^{op}_{\geq 0}$ such that $F\in \Poly^{\leq m}(\Sp^{\J})$ and morphism $\sigma :[n]\to [m]$, any $q$-Cartesian edge lying over $\sigma$ with target $(F, [m])$ is also in the full subcategory $(\Sp^{\J}\times \Z^{op}_{\geq 0})_{\text{poly}}.$  
Since $p$ is certainly an inner fibration (by construction as a full subcategory), this implies that $p$ is also a Cartesian fibration and that the inclusion $i$ carries $p$-Cartesian edges to $q$-Cartesian edges.  Since any Cartesian fibration is a categorical fibration (\cite[Proposition 3.3.1.7]{HTT}), conditions (1) and (3) are verified.  

We now wish to look at the adjunction at the level of sections of $q$ and $p$.  Considering functors from $\Z_{\geq 0}^{op}$ into Diagram \ref{dia:reladj}, we obtain a new diagram 
$$
\begin{tikzcd}
\Fun(\Z^{op}_{\geq 0},\Sp^{\J}\times \Z^{op}_{\geq 0}) \arrow[rr, bend left=10,"P^{\text{total}}_*"] \arrow[rd,"q_*"]& &\Fun(\Z^{op}_{\geq 0},(\Sp^{\J}\times \Z^{op}_{\geq 0})_{\text{poly}}) \arrow[ld,"p_*"]\arrow[ll,"i_*"]  \\
& \Fun(\Z^{op}_{\geq 0}, \Z^{op}_{\geq 0})
\end{tikzcd}
$$
which exhibits $P^{\text{total}}_*$ as a left adjoint of $i_*$ relative to $\Fun(\Z_{\geq 0}^{op},\Z_{\geq 0}^{op}).$  Proposition 7.3.2.5 of \cite{HA} ensures that there is an adjunction at the level of fibers above $\text{id}\in \Fun(\Z_{\geq 0}^{op},\Z_{\geq 0}^{op})$: $$\mathcal{P}: \Cofil(\Sp^{\J}) \xrightleftharpoons{\quad} \Cofil(\Sp^{\J})_{\text{poly}}:j .$$  
Finally, observe that the unique functor $r: \Z^{op}_{\geq 0} \to *$ induces an adjunction $$r^*: \Sp^{\J} \xrightleftharpoons{\quad} \Cofil(\Sp^{\J}): \lim$$ where $r^*$ is the constant functor and $\lim$ is the same as right Kan extension along $r$.  We now compose these adjunctions, denoting $\mathrm{Tow} = \mathcal{P}\circ r^*$ to obtain: $$\mathrm{Tow}: \Sp^{\J} \xrightleftharpoons{\quad}  \Cofil(\Sp^{\J})_{\text{poly}}: \lim.$$

%It remains to check that $\mathrm{Tow}$ actually recovers the Weiss tower.  **TBD**.  Maybe let's just say this:
By construction, $\mathrm{Tow}(F)$ is the cofiltered spectrum $$P_0 F\longleftarrow P_1 F \longleftarrow P_2 F \longleftarrow \cdots .$$  This concludes the construction of $\mathrm{Tow}.$
%Tow is the right functor...what does that mean?  It certainly does the right thing on objects... might want the morphisms to be the natural one.  Certainly it can't be anything else, but seems like it might be a hassle to check this.  I guess maybe we only care about the individual morphisms...so that's doable but perhaps long :(
\end{cnstr}

\subsection{Multiplicativity of Tow}
The next task is to understand the multiplicative structure of $\mathrm{Tow}$.  The idea is that we would like to express $\mathrm{Tow}(F\wedge F)$ in terms of $\mathrm{Tow}(F)$ and the Day convolution monoidal structure on $\Cofil(\Sp^{\J}).$  We start with the following lemma:  

\begin{lem}
The Weiss tower functor $\mathrm{Tow}$ defines an oplax symmetric monoidal functor $$\mathrm{Tow}: \Sp^{\J} \to \Cofil(\Sp^{\J}).$$
\end{lem}
\begin{proof}
Recall that the Weiss tower functor was defined as a composite $\mathrm{Tow} = \mathcal{P} \circ r^*.$  The functor $r^*$ is just the constant functor, so it is symmetric monoidal.  On the other hand, $\mathcal{P}$ is adjoint to the inclusion $j: \Cofil(\Sp^{\J})_{\text{poly}} \to \Cofil(\Sp^{\J}).$  Since the class of functors which are polynomial of degree $n$ is closed under finite limits, the subcategory $(\Cofil(\Sp^{\J}))_{\text{poly}}$ is closed under the convolution tensor product.  We may therefore give it the structure of a symmetric monoidal $\infty$-category such that $j$ is symmetric monoidal.  This makes the left adjoint $\mathcal{P}$ an oplax symmetric monoidal functor, which induces an oplax symmetric monoidal structure on $\mathrm{Tow}.$  \end{proof}

Concretely, this oplax structure can be described on the $n$th filtered piece as follows: suppose $F,G \in \Sp^{\J}$; since $\mathrm{Tow}(F) \otimes \mathrm{Tow}(G) \in  \Cofil(\Sp^{\J})_{\text{poly}}$, the filtered piece $(\mathrm{Tow}(F) \otimes \mathrm{Tow}(G))_n$ is polynomial of degree $n$.  It follows that the natural map from $F\wedge G$ factors through a map $\varphi_n: P_n(F\wedge G)\to (\mathrm{Tow}(F) \otimes \mathrm{Tow}(G))_n$.  

In order to show that $\mathrm{Tow}$ is a symmetric monoidal functor, one would need to show that each $\varphi_n$ is an equivalence for all $n$.  We will do this after restricting to a smaller subcategory of functors with nice convergence properties:

\begin{dfn}
Let $F\in \Sp^{\J}$ be a functor.  Call $F$ \emph{rapidly convergent} if $F$ takes values in connective spectra and there exist real numbers $c,\alpha>0$ such that the natural map $F(W) \to P_nF(W)$ is $(\alpha n)\text{dim}(W) - c$ connected.  We denote by $\Sp^{\J}_{\text{conv}}$ the category of rapidly convergent functors.  
\end{dfn}

\begin{exm} \label{ex:aronefunctor}
Let $V\in \J$ be a complex vector space.  
The functor $F_V\in \Sp^{\J}$ defined by $$F_V(W) = \Sigma^{\infty}_+ \Omega \J(V, V\oplus W)$$ is rapidly convergent.  Indeed, \cite{Arone} shows that its homogeneous layers are given by $$D_nF_V(W) = \Omega^{\infty}(s_n(V) \smsh S^{nW})_{hU(n)}$$ where $s_n(V)$ is the suspension spectrum of a space.  Since colimits do not lower connectivity, this implies $D_nF_V(W)$ is at least $(2n)\text{dim}(W)-1$-connected.  It follows from the Milnor sequence that $F_V$ is rapidly convergent, where $\alpha$ can be taken to be $2$.  
\end{exm}

Observe that rapidly convergent functors in particular have convergent Weiss towers.  However, the following lemma of Weiss about functors ``agreeing up to order $n$'' allows us to say more:

\begin{lem}[\cite{WeissErratum}]\label{lem:ordernagree}
Let $F,G\in \Sp^{\J}$ be functors, $\eta: F\to G$ a natural transformation, and $n\geq 0$ an integer.  Suppose that there exists $c>0$ such that for all $W\in \J$, the map of spectra $\eta_W: F(W) \to G(W)$ is $(n+1)\text{dim}(W) -c$ connected.  Then the natural transformation $P_n\eta: P_n F\to P_n G$ is an equivalence.  
\end{lem}

The following corollary is immediate:

\begin{cor} \label{cor:rapidconv}
Let $F,G\in \Sp^{\J}_{\text{conv}}$ be rapidly convergent and $n>0$ an integer.  Then there exists an integer $M$ such that for $m>M$, the natural transformation $F\smsh G \to P_m F\smsh P_m G$ is an equivalence after applying $P_k$ for all integers $0\leq k \leq n.$  
\end{cor}

We will show that this implies the following further corollary:

\begin{cor}\label{cor:varphiequiv}
The map $\varphi_n$ constructed above is an equivalence for all $n$ when $F,G\in \Sp^{\J}_{\text{conv}}$ are rapidly convergent functors.  
\end{cor}
The proof will require the following basic lemma whose proof we will record at the end of this section: 
\begin{lem}\label{lem:cubes}
Let $X$, $Y \in \Cofil(\Sp^{\J})$ and $n>0$ an integer.  Then we have the following formula for the successive fibers: $$\fib ((X\otimes Y)_{n} \to (X\otimes Y)_{n-1}) \simeq \coprod_{p+q=n} \fib (X_p\to X_{p-1}) \wedge \fib (Y_q \to Y_{q-1}).$$
\end{lem}

We now prove the corollary:

\begin{proof}[Proof of Corollary \ref{cor:varphiequiv}]
Corollary \ref{cor:varphiequiv} implies that by replacing $F$ and $G$ by appropriate polynomial approximations, it suffices to consider the case where $F$ and $G$ are polynomial of degree $m$ for some $m$ (and thus, have finite Weiss towers).   Note further that Lemma \ref{lem:cubes} applied to the case $X = \mathrm{Tow}(F)$, $Y=\mathrm{Tow}(G)$ implies that $$\text{fib}((\mathrm{Tow}(F)\otimes \mathrm{Tow}(G) )_{n+1} \to (\mathrm{Tow}(F) \otimes \mathrm{Tow}(G))_n)$$ is homogeneous of degree $n+1$.  It follows that the fiber of the natural map $$F\smsh G \to (\mathrm{Tow}(F)\otimes \mathrm{Tow}(G))_n$$  is a finite limit of functors killed by $P_n$.  Since $P_n$ commutes with finite limits, we conclude that the natural map $\varphi_n: P_n(F\smsh G)\to (\mathrm{Tow}(F)\otimes \mathrm{Tow}(G))_n$ is an equivalence.  
\end{proof}

The proof shows further that rapidly convergent functors are closed under the tensor product.  In total, we have now proven the following theorem:

\begin{thm}\label{thm:weissmonoidal}
The Weiss tower defines a symmetric monoidal functor $$\mathrm{Tow}: \Sp^{\J}_{\text{conv}} \to \Cofil(\Sp^{\J}_{\text{conv}}).$$
\end{thm}
\begin{rmk}\label{rmk:goodwilliecase}
Theorem \ref{thm:weissmonoidal} and its proof work equally well in Goodwillie calculus.  There, the hypothesis on convergence can be removed, and the adjoining discussion can be replaced with the observation \cite[Lemma 6.10]{GoodwillieIII} that the smash product of an $n$-reduced functor and an $m$-reduced functor is $(n+m)$-reduced.  In Weiss calculus, this fact is also true but not in the literature, so we have opted to give the above proof which is sufficient for our application.  
\end{rmk}

Combining this with Example \ref{ex:aronefunctor}, we obtain:

\begin{cor}\label{cor:aronemonoidal}
Let $V\in \J$ be a complex vector space.  
The functor $F_V\in \Sp^{\J}$ defined by $$F_V(W) = \Sigma^{\infty}_+ \Omega \J(V, V\oplus W)$$ determines a cofiltered associative algebra $\mathrm{Tow}(F_V) \in \Alg_{\mathbb{A}_{\infty}}(\Cofil(\Sp^{\J})).$
\end{cor}


We finish with a proof omitted earlier:
\begin{proof}[Proof of Lemma \ref{lem:cubes}]
Let $A_n\subset \Z^{op}_{\geq 0} \times \Z^{op}_{\geq 0} $ be the full subcategory spanned by pairs $(p,q)$ with $p+q \leq n$.  

Define a functor $T:\Z^{op}_{\geq 0} \times \Z^{op}_{\geq 0}  \to \Sp^{\J}$ by the formula $T(p,q) = X_p \wedge Y_q.$  We have by definition that $\lim T|_{A_n} \simeq (X\otimes Y)_{n}$.%, so we wish to calculate the fiber of the natural map $\lim T|_{A_n} \to \lim T|_{A_{n-1}}.$  

Define the functor $\overline{T_n}: A_n \to \Sp^{\J}$ as the right Kan extension of $T|_{A_{n-1}}$ along the inclusion $A_{n-1} \to A_{n}.$   Then, $\overline{T_n}$ has the following properties:
\begin{enumerate}
\item $\lim \overline{T_n} = \lim T|_{A_{n-1}}.$
\item $\overline{T_n}|_{A_{n-1}} = T|_{A_{n-1}}.$
\item $\overline{T_n}(n,0) = X_{n-1} \wedge Y_0$ and $\overline{T_n}(0,n) = X_0\wedge Y_{n-1}.$
\item $\overline{T_n}(p,q) = X_{p-1}\wedge Y_{q} \times_{X_{p-1}\wedge Y_{q-1}} X_p\wedge Y_{q-1}$ for $p+q=n$, $p,q\geq 1$.  
\end{enumerate}

We may therefore compute $$\text{fib}\big(\lim_{A_n} T|_{A_n} \to \lim_{A_{n-1}} T|_{A_{n-1}}\big) = \text{fib}\big(\lim_{A_n} T|_{A_n} \to \lim_{A_n} \overline{T_n}\big) = \lim_{A_n} \text{fib}\big(T|_{A_n} \to \overline{T_n}\big).$$

The conclusion now follows immediately from the usual fact that $$\text{fib}(X_p\wedge Y_q \to X_{p-1}\wedge Y_{q} \times_{X_{p-1}\wedge Y_{q-1}} X_p\wedge Y_{q-1}) = \text{fib}(X_p\to X_{p-1}) \wedge \text{fib}(Y_q \to Y_{q-1}).$$
\end{proof}

