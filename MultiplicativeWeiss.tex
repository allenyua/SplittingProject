

In this section, we will briefly review notions of Weiss calculus to set notation and then prove a statement about its multiplicative properties.  The reader is referred to \cite{Weiss} for proofs and additional details.  We note that the discussion there is in the case of real vector spaces, but the results work just the same in the complex case.  We shall also work in the language of $\infty$-categories rather than topological categories, and Remark \ref{rmk:infinityweiss} justifies this passage.  

Let $\J$ be the $\infty$-category which is the nerve of the topological category whose objects are finite dimensional complex vector spaces equipped with a Hermitian inner product and whose morphisms are spaces of linear isometries.  

The theory of Weiss calculus studies functors out of $\J$ in a way analogous to Goodwillie calculus, by understanding successive ``polynomial approximations'' to these functors.  Here, we will discuss only the stable setting where we apply the theory to the functor category $\Sp^{\J}$. The central definition is:

\begin{dfn}\label{dfn:polyfun}
A functor $F\in \Sp^{\J}$ is polynomial of degree $n$ if the natural map $$F(V) \to \lim_U F(U\oplus V)$$ is an equivalence, where the limit is indexed over the $\infty$-category of nonzero subspaces $U\subset \mathbb{C}^{n+1}.$
\end{dfn}

As in Goodwillie calculus, the inclusion of the full subcategory $\Poly^{\leq n}(\Sp^{\J}) \subset \Sp^{\J}$ of functors which are polynomial of degree $n$ admits a left adjoint $$P_n: \Sp^{\J} \xrightleftharpoons{\quad} \Poly^{\leq n}(\Sp^{\J}): j_n.$$ %center this...
 The unit $\eta_n$ of this adjunction provides for each $F\in \Sp^{\J}$ a natural transformation $F \to P_nF$ which we will refer to as the \emph{degree $n$ polynomial approximation} of $F$. 

\begin{rmk}\label{rmk:infinityweiss}
This universal property was not explicitly stated in \cite{Weiss}, but it follows formally from Weiss's results as follows: the functor $P_n$ and the transformation $\eta_n$ can be defined explicitly as in \cite{Weiss} by iteratively applying the functor $\tau_n: \Sp^{\J} \to \Sp^{\J}$ defined by the formula $$\tau_n F(V) = \lim_U F(U\oplus V)$$ with the limit indexed as in Definition \ref{dfn:polyfun}.   The facts required of the functors $P_n$ in the proof of Theorem 6.1.1.10 in \cite{HA} are precisely the content of Theorem 6.3 of \cite{Weiss}.  
\end{rmk}

Given this universal property, Proposition 5.4 of \cite{Weiss} ensures the existence of a natural Taylor tower $$F \longrightarrow \cdots \longrightarrow P_{n} F \xrightarrow{p_{n-1}} P_{n-1} F \longrightarrow \cdots \longrightarrow P_0F$$ living under any functor $F\in \Sp^{\J}.$  The fiber $D_n F$ of $p_{n-1}$ has the special property that it is polynomial of degree $n$ and $P_{n-1} D_n F \simeq 0$.  Such a functor is called \emph{$n$-homogeneous}; such functors are completely classified by the following theorem:

\begin{thm}[{{\cite[Theorem 7.3]{Weiss}}}]
Let $F\in \Sp^{\J}$.  Then $F$ is an $n$-homogeneous functor if and only if there exists a spectrum $\Theta$ with an action of the unitary group $U(n)$ such that $$F(V) = (\Theta \wedge S^{nV})_{hU(n)}.$$
\end{thm}


The observation of Goodwillie, as exploited by \cite{Arone}, is that this provides a canonical way to split certain functorial filtrations whose successive quotients are homogeneous.  More precisely, we have the following theorem:

\begin{thm}[\cite{Arone}]\label{thm:aronesplit}
Suppose $F \in \Sp^{\J}$ is a functor together with an increasing filtration $$0 = F^{(0)} \longrightarrow F^{(1)}\longrightarrow F^{(2)} \longrightarrow  \cdots F$$ by functors $F^{(i)}\in \Sp^{\J}$ with the property that the successive quotients $F^{(n)}/F^{(n-1)}$ are $n$-homogeneous for all integers $n>0$.  Then, each functor $F^{(n)}$ is polynomial of degree $n$ and each composite $F^{(n-1)} \longrightarrow F^{(n)} \xrightarrow{\eta_{n-1}} P_{n-1} F^{(n)}$ is an equivalence.
\end{thm}
%we need to at some point deal with the fact that Weiss doesn't actually deal with homog. functors to spectra

In \cite{Arone}, this result is applied to the functor $F\in \Sp^{\J}$ defined by the formula $$F_V(W) = \Sigma^{\infty}_+ \Omega \J(V,V\oplus W)$$ where $V\in \J$ is a fixed finite dimensional complex vector space.  Arone provides a filtration $F^{(0)}_V(W) \subset F^{(1)}_V(W) \subset \cdots F_V(W)$ which is functorial in both $V$ and $W$, and which satisfies the constraints of Theorem \ref{thm:aronesplit} for fixed $V$.  This provides a stable splitting of the space $\Omega \J(V,V\oplus W).$  Letting $W=\mathbb{C}$ and $V=\mathbb{C}^{n-1}$, we obtain splittings of the loop groups $\Omega SU(n)$, and for higher dimension $W$, this provides splittings of the loop spaces of Stiefel manifolds.  


In order to upgrade the results of \cite{Arone} to structured multiplicative splittings, we must understand the multiplicative properties of the polynomial approximation functors.  More precisely, for a functor $F\in \Sp^{\J}$, we aim to understand the Taylor tower of $F\wedge F$ in terms of the tower for $F.$  The results in this section are likely known to experts, but the authors were not able to locate it in the literature.  They thank Jacob Lurie for suggesting that Proposition \ref{prop:weissmonoidal} is true.  

The idea is to consider all the polynomial approximations at once.  To do so, we first set some additional notation:

\begin{dfn} 
Let $\Cofil(\Sp^{\J})$ denote the functor category $\Fun(\Z_{\geq 0}^{op}, \Sp^{\J}).$  We shall refer to $\Cofil(\Sp^{\J})$ as the category of cofiltered objects in $\Sp^{\J}$.  Its objects can be thought of as towers of functors $Y_0\leftarrow Y_1\leftarrow Y_2 \leftarrow \cdots \in \Sp^{\J}$.
\end{dfn}%ok maybe I'll want cofil^+ or some garbage....

The category $\Cofil(\Sp^{\J})$ is the natural target for the Weiss tower.  The following construction makes this precise:

\begin{cnstr}\label{cnstr:tower}
We now construct a functor $$\text{Tow}: \Sp^{\J} \to \Cofil(\Sp^{\J})$$ with the property that it sends a functor $F\in \Sp^{\J}$ to its Taylor tower $$\text{Tow}(F) = P_0F \longleftarrow P_1F \longleftarrow P_2F \longleftarrow \cdots.$$


Recall that the $P_n$ functors are given as left adjoints of the fully faithful inclusions $\Poly^{\leq n}(\Sp^{\J}) \subset \Sp^{\J}$.  We proceed by telling a parametrized version of this story that includes all $n$ simultaneously.  The proper framework for such a story is the formalism of \emph{relative adjunctions}; these are developed in the $\infty$-categorical context in \cite{HA}, Section 7.3.2.  

Consider the category $\Sp^{\J}\times \Z^{op}_{\geq 0}$ together with the full subcategory $(\Sp^{\J}\times \Z^{op}_{\geq 0})_{\text{poly}} \subset \Sp^{\J}\times \Z^{op}_{\geq 0}$ on the pairs $(F, [n])$ such that $F\in \Poly^{\leq n}(\Sp^{\J}).$  Via projection, these fit into a diagram
$$
\begin{tikzcd}\label{dia:reladj}
\Sp^{\J}\times \Z^{op}_{\geq 0} \arrow[rd,"q"]& &(\Sp^{\J}\times \Z^{op}_{\geq 0})_{\text{poly}} \arrow[ld,"p"]\arrow[ll,"i"]  \\
& \Z^{op}_{\geq 0}
\end{tikzcd}
$$
 This will be relevant to us because the category of sections of $q$ are precisely $\Cofil(\Sp^{\J}).$  The sections of $p$ can be thought of those cofiltered functors such that the $n$th piece is polynomial of degree $n$.  We will denote this category of sections of $p$ by $\Cofil(\Sp^{\J})_{\text{poly}}.$ 

On the fibers over an integer $[n] \in \Z^{op}_{\geq 0}$, we see the inclusion $\Sp^{\J} \leftarrow \Poly^{\leq n}(\Sp^{\J}).$  It is in this sense that the current picture is a parametrized version of the ordinary polynomial approximations.  We now claim that $i$ admits a left adjoint $P^{\text{total}}: \Sp^{\J}\times \Z^{op}_{\geq 0} \to (\Sp^{\J}\times \Z^{op}_{\geq 0})_{\text{poly}}$ \emph{relative} to $\Z^{op}_{\geq 0}.$    The strategy is to use Proposition 7.3.2.6 of \cite{HA}, which tells us that we need to check the following three statements:
\begin{enumerate}
\item The functors $p$ and $q$ are locally Cartesian categorical fibrations.
\item For each $[n]\in \Z^{op}_{\geq 0}$, the functor on fibers $i|_{p^{-1}[n]}:p^{-1}[n] \to q^{-1}[n]$ admits a right adjoint.  
\item The functor $i$ carries locally $p$-Cartesian morphisms of $(\Sp^{\J}\times \Z^{op}_{\geq 0})_{\text{poly}}$ to locally $q$-Cartesian morphisms of $\Sp^{\J}\times \Z^{op}_{\geq 0}$.
\end{enumerate}

Condition (2) is clear from the existence of polynomial approximations in Weiss calculus.  To see conditions (1) and (3), we first note that $q$ is in fact a Cartesian fibration because it is a projection from a product.  Moreover, the $q$-Cartesian morphisms are precisely those morphisms which are equivalences on the $\Sp^{\J}$ coordinate.  We now observe that for any pair $(F, [m]) \in \Sp^{\J}\times \Z^{op}_{\geq 0}$ such that $F\in \Poly^{\leq m}(\Sp^{\J})$ and morphism $\sigma :[n]\to [m]$, any $q$-Cartesian edge lying over $\sigma$ with target $(F, [m])$ is also in the full subcategory $(\Sp^{\J}\times \Z^{op}_{\geq 0})_{\text{poly}}.$  This implies that $p$ is also a Cartesian fibration and that the inclusion $i$ carries $p$-Cartesian edges to $q$-Cartesian edges.  Since any Cartesian fibration is a categorical fibration (\cite[Proposition 3.3.1.7]{HTT}), conditions (1) and (3) are verified.  

We now wish to look at the adjunction at the level of sections of $q$ and $p$.  Considering functors from $\Z_{\geq 0}^{op}$ into Diagram \ref{dia:reladj}, we obtain a new diagram 
$$
\begin{tikzcd}
\Fun(\Z^{op}_{\geq 0},\Sp^{\J}\times \Z^{op}_{\geq 0}) \arrow[rr, bend left=10,"P^{\text{total}}_*"] \arrow[rd,"q_*"]& &\Fun(\Z^{op}_{\geq 0},(\Sp^{\J}\times \Z^{op}_{\geq 0})_{\text{poly}}) \arrow[ld,"p_*"]\arrow[ll,"i_*"]  \\
& \Fun(\Z^{op}_{\geq 0}, \Z^{op}_{\geq 0})
\end{tikzcd}
$$
which exhibits $P^{\text{total}}_*$ as a left adjoint of $i_*$ relative to $\Fun(\Z_{\geq 0}^{op},\Z_{\geq 0}^{op}).$  Proposition 7.3.2.5 of \cite{HA} ensures that there is an adjunction at the level of fibers above $\text{id}\in \Fun(\Z_{\geq 0}^{op},\Z_{\geq 0}^{op})$: $$\mathcal{P}: \Cofil(\Sp^{\J}) \xrightleftharpoons{\quad} \Cofil(\Sp^{\J})_{\text{poly}}:j .$$  
Finally, observe that the unique functor $r: \Z^{op}_{\geq 0} \to *$ induces an adjunction $$r^*: \Sp^{\J} \xrightleftharpoons{\quad} \Cofil(\Sp^{\J}): \lim$$ where $r^*$ is the constant functor and $\lim$ is the same as right Kan extension along $r$.  We now compose these adjunctions, denoting $\text{Tow} = \mathcal{P}\circ r^*$ to obtain: $$\text{Tow}: \Sp^{\J} \xrightleftharpoons{\quad}  \Cofil(\Sp^{\J})_{\text{poly}}: \lim.$$

It remains to check that $\text{Tow}$ actually recovers the Weiss tower.  TBD%Tow is the right functor...what does that mean?  It certainly does the right thing on objects... might want the morphisms to be the natural one.  Certainly it can't be anything else, but seems like it might be a hassle to check this.  I guess maybe we only care about the individual morphisms...so that's doable but perhaps long :(


This concludes the construction of $\text{Tow}.$ 
\end{cnstr}


The next task is to understand the monoidal structure on $\text{Tow}$.  The idea is that we would like to express $\text{Tow}(F\wedge F)$ in terms of $\text{Tow}(F)$ and a ``Day convolution'' monoidal structure on $\Cofil(\Sp^{\J}).$  
TODO: actually define it here
%actually this is something that you can do.  This provides a convolution product on \Cofil(\Sp^{\J}).  

We then have the following result about the monoidality of the Weiss tower functor.  

\begin{prop}\label{prop:weissmonoidal}
The Weiss tower functor $\text{Tow}$ defines a symmetric monoidal functor $$\text{Tow}: \Sp^{\J} \to \Cofil(\Sp^{\J}).$$
\end{prop}
\begin{proof}
Recall that the Weiss tower functor was defined as a composite $\text{Tow} = \mathcal{P} \circ r^*.$  The functor $r^*$ is just the constant functor, so it is symmetric monoidal.  On the other hand, $\mathcal{P}$ is adjoint to the inclusion $j: \Cofil(\Sp^{\J})_{\text{poly}} \to \Cofil(\Sp^{\J}).$  Since the class of functors which are polynomial of degree $n$ is closed under finite limits, the subcategory $\Cofil(\Sp^{\J})$ is closed under the convolution tensor product.  We may therefore give it the structure of a symmetric monoidal $\infty$-category such that $j$ is symmetric monoidal.  This makes the left adjoint $\mathcal{P}$ an oplax monoidal functor, which induces an oplax monoidal structure on $\text{Tow}.$  

Concretely, this oplax structure can be described on the $n$th filtered piece as follows: since $\text{Tow}(F) \in  \Cofil(\Sp^{\J})_{\text{poly}}$, the filtered piece $(\text{Tow}(F) \otimes \text{Tow}(F))_n$ is polynomial of degree $n$.  It follows that the natural map from $F\wedge F$ factors through a map $\varphi_n: P_n(F\wedge F)\to (\text{Tow}(F) \otimes \text{Tow}(F))_n$.  To see that $\text{Tow}$ is a symmetric monoidal functor, it suffices to show that each $\varphi_n$ is an equivalence.   TODO
%need to check the oplax structure map is an equivalence; do the cubes thing...

\end{proof}

\begin{rmk}
Proposition \ref{prop:weissmonoidal} is written in the language of Weiss calculus as that is the present application, but the proof works equally well in Goodwillie calculus.  
\end{rmk}
