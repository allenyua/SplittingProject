\label{sect:monoidal2}

Here we discuss some additional constructions and results that we will need for the more technical parts of this paper.  %at some point, warn that we'll have to use infinity operads

The monoidal structures on our categories will arise from Day convolution.  This was studied for $\infty$-categories by Glasman \cite{Glasman} and Lurie \cite{LurieRot, HA} at varying levels of generality.  We will find it convenient to use the formulation from Section 2.2.6 of \cite{HA}.  

\begin{thm}[\cite{HA}, Example 2.2.6.9]
Let $\C$ and $\D$ be symmetric monoidal $\infty$-categories.  Then there is an $\infty$-operad $\Fun(\C, \D)^{\otimes} $ with the following properties:
\begin{enumerate}
\item The underlying $\infty$-category of $\Fun(\C,\D)^{\otimes}$ is the functor category $\Fun(\C, \D)$.
\item The $\infty$-category $\Alg_{\E_\infty}(\Fun(\C, \D)^{\otimes})$ of $\E_\infty$ algebras in $\Fun(\C,\D)^{\otimes}$ is equivalent to the category of lax symmetric monoidal functors from $\C$ to $\D$.  

\end{enumerate}
\end{thm}

In order for the $\infty$-operad $\Fun(\C,\D)^{\otimes}$ to actually be a symmetric monoidal $\infty$-category, one needs to make additional assumptions.  

\begin{prop}[\cite{HA}, Proposition 2.2.6.16]\label{prop:dayconvsmc}
Let $\C$ and $\D$ be symmetric monoidal $\infty$-categories.  Suppose that $\kappa$ is an uncountable regular cardinal such that:
\begin{enumerate}
\item $\C$ is essentially $\kappa$-small.
\item $\D$ admits $\kappa$-small colimits.
\item The tensor product on $\D$ preserves $\kappa$-small colimits separately in each variable.  
\end{enumerate}
Then $\Fun(\C,\D)^{\otimes}$ is a symmetric monoidal $\infty$-category.  
\end{prop}

Recall that the Day convolution is defined classically via left Kan extension.  Assumptions (1) and (2) ensure that the relevant Kan extensions exist.  Assumption (3) then ensures that the multiplication is associative by allowing the colimits taken in the formula for left Kan extension to commute with the tensor product.  

As stated before, Proposition \ref{prop:dayconvsmc} is sufficient to construct symmetric monoidal $\infty$-categories $\Fil(\Sp)$ and $\Gr(\Sp)$.  However, we wish to understand the interaction of the Weiss calculus with multiplicative structure; there, the filtrations go the other way.


We would like to make $\Cofil(\Sp)$ a symmetric monoidal $\infty$-category by putting the Day convolution on its opposite, $\Fun(\Z_{\geq 0}, \Sp^{op}).$  However, the smash product of spectra does not preserve small colimits separately in each variable.  Nevertheless, it does preserve \emph{finite} colimits separately in each variable.  In fact, these are the only colimits that are needed in the case at hand and so we have the following variant of Proposition \ref{prop:dayconvsmc}:

\begin{var}\label{var:day}
Let $\C$ and $\D$ be symmetric monoidal $\infty$-categories.  Suppose that:
\begin{enumerate}
\item Let $I$ be a nonempty finite set and consider the multiplication map $\Pi_{i\in I} \C \to \C$.  For every $C\in \C$, the slice category $\Pi_{i\in I}\C \times_{\C} \C_{/C}$ has a finite cofinal subcategory.  
\item $\D$ admits finite colimits. 
\item The tensor product on $\D$ preserves finite colimits separately in each variable.  
\end{enumerate}
Then $\Fun(\C, \D)^{\otimes}$ is a symmetric monoidal $\infty$-category.  
\end{var}
\begin{proof}
This follows directly from the same arguments as Proposition \ref{prop:dayconvsmc}.  In \cite[Corollary 2.2.6.14]{HA}, the assumptions are used to guarantee the existence of a left Kan extension; this again exists by assumptions (1) and (2) and \cite[Lemma 4.3.2.13]{HTT}.  Similarly, the proof of \cite[Proposition 2.2.6.16]{HA} only makes reference to commuting tensor products in $\D$ with finite colimits, which is ensured by assumption (3).  
\end{proof}

In Section \ref{app:SplittingMachine}, we will need to consider not only the Day convolution monoidal structure on $\Fun(\C,\D)$ but its functoriality as $\C$ varies.  For instance, we would for symmetric monoidal functors $\C_1 \to \C_2$ to induce symmetric monoidal functors $\Fun(\C_1,\D) \to \Fun(\C_2,\D)$ via left Kan extension.  

We give a very close variant of \cite[Corollary 3.8]{Nikolaus} in our current framework:

\begin{prop}\label{prop:kanmonoidal}
Let $\C_1$, $\C_2$, and $\D$ be symmetric monoidal $\infty$-categories and let $f:\C_1 \to \C_2$ be a symmetric monoidal functor.  Suppose that one of the following conditions hold:
\begin{enumerate}
\item The pairs $(\C_1, \D)$ and $(\C_2, \D)$ satisfy the hypotheses of Proposition \ref{prop:dayconvsmc}
\item The pairs $(\C_1, \D)$ and $(\C_2, \D)$ satisfy the hypotheses of Variant \ref{var:day} and for any object $c\in \C_2$, the slice category $\C_1 \times_{\C_2} {\C_2}_{/c}$ has a finite cofinal subset.  
\end{enumerate}
   Then there is an adjunction 
$$ f_! : \Fun(\C_1, \D) \xrightleftharpoons{\quad} \Fun(\C_2, \D): f_* $$ %okay need to be careful here
where $f_*$ denotes restriction and $f_!$ denotes left Kan extension.  Moreover, the functor $f_*$ is lax symmetric monoidal and $f_!$ is symmetric monoidal.  
\end{prop}
\begin{proof}
The universal property of $\Fun(\C_1, \D)^{\otimes}$ immediately implies the existence of a map of $\infty$-operads $\Fun(\C_2,\D)^{\otimes} \to \Fun(\C_1,\D)^{\otimes}$, which makes $f_*$ a lax symmetric monoidal functor.  

Assumptions (1) and (2) of Proposition \ref{prop:dayconvsmc} guarantee that the adjunction exists at the level of $\infty$-categories.  The rest of the proof from \cite[Corollary 3.8]{Nikolaus} carries over verbatim.  
\end{proof}


%should check I wrote lax *symmetric* monoidal, not just lax monoidal everywhere
