\documentclass[reqno, oneside]{amsart}
\usepackage{etex}
\usepackage{hyperref}

\usepackage{chemarr}
\usepackage{amssymb}
\usepackage{comment}

\usepackage[a4paper]{geometry}


%----------------------------------------------------------------------%


\theoremstyle{definition}
\newtheorem{nul}{}[section]
\newtheorem{dfn}[nul]{Definition}
\newtheorem{axm}[nul]{Axiom}
\newtheorem{rmk}[nul]{Remark}
\newtheorem{term}[nul]{Terminology}
\newtheorem{cnstr}[nul]{Construction}
\newtheorem{cnv}[nul]{Convention}
\newtheorem{ntn}[nul]{Notation}
\newtheorem{exm}[nul]{Example}
\newtheorem{obs}[nul]{Observation}
\newtheorem{ctrexm}[nul]{Counterexample}
\newtheorem{rec}[nul]{Recollection}
\newtheorem{exr}[nul]{Exercise}
\newtheorem{wrn}[nul]{Warning}
\newtheorem{qst}{Question}
\newtheorem*{dfn*}{Definition}
\newtheorem*{axm*}{Axiom}
\newtheorem*{ntn*}{Notation}
\newtheorem*{exm*}{Example}
\newtheorem*{exr*}{Exercise}
\newtheorem*{int*}{Intuition}
\newtheorem*{qst*}{Question}
\newtheorem*{rmk*}{Remark}
\newtheorem*{comp1}{Computation 1:}

\theoremstyle{plain}
\newtheorem{sch}[nul]{Scholium}
\newtheorem{claim}[nul]{Claim}
\newtheorem{thm}[nul]{Theorem}
\newtheorem{prop}[nul]{Proposition}
\newtheorem{lem}[nul]{Lemma}
\newtheorem{var}[nul]{Variant}
\newtheorem{sublem}{Lemma}[nul]
\newtheorem{por}[nul]{Porism}
\newtheorem{cnj}[nul]{Conjecture}
\newtheorem{cor}{Corollary}[nul]
\newtheorem*{thm*}{Theorem}
\newtheorem*{prop*}{Proposition}
\newtheorem*{cor*}{Corollary}
\newtheorem*{lem*}{Lemma}
\newtheorem*{cnj*}{Conjecture}
\newtheorem{innercustomthm}{Theorem}
\newenvironment{customthm}[1]
  {\renewcommand\theinnercustomthm{#1}\innercustomthm}
  {\endinnercustomthm}



%----------------------------------------------------------------------%

\DeclareMathOperator{\Aut}{\text{Aut}}
\DeclareMathOperator{\Tr}{\text{Tr}}
\DeclareMathOperator{\Res}{\text{Res}}
\DeclareMathOperator{\im}{\text{im}}
\DeclareMathOperator*{\colim}{\mathrm{colim}}
\DeclareMathOperator{\Map}{\text{Map}}
\DeclareMathOperator{\cofiber}{\text{cofiber}}
\DeclareMathOperator{\fiber}{\text{fiber}}
\DeclareMathOperator{\gr}{\mathrm{gr}}
\DeclareMathOperator{\fib}{\mathrm{fib}}
\DeclareMathOperator{\Hom}{\text{Hom}} 
\DeclareMathOperator{\Skel}{\text{Skel}}
\DeclareMathOperator*{\hocolim}{\text{hocolim}}
\DeclareMathOperator*{\holim}{\text{holim}}
\DeclareMathOperator{\smsh}{\wedge}



\DeclareMathOperator{\C}{\mathcal{C}}
\DeclareMathOperator{\D}{\mathcal{D}}
\DeclareMathOperator{\CP}{\mathbb{CP}}
\DeclareMathOperator{\Z}{\mathbb{Z}}
\DeclareMathOperator{\E}{\mathbb{E}}
\DeclareMathOperator{\mE}{\mathcal{E}}
\DeclareMathOperator{\N}{\mathrm{N}}
\DeclareMathOperator{\Q}{\mathbb{Q}}
\DeclareMathOperator{\m}{\mathfrak{m}}
\DeclareMathOperator{\G}{\mathbb{G}}
\DeclareMathOperator{\F}{\mathbb{F}}
\DeclareMathOperator{\cG}{\mathcal{G}}
\DeclareMathOperator{\cF}{\mathcal{F}}
\DeclareMathOperator{\cS}{\mathcal{S}}
\DeclareMathOperator{\Ring}{\textbf{Ring}}
\DeclareMathOperator{\Aff}{\textbf{Aff}}
\DeclareMathOperator{\Ran}{\mathrm{Ran}}
\DeclareMathOperator{\Spec}{\text{Spec}}
\DeclareMathOperator{\Poly}{\text{Poly}}
\DeclareMathOperator{\pic}{\mathrm{pic}}
\DeclareMathOperator{\Pic}{\mathrm{Pic}}
\DeclareMathOperator{\Fin}{\mathrm{Fin}}
\DeclareMathOperator{\Ext}{\text{Ext}}
\DeclareMathOperator{\Gra}{\mathrm{Gr}}
\DeclareMathOperator{\nil}{\text{nil}}
\DeclareMathOperator{\ac}{\tilde{c}}
\DeclareMathOperator{\fin}{\text{fin}}
\DeclareMathOperator{\Gr}{\mathbf{Gr}}
\DeclareMathOperator{\Fil}{\mathbf{Fil}}
\DeclareMathOperator{\Fun}{\text{Fun}}
\DeclareMathOperator{\Alg}{\mathrm{Alg}}
\DeclareMathOperator{\Sp}{\mathrm{Sp}}
\DeclareMathOperator{\Spaces}{\mathcal{S}}
\DeclareMathOperator{\J}{\mathcal{J}}
\DeclareMathOperator{\Cofil}{\mathbf{Cofil}}

%----------------------------------------------------------------------%

\usepackage{tikz}
\usetikzlibrary{matrix,arrows,decorations}
\usepackage{tikz-cd}

\usepackage{adjustbox}

%----------------------------------------------------------------------%



\begin{document}

\title{Exotic Multiplications on Periodic Complex Bordism}
\author{Jeremy Hahn}
\address{Department of Mathematics, Massachusetts Institute of Technology, Cambridge, MA 02139}
\email{jhahn01@mit.edu}

\author{Allen Yuan}
\address{Department of Mathematics, Massachusetts Institute of Technology, Cambridge, MA 02139}
\email{alleny@mit.edu}

\begin{abstract}
Victor Snaith gave a construction of periodic complex bordism by inverting the Bott element in the suspension spectrum of $BU$.  This presents an $\mathbb{E}_\infty$ structure on periodic complex bordism by different means than the usual Thom spectrum definition of the $\mathbb{E}_\infty$-ring $MUP$.  Here we prove that these two $\mathbb{E}_\infty$-rings are in fact different, though the underlying $\mathbb{E}_2$-rings are equivalent.  Nonetheless, we prove that both $\mathbb{E}_\infty$-rings orient all height $1$ Morava $E$-theories at the prime $2$.
\end{abstract}



%----------------------------------------------------------------------%
%----------------------------------------------------------------------%


\setcounter{tocdepth}{1}
\maketitle

\tableofcontents

\vbadness 5000

%----------------------------------------------------------------------%

\section{Introduction}

We study the homotopy type of the affine Grassmannian of $SL_n(\mathbb{C})$, which is equivalent to the space $\Omega SU(n)$ of based loops in $SU(n)$.  There are essentially two multiplications on this homotopy type, one arising from the composition of loops and the other from the group multiplication on $SL_n(\mathbb{C})$.  Together, these two multiplications interact to give $\Omega SU(n)$ the structure of an $\mathbb{E}_2$ or fusion algebra.  In geometric representation theory, this structure is witnessed by the existence of the Beilinson--Drinfeld Grassmannian.

Using either of the above (homotopy equivalent) products, it is possible to make $H_*(\Omega SU(n);\mathbb{Z})$ into a graded ring.  To describe this ring, let us first name some of its elements.  For each one-dimensional subspace $V \subset \mathbb{C}^n$, there is a loop $\lambda_V:S^1 \rightarrow U(n)$ given by the formula
$$\lambda_V(z)=\left( \begin{array}{cc} z & 0 \\ 0 & I \end{array} \right),$$
with the matrix presented in terms of the decomposition $\mathbb{C}^n \cong V \oplus V^{\perp}$.  Fixing a particular line $W \subset \mathbb{C}^n$, the construction $V \mapsto \lambda_W^{-1} \cdot \lambda_V$ defines a well-known map
$$\mathbb{CP}^{n-1} \rightarrow \Omega SU(n).$$
Let $b_1,b_2,...,b_{n-1}$, $|b_i|=2i$, denote the images in $H_*(\Omega SU(n);\mathbb{Z})$ of the generators of $H_*(\mathbb{CP}^{n-1};\mathbb{Z})$.  It is a result of Bott \cite{Bott} that
$$H_*(\Omega SU(n);\mathbb{Z}) \cong \mathbb{Z}[b_1,b_2,\cdots],$$
with the latter denoting the polynomial algebra on the classes $b_i$.

Notice, in particular, that $H_*(\Omega SU(n);\mathbb{Z})$ is naturally a \textit{bigraded} ring, one grading being given by $*$ and the other by assigning each $b_i$ degree $1$.  Mahowald observed that the action of the Steenrod algebra on $H_*(\Omega SU(n);\mathbb{F}_2)$ preserves this second degree, and he conjectured a geometric splitting to be responsible.  Indeed, it was eventually proven by Mitchell and Richter \cite[Theorem 2.1]{CrabbMitchell}, that the suspension spectrum $\Sigma^{\infty} \Omega SU(n)$ splits as an infinite wedge sum:
$$\Sigma^{\infty}_+ \Omega SU(n) \simeq \mathbb{S} \vee \Sigma^{\infty} \mathbb{CP}^{n-1} \vee \cdots.$$

In order to prove this splitting, Mitchell \cite{MitchellSU(n)} (and, independently, Segal \cite{Segal}) first constructed a filtration of the space $\Omega SU(n)$.  Following Mitchell, we name this the \textit{Bott filtration} of $\Omega SU(n)$.  The first filtered piece is given by the above map $\mathbb{CP}^{n-1} \rightarrow \Omega SU(n)$, and the theorem of Mitchell and Richter is that the filtration stably splits.  The construction of the Bott filtration is somewhat involved, and we review it in Section \ref{sec:MRFil}--it is a subfiltration of the Bruhat ordering on (closures of) Iwahori orbits.

In Section \ref{sec:FilGra}, we review the symmetric monoidal structures on the ($\infty$)-categories of filtered and graded spectra.  This allows us to properly state our first main theorem, proven in Section \ref{sec:MRFil}:

\begin{thm} \label{thm:BottIsAoo}
The suspension of the Bott filtration 
$$\mathbb{S} \longrightarrow \Sigma_+^{\infty} \mathbb{CP}^{n-1} \simeq \Sigma_+^{\infty} F_{n,1} \longrightarrow \Sigma_+^{\infty} F_{n,2} \longrightarrow \cdots \longrightarrow \Sigma^{\infty}_+ \Omega SU(n).$$
is an $\mathbb{A}_\infty$-algebra object in filtered spectra.
\end{thm}

\begin{rmk}
The Bott filtration is multiplicative before suspension, but for technical reasons we prefer to phrase our results in terms of filtered spectra instead of filtered spaces.
\end{rmk}

\begin{qst}
Is the Bott filtration an $\mathbb{E}_2$ filtration?  We do not know the answer--for some thoughts about the problem, see Remark \ref{rmk:E2fil}.
\end{qst}

The proof of Theorem \ref{thm:BottIsAoo} is fairly straightforward, once given access to the sophisticated machinery behind the Beilinson--Drinfeld Grassmannian.  For example, we will explain in Section \ref{sec:MRFil} that this machinery immediately dispenses with a conjecture of Mahowald and Richter \cite{MahowaldRichter}.  Nonetheless, there are some subtleties involved, and it is these subtleties that prevent us from determining if the Bott filtration is $\mathbb{E}_2$.  The problem is readily visible in the case $n=\infty$:

\begin{exm}
The limiting case of the Bott filtration of $\Omega SU(n)$ as $n$ tends to $\infty$ is the filtration
$$* \longrightarrow BU(1) \longrightarrow BU(2) \longrightarrow BU(3) \longrightarrow \cdots \longrightarrow BU \simeq \Omega SU.$$
It is easy to see that $\coprod BU(n)$ is a graded $\mathbb{E}_2$-algebra in spaces (in fact, it is a graded $\mathbb{E}_\infty$-algebra, being the nerve of the category of vector spaces).  However, the filtered object is much more subtle.  For example, the squares
$$
\begin{tikzcd}
BU(i) \times BU(j) \arrow{d} \arrow{r} & BU(i) \times BU(j+1) \arrow{d} \\
BU(i+1) \times BU(j) \arrow{r} & BU(i+1) \times BU(j+1)
\end{tikzcd}
$$
do not commute on the nose, but only up to non-canonical homotopy.
\end{exm}

The rest of the paper is concerned with the stable splitting of this Bott filtration.  Our main results are as follows:

\begin{thm} \label{thm:MainAoo}
As an $\mathbb{A}_\infty$-algebra object in filtered spectra, the Bott filtration of $\Sigma^{\infty}_+ \Omega SU(n)$ is equivalent to its associated graded.
\end{thm}

\begin{cor}
For any homology theory $E$, $E_*(\Omega SU(n))$ is a bigraded ring.  One grading is given by $*$, and the other by the associated graded of the Bott filtration.
\end{cor}

\begin{thm} \label{thm:MainObstruction}
Suppose $n \ge 4$.  If the Bott filtration of $\Sigma^{\infty}_+ \Omega SU(n)$ may be made into an $\mathbb{E}_2$-algebra object in filtered spectra, then it is \textbf{not} equivalent to its $\mathbb{E}_2$ associated graded.
\end{thm}

\begin{thm} \label{thm:MainMUE2}
Let $MU$ denote the $\mathbb{E}_\infty$-ring spectrum of complex bordism, and let $\text{gr}(\Sigma^{\infty}_+\{F_{n,k}\})$ denote the associated graded of the Bott filtration of $\Sigma^{\infty}_+ \Omega SU(n)$.  Then
\begin{enumerate}
\item There exists a graded $\mathbb{E}_2$-algebra structure on the graded spectrum $\text{gr}(\Sigma^{\infty}_+ \{F_{n,k}\})$ that extends the canonical graded $\mathbb{A}_\infty$-algebra structure.

\item For any $\mathbb{E}_2$-algebra structure on the underlying (ungraded) $\mathbb{A}_\infty$-ring $\text{gr}(\Sigma^{\infty}_+\{F_{n,k}\})$, there is an equivalence of $\mathbb{E}_2$-$MU$-algebras
$$MU \smsh \Sigma^{\infty}_+ \Omega SU(n) \simeq MU \smsh \text{gr}(\Sigma^{\infty}_+\{F_{n,k}\}).$$
\end{enumerate}
\end{thm}

\begin{rmk}
There exist exotic $\mathbb{E}_2$-algebra structures on $gr(\Sigma^{\infty}_+ \{F_{n,k}\})$ before smashing with $MU$.  For example, as $n$ tends to infinity we recover the Snaith splitting \cite{SnaithBook}
$$\Sigma^{\infty}_+ BU \simeq \bigvee_n MU(n),$$
where $MU(n)$ is the Thom spectrum of the canonical bundle over $BU(n)$.  The $\mathbb{E}_2$-ring structure arising from the Thom spectrum construction applied to
$$\coprod BU(n) \stackrel{J}{\longrightarrow} Pic(\mathbb{S})$$
does not agree with the $\mathbb{E}_2$-ring structure on $\Sigma^{\infty}_+ BU$ that arises from the double loop space structure on $BU$.
\textbf{Allen are we sure these don't agree?}
\end{rmk}

The final result above, regarding $MU$-module spectra, can be seen as a once-looped analogue of work of Kitchloo \cite{Kitchloo}.   Kitchloo studied a splitting, due to Miller \cite{MillerSplitting}, of $\Sigma^{\infty}_+ SU(n)$.  His theorem is that, \textit{for complex-oriented $E$}, the corresponding direct sum decomposition of $E_*(SU(n))$ is multiplicative.

Our proof of Theorem \ref{thm:MainMUE2} is by obstruction theory.  We show in Section \ref{sec:MUE2} that all obstructions to an $\mathbb{E}_2$-equivalence vanish.  On the other hand, we prove Theorem \ref{thm:MainObstruction} by explicitly calculating a non-zero obstruction in Section \ref{sec:Obstruction}.

It remains to discuss Theorem \ref{thm:MainAoo}, the $\mathbb{A}_\infty$ splitting.

\textbf{SOMETHING ABOUT STIEFEL MANIFOLDS}

Open questions regarding natural extensions of our work:

\textbf{What is the structure of the equivariant splitting?}

\textbf{What is the proper motivic analogue of our result?}



%%Might be worth making sure that at least in the statements of the theorems/results, we use "infty-category" instead of "category

%acknowledge
%1. Jacob, Mike
%2. arone? akhil, denis?, dyang?, justin? 

%Notations, todo?
%\Sp for spectra, S for spaces? should also say that by default, these are given \smash, and \times.


\section{Preliminaries}

\input{Preliminaries.tex}

\section{Obstructions} \label{sec:PowOp}

\input{PowerOperation.tex}

\section{Periodic orientations of $KU$}\label{sec:MUPKU}

\input{MUPKU.tex}

\bibliographystyle{alpha}
\bibliography{Bibliography}

\end{document}