In this section we recall the Mitchell--Segal Bott filtration \cite{MitchellLoopGroup} on $\Omega SU(n)$.  We prove that the Bott filtration is at least $\mathbb{A}_\infty$, meaning in particular that its suspension is an $\mathbb{A}_\infty$-filtered spectrum in the sense of Section \ref{sec:FilGra}.

It is most efficient to describe the filtration in the language of algebraic geometry, and in particular we will need to recall the theory of affine and Beilinson--Drinfeld Grassmannians.  A good general reference is \cite{Zhu}.  We use $D$ to denote the formal disk $\text{Spec}(\mathbb{C}[[t]])$ and $D^*$ to denote the punctured disk $\text{Spec}(\mathbb{C}((t)))$.  For $R$ a $\mathbb{C}$-algebra, we use $D_R$ to denote $\text{Spec}(\mathbb{C}[[t]] \hat{\otimes} R)$ and $D^*_R$ to denote $\text{Spec}(\mathbb{C}((t)) \hat{\otimes} R)$.

\begin{dfn}
Let $G$ denote a smooth affine algebraic group over $\mathbb{C}$ (we will be interested only in the cases $G=SL_n,GL_n$).  The \textit{affine Grassmannian} $Gr_G$ of $G$ is the Ind-scheme with functor of points
$$Gr_G(R) = \{(\mathcal{E},\beta)\},\text{ where}$$
$\mathcal{E}$ is a $G$-torsor over $D_{R}$ and $\beta:\mathcal{E}|_{D^*_{R}} \cong \mathcal{E}^0_{D^*_{R}}$ is a trivialization over $D^*_{R}$.
\end{dfn}

The complex points $Gr_G(\mathbb{C})$ are a model for the topological space $\Omega G$.  The idea is that $\Hom(D^*,G)$ is the space of algebraic free (i.e., unbased) loops in $G$.  One thinks of the complex points of $Gr_G$ as the homogeneous space
$$G(\mathbb{C}((t)))/G(\mathbb{C}[[t]]),$$
which up to homotopy is the quotient of the free loop space on $G$ by the action of $G$.  

We use $\mathbb{X}^{\bullet}$ to denote the lattice of weights $\Hom(G,\mathbb{G}_m)$, and $\mathbb{X}_{\bullet}$ to denote the dual lattice of coweights.  Inside $\mathbb{X}^{\bullet}$ is the set $\Phi$ of roots.  We fix a particular Borel subgroup $B \subset G$, determining a choice of positive roots $\Phi^+ \subset \Phi$ and a semi-group of dominant coweights $\mathbb{X}^+_\bullet \subset \mathbb{X}_\bullet$.  There is a natural bijection
$$\mathbb{X}_{\bullet}^+ \cong G(\mathbb{C}[[t]])\backslash G(\mathbb{C}((t)))/G(\mathbb{C}[[t]])$$
of dominant coweights with the above double cosets.  Each coweight $\mu \in \Hom(\mathbb{G}_m,G)$ may be thought of as a specific loop $t^{\mu}$ in the free loop space of $G$, and hence under projection as a point in $\Omega G$.

There is a double-coset decomposition of the free loop space
$$
\coprod_{\mu \in \mathbb{X}_{\bullet}^+} G(\mathbb{C}[[t]]) t^{\mu} G(\mathbb{C}[[t]]).
$$
Projecting onto the affine Grassmannian, one learns that the $G(\mathbb{C}[[t]])$-orbits of $Gr_G$ are indexed by $\mu \in \mathbb{X}_{\bullet}^+$.  We will use $Gr_{G,\le \mu}$ to denote the \textit{closure} of the orbit corresponding to $\mu$.  The closure $Gr_{G, \le \mu_1}$ contains $Gr_{G, \le \mu_2}$ if and only if $\mu_1-\mu_2$ is a sum of dominant coroots.  We call $\{Gr_{G,\le \mu}|\mu \in \mathbb{X}^+_{\bullet}\}$ the \textit{Schubert filtration} of $Gr_G$.

\begin{exm} \label{sl2example}
Suppose $G=SL_2(\mathbb{C})$.  Then a coweight $\mu \in \mathbb{X}_\bullet$ consists of a pair $(a,b)$ of integers with $a+b=0$.  We choose a Borel so that a coweight is dominant if $a \ge b$.  The conjugation action of $SL_2(\mathbb{C})$ on $\Omega SL_2(\mathbb{C})$ has one orbit for each pair $(a,-a)$ with $a \ge 0$.  The orbit corresponding to $(a,-a)$ contains the loop $\mathbb{G}_m \rightarrow \Omega SL_2(\mathbb{C})$ given by
$$
t \mapsto \left( \begin{array}{cc} t^a & 0 \\ 0 & t^{-a}  \end{array} \right).
$$
The closure of the $(a,-a)$ orbit contains the $(b,-b)$ orbit if and only if $b \le a$.  To topologists, $\Omega SL_2(\mathbb{C}) \simeq \Omega \Sigma S^2$ is recognizable as the free $\mathbb{A}_\infty$-algebra on the pointed space $S^2$.  In particular, $Gr_{SL_2}(\mathbb{C})$ is naturally equipped with the James filtration by word length.  The closure of the $(a,-a)$ orbit turns out to be the $(2a)$th component of the James filtration, so that the Schubert filtration is strictly coarser than the James filtration.  In other words, the $S^2$ that appears as the first James filtered piece of $\Omega SL_2(\mathbb{C})$ is not closed under the $SL_2(\mathbb{C})$ conjugation action.  Only the collection of words of length $2$ or less is closed under the $SL_2(\mathbb{C})$ action. 
\end{exm}

The $\mathbb{E}_2$-algebra structure on $\Omega G$ is elegantly encoded in algebraic geometry through the notion of Beilinson--Drinfeld Grassmannian:

\begin{dfn}
The \textit{Ran space} $\text{Ran}(\mathbb{A}^1)$ is the presheaf that assigns to every $\mathbb{C}$-algebra $R$ the set of non-empty finite subsets of $\text{Spec}(R) \times \mathbb{A}^1$.   The Beilinson--Drinfeld Grassmannian is the presheaf $Gr_{G,Ran}$ that assigns to each $\mathbb{C}$-algebra $R$ the set of triplets $(x,\mathcal{E},\beta)$, where $x \in \text{Ran}(\mathbb{A}^1)(R)$, $\mathcal{E}$ is a $G$-torsor on $\mathbb{A}^1 \times \text{Spec}(R)$, and $\beta$ is a trivialization of $\mathcal{E}$ away from the graph of $x$ in $\mathbb{A}^1 \times \text{Spec}(R)$.
\end{dfn}

One thinks of the Beilinson--Drinfeld Grassmannian as fibered over the Ran space.  In other words, for every collection of points $I \subset \mathbb{A}^1$, there is a corresponding point $x$ in the Ran space.  The fiber of the Beilinson--Drinfeld Grassmannian over $x$ is the moduli of $G$-bundles on $\mathbb{A}^1$ equipped with a trivialization away from the points in $I$.  This fiber is naturally isomorphic to the product of $|I|$ copies of $Gr_G$.  The multiplication on $Gr_G$ is encoded by degeneration of fibers as points collide in the Ran space.  For more details, see \cite[\S 3]{Zhu}.

The connection of the above structure with the notion of $\mathbb{E}_2$-algebra in homotopy theory was spelled out explicitly by Jacob Lurie in \cite[\S 5.5]{HA}.  In the language of Lurie's work, the complex points of the Beilinson--Drinfeld Grassmannian form a factorizable cosheaf, valued in spaces, on $Ran(\mathbb{C})$.  Lurie proves \cite[Theorem 5.5.4.10]{HA} that this is enough to equip the complex points of $Gr_G$ (namely $\Omega G$) with the structure of a non-unital $\mathbb{E}_2$ algebra.  This in turn makes $\Sigma^{\infty}_+ \Omega G$ into a unital (in fact augmented) $\mathbb{E}_2$-ring spectrum.

It is through the Beilinson--Drinfeld perspective that we can most easily see the interaction of the Schubert filtration on $Gr_G$ with its $\mathbb{E}_2$-algebra structure.  The key point is the fact (see, e.g., \cite[3.1.14]{Zhu}) that, as points collide in the Beilinson--Drinfeld Grassmannian, the fiber $Gr_{G,\le \mu_1} \times Gr_{G,\le \mu_2}$ degenerates to $Gr_{G, \le \mu_1+\mu_2}$.

That is already enough to prove that, for example, the Schubert filtration on $\Sigma^{\infty}_+ \Omega SU(2)$ described in Example \ref{sl2example} is an $\mathbb{E}_2$-filtered spectrum in the sense of Section \ref{sec:FilGra}.  What we will actually want to be $\mathbb{E}_2$, or at least $\mathbb{A}_\infty$, is the James filtration on $\Sigma^{\infty}_+ \Omega SU(2)$.  In general, it turns out that the Schubert filtration on the Beilinson--Drinfeld Grassmannian for $SL_n(\mathbb{C})$ provides only direct access to every $n$th piece of the Bott filtration on $\Sigma^{\infty}_+ \Omega SL_n(\mathbb{C})$.  We will follow Segal \cite{Segal} and access the Bott filtration on $Gr_{SL_n}$ in a somewhat indirect manner, by considering not $Gr_{SL_n}$ but $Gr_{GL_n}$:

\begin{dfn}
Consider the affine Grassmannian $Gr_{GL_n}$.  We denote by $F_{n,k}$ the subset of $Gr_{GL_n}$ that is the closure of the $GL_n(\mathbb{C}[[t]])$ orbit containing:
$$t \mapsto \left( \begin{array}{cccc} t^k & 0 & \cdots & 0 \\ 0 & 1 & \cdots & 0 \\ \vdots & \vdots & \ddots & \vdots \\ 0 & 0 & \cdots & 1 \end{array} \right).$$
In other words, $F_{n,k} = Gr_{GL_n, \le (k,0,\cdots,0)}$.
\end{dfn}

We draw the following lemma, affirming a conjecture of Mahowald and Richter, as an immediate corollary of the abstract machinary of Beilinson--Drinfeld \cite[3.1.14]{Zhu} and \cite[5.5.4.10]{HA}:

\begin{lem} [Conjecture of Mahowald--Richter \cite{MahowaldRichter}] 
The inclusion 
$$\coprod_k F_{n,k} \subset \Omega GL_n(\mathbb{C})$$
may be made into a map of non-unital $\mathbb{E}_2$-algebras.
\end{lem}