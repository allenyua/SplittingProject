
We remark in this section that the $\mathbb{A}_\infty$ splitting $$\Sigma^{\infty}_+ \Omega SU(n) \simeq gr(\Sigma^{\infty}_+ \{F_{n,k}\})$$ becomes $\mathbb{E}_2$ after base-change to complex bordism.  More precisely, suppose that $gr(\Sigma^{\infty}_+ \{F_{n,k}\})$ is equipped with some graded $\mathbb{E}_2$-ring structure extending the natural graded $\mathbb{A}_\infty$-ring structure.  Construction \ref{cnstr:E2GrConstruction} provides one possible way to do this.  We will by abuse of notation use $gr(\Sigma^{\infty}_+ \{F_{n,k}\})$ also to denote the underlying \textit{ungraded} $\mathbb{E}_2$-algebra, and our main theorem is that there is an equivalence of (ungraded) $\mathbb{E}_2$-$MU$-algebras:

$$MU \smsh \Sigma^{\infty}_+ \Omega SU(n) \simeq MU \smsh gr(\Sigma^{\infty}_+ \{F_{n,k}\}).$$

Notice that the results of Section \ref{sec:AooSplit} give an $\mathbb{A}_\infty$-equivalence between these two $MU$-algebras.  This $\mathbb{A}_\infty$-$MU$-algebra equivalence is adjoint to a map of $\mathbb{A}_\infty$-$\mathbb{S}$-algebras
\begin{equation} \label{SplittingMap}
\Sigma_+^{\infty} \Omega SU(n) \longrightarrow MU \smsh gr(\Sigma^{\infty}_+ \{F_{n,k}\}).
\end{equation}

Our task in this section will be to show that (\ref{SplittingMap}) may be refined to a morphism of $\mathbb{E}_2$-ring spectra.  We do so by obstruction theory--the key fact powering our proof is that 
$$MU_{2*+1}\left(\Omega SU(n)\right) \cong 0.$$
This classical vanishing result may be proven via Atiyah--Hirzerburch spectral sequence, using the even cell-decomposition of $Gr_{SL_n}(\mathbb{C})$.

Inspired by \cite{ChadwickMandell}, we prove the following general result (implying in particular Theorem \ref{thm:MainMUE2}):

\begin{thm}
Suppose that $R$ is an $\mathbb{E}_2$-ring spectrum with no homotopy groups in odd degrees.  Then any homotopy commutative ring homomorphism
$$\Sigma^{\infty}_+ \Omega SU(n) \rightarrow R$$
lifts to a morphism of $\mathbb{E}_2$-ring spectra.  Moreover, any chosen $\mathbb{A}_\infty$ lift may be extended to an $\mathbb{E}_2$ lift.
\end{thm}

\begin{proof} 
By taking connective covers, one learns that any ring homomorphism
$$\Sigma^{\infty}_+ \Omega SU(n) \rightarrow R$$
must factor through the natural $\mathbb{E}_2$-algebra map $\tau_{\ge 0} R \rightarrow R$.  Thus, without loss of generality we will assume that $R$ is $(-1)$-connected.

It is clear that the composite ring homomorphism
$$\Sigma^{\infty}_+ \Omega SU(n) \longrightarrow R \longrightarrow \tau_{\le 0} R \simeq H\pi_0(R)$$
may be lifted to an $\mathbb{E}_2$-ring homomorphism factoring through $\tau_{\le 0} \Sigma^{\infty}_+ \Omega SU(n) \simeq H\mathbb{Z}$.   Suppose now for $q>0$ that we have chosen an $\mathbb{E}_2$-ring homomorphism 
$$\Sigma^{\infty}_+ \Omega SU(n) \longrightarrow \tau_{\le q-1} R$$
We will show that there is no obstruction to the existence of a further $\mathbb{E}_2$-lift $$\Sigma^{\infty}_+ \Omega SU(n) \longrightarrow \tau_{\le q} R,$$
and that one may be chosen lifting any specified $\mathbb{A}_\infty$ map $\Sigma^{\infty}_+ \Omega SU(n) \rightarrow \tau_{\le q} R$.

According to \cite[Theorem $4.1$]{ChadwickMandell}, there is a diagram of principal fibrations
$$
\begin{tikzcd}
\mathbb{E}_2\text{-Ring}(\Sigma^{\infty}_+ \Omega SU(n), \tau_{\le q} R) \arrow{r} \arrow{d} & \mathbb{A}_\infty\text{-Ring}(\Sigma^{\infty}_+ \Omega SU(n), \tau_{\le q} R) \arrow{d} \\
\mathbb{E}_2\text{-Ring}(\Sigma^{\infty}_+ \Omega SU(n), \tau_{\le q-1} R) \arrow{r} \arrow{d} & \mathbb{A}_\infty\text{-Ring}(\Sigma^{\infty}_+ \Omega SU(n), \tau_{\le q-1} R) \arrow{d} \\
\cS_*(BSU(n),K(\pi_q R,q+3)) \arrow{r} & \cS_*(SU(n),K(\pi_q R,q+2))
\end{tikzcd}
$$
For $q$ odd, $\tau_{\le q-1} R \simeq \tau_{\le q} R$, so there is no obstruction.  Let us therefore assume that $q$ is even.

Since the cohomology of $BSU(n)$ is even-concentrated with coefficients in any abelian group, we have that $\pi_0 \cS_*(BSU(n),K(\pi_q R,q+3)) \cong H^{q+3}(BSU(n);\pi_q R)$ is zero.  It follows then that the given class $$x \in \pi_0 \mathbb{E}_2\text{-Ring}(\Sigma^{\infty}_+ \Omega SU(n), \tau_{\le q-1} R)$$ admits some lift $$\widetilde{x} \in \mathbb{E}_2\text{-Ring}(\Sigma^{\infty}_+ \Omega SU(n), \tau_{\le q} R).$$  We may need to modify $\widetilde{x}$ to match our chosen $\mathbb{A}_\infty$-ring homomorphism.  This is always possible so long as the map
$$\pi_1(\cS_*(BSU(n),K(\pi_q R,q+3))) \longrightarrow \pi_1(\cS_*(SU(n),K(\pi_q R,q+2)))$$
is surjective.  Said in other terms, this is just the map
$$H^{2q+2}(BSU(n);\pi_q R) \longrightarrow H^{2q+1}(SU(n);\pi_q R) \cong H^{2q+2}(\Sigma SU(n);\pi_q R)$$
induced by the natural map $\Sigma SU(n) \rightarrow BSU(n)$.  It is a classical fact that this map is surjective (it follows from a calculation with the bar spectral sequence, using the fact that the cohomology of $SU(n)$ is exterior).
\end{proof}