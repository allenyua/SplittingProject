\documentclass[reqno, oneside]{amsart}
\usepackage{hyperref}

%\usepackage[export]{adjustbox}
%\usepackage[dvips]{pict2e}
%\usepackage{amsmath,amsthm}
\usepackage{comment}
\usepackage{amsfonts, fullpage, fancyhdr, mathtools, qtree, amsmath, tipa, amssymb, hyperref, url, amsthm, subfigure, xy, tikz-cd, verbatim}

\usepackage[color=cyan!40]{todonotes}



\usepackage[a4paper]{geometry}


%----------------------------------------------------------------------%

%\renewcommand{\appendixname}{Example}
%\swapnumbers

\theoremstyle{definition}
\newtheorem{nul}{}[section]
\newtheorem{dfn}[nul]{Definition}
\newtheorem{axm}[nul]{Axiom}
\newtheorem{rmk}[nul]{Remark}
\newtheorem{term}[nul]{Terminology}
\newtheorem{cnstr}[nul]{Construction}
\newtheorem{ntn}[nul]{Notation}
\newtheorem{exm}[nul]{Example}
\newtheorem{obs}[nul]{Observation}
\newtheorem{ctrexm}[nul]{Counterexample}
\newtheorem{rec}[nul]{Recollection}
\newtheorem{exr}[nul]{Exercise}
\newtheorem{wrn}[nul]{Warning}
\newtheorem*{dfn*}{Definition}
\newtheorem*{axm*}{Axiom}
\newtheorem*{ntn*}{Notation}
\newtheorem*{exm*}{Example}
\newtheorem*{exr*}{Exercise}
\newtheorem*{int*}{Intuition}
\newtheorem*{qst*}{Question}


\theoremstyle{plain}
\newtheorem{sch}[nul]{Scholium}
\newtheorem{thm}[nul]{Theorem}
\newtheorem{prop}[nul]{Proposition}
\newtheorem{lem}[nul]{Lemma}
\newtheorem{sublem}{Lemma}[nul]
\newtheorem{por}[nul]{Porism}
\newtheorem{cnj}[nul]{Conjecture}
\newtheorem{cor}{Corollary}[nul]
\newtheorem*{thm*}{Theorem}
\newtheorem*{prop*}{Proposition}
\newtheorem*{cor*}{Corollary}
\newtheorem*{lem*}{Lemma}
\newtheorem*{cnj*}{Conjecture}


%----------------------------------------------------------------------%

\DeclareMathOperator{\Aut}{\text{Aut}}
\DeclareMathOperator{\Tr}{\text{Tr}}
\DeclareMathOperator{\Res}{\text{Res}}
\DeclareMathOperator{\im}{\text{im}}
\DeclareMathOperator*{\colim}{\text{colim}}
\DeclareMathOperator{\Map}{\text{Map}}
\DeclareMathOperator{\cofiber}{\text{cofiber}}
\DeclareMathOperator{\fiber}{\text{fiber}}
\DeclareMathOperator{\Hom}{\text{Hom}}
\DeclareMathOperator{\Skel}{\text{Skel}}
\DeclareMathOperator*{\hocolim}{\text{hocolim}}
\DeclareMathOperator*{\holim}{\text{holim}}
\DeclareMathOperator{\smsh}{\wedge}



\DeclareMathOperator{\C}{\mathcal{C}}
\DeclareMathOperator{\D}{\mathcal{D}}
\DeclareMathOperator{\CP}{\mathbb{CP}}
\DeclareMathOperator{\Z}{\mathbb{Z}}
\DeclareMathOperator{\E}{\mathbb{E}}
\DeclareMathOperator{\Q}{\mathbb{Q}}
\DeclareMathOperator{\m}{\mathfrak{m}}
\DeclareMathOperator{\G}{\mathbb{G}}
\DeclareMathOperator{\F}{\mathbb{F}}
\DeclareMathOperator{\cG}{\mathcal{G}}
\DeclareMathOperator{\cF}{\mathcal{F}}
\DeclareMathOperator{\cS}{\mathcal{S}}
\DeclareMathOperator{\Ring}{\textbf{Ring}}
\DeclareMathOperator{\Aff}{\textbf{Aff}}
\DeclareMathOperator{\Spec}{\text{Spec}}
\DeclareMathOperator{\Poly}{\text{Poly}}
\DeclareMathOperator{\Ext}{\text{Ext}}
\DeclareMathOperator{\nil}{\text{nil}}
\DeclareMathOperator{\fin}{\text{fin}}
\DeclareMathOperator{\Gr}{\textbf{Gr}}
\DeclareMathOperator{\Fil}{\textbf{Fil}}
\DeclareMathOperator{\Fun}{\text{Fun}}
\DeclareMathOperator{\Alg}{\text{Alg}}
\DeclareMathOperator{\Sp}{\text{Sp}}
\DeclareMathOperator{\Spaces}{\mathcal{S}}
\DeclareMathOperator{\S0}{\mathbb{S}}
\DeclareMathOperator{\J}{\mathcal{J}}
\DeclareMathOperator{\Cofil}{\textbf{Cofil}}

%----------------------------------------------------------------------%

\usepackage{tikz}
\usetikzlibrary{matrix,arrows,decorations}
\usepackage{tikz-cd}

\usepackage{adjustbox}

%----------------------------------------------------------------------%

\hyphenation{Mack-ey mon-oid-al Wald-hau-sen}

%----------------------------------------------------------------------%
%----------------------------------------------------------------------%


\begin{document}

\title{Structured Splittings of $\Omega SU(n)$ and Snaith's Construction of Periodic Complex Bordism}
\author{Jeremy Hahn and Allen Yuan}

\begin{abstract}
We show that the ??? filtration on the affine Grassmannian of $SL_n(\mathbb{C})$, known to topologists as the Bott filtration on $\Omega SU(n)$, stably splits as an $\mathbb{A}_\infty$ but not as an $\mathbb{E}_2$ filtration.  We further prove that the splitting becomes $\mathbb{E}_2$ after smashing with any complex-oriented homology theory.  As a limiting case, we study the coherence of Snaith's construction of periodic complex cobordism.  We determine that it is a construction of $MUP$ as an $\mathbb{E}_2$-ring spectrum, but not as an $\mathbb{E}_3$-ring spectrum.
\end{abstract}



%----------------------------------------------------------------------%
%----------------------------------------------------------------------%


\setcounter{tocdepth}{1}
\maketitle

\tableofcontents


%----------------------------------------------------------------------%

\section{Introduction}

The complex cobordism spectrum $MU$ has played a central role in homotopy theory ever since Quillen \cite{Quillen} connected its homotopy with the theory of one-dimensional formal group laws.  As the Thom spectrum of the canonical bundle over $BU$, $MU$ naturally acquires an immense amount of structure: it is an $\mathbb{E}_\infty$-ring spectrum.  To this day, however, much remains unknown about the full nature of this $\mathbb{E}_\infty$ structure.  For example, almost nothing is known about the k-invariants of the spectrum of units $gl_1(MU)$ \cite{LawsonBP}.

As a spectrum, $MU$ is classically constructed as the direct limit of the sequence
$$\mathbb{S}=MU(0) \longrightarrow \Sigma^{-2} MU(1) \longrightarrow \cdots \longrightarrow \Sigma^{-2n} MU(n) \longrightarrow \cdots,$$
where $MU(n)$ is the Thom spectrum of the canonical bundle over $BU(n)$.

This suggests that perhaps the more elemental object is $$\bigvee MU(n),$$ the Thom spectrum of the $J$-homomorphism
$$\textbf{Vect} \stackrel{J}{\longrightarrow} \text{Pic}(\mathbb{S})$$
that takes a vector space $V$ to its one-point compactification $J(V)=S^V$.  Since $$J(V \oplus W) \simeq S^V \smsh S^W,$$ the Thom spectrum $\bigvee MU(n)$
is naturally an $\mathbb{E}_\infty$-ring spectrum.  Inverting the Bott element $\beta \in \pi_2(MU(1)) \cong \pi_2(\mathbb{CP}^{\infty})$, one obtains the \textit{periodic} complex cobordism spectrum
$$MUP \simeq \left(\bigvee MU(n) \right)[\beta^{-1}].$$

This periodic spectrum $MUP$ is a minor variation on $MU$ itself--there is a wedge decomposition $$MUP \simeq \bigvee_{a \in \mathbb{Z}} \Sigma^{2a} MU,$$ and the inclusion $MU \rightarrow MUP$ onto the $a=0$ factor is an an $\mathbb{E}_\infty$-ring homomorphism.

In 1979, Victor Snaith \cite{SnaithBook} gave a second, and fundamentally different, presentation of periodic complex cobordism:

\begin{thm}[Snaith] \label{SnaithSplitting}
As homotopy commutative ring spectra, $$MUP \simeq \Sigma^{\infty}_+ BU[\beta^{-1}].$$  More generally, there is an equivalence of homotopy commutative rings
$$\bigvee MU(n) \simeq \Sigma^{\infty}_+ BU.$$
\end{thm}

\begin{rmk} In the above, $\Sigma^{\infty}_+ BU$ acquires an $\mathbb{E}_\infty$-structure (and hence a homotopy commutative ring structure) from the fact that $BU$ is an infinite loop space.  One may think of $\Sigma^{\infty}_+ BU$ as the group ring of the topological group $BU$.
\end{rmk}

The genesis of this paper was an attempt to use Snaith's theorem to study the $\mathbb{E}_\infty$-ring structure on $MUP$.  This seemed like an especially reasonable idea in light of two facts:

\begin{enumerate}
\item By another theorem of Snaith, periodic $K$-theory $KU$ may be constructed as $$KU \simeq \Sigma^{\infty}_+ \mathbb{CP}^{\infty}[\beta^{-1}].$$  This is an equivalence of $\mathbb{E}_\infty$-ring spectra \textbf{CITE}.
\item In \cite{GepnerSnaith}, Gepner and Snaith prove a motivic analogue of Theorem \ref{SnaithSplitting}.  In particular, they prove an equivalence 
$$\Sigma^{\infty}_+ BGL[\beta^{-1}] \simeq PMGL.$$
They then use the natural $\mathbb{E}_\infty$-ring structure on $\Sigma^{\infty}_+ BGL[\beta^{-1}]$ to \textbf{define} an $\mathbb{E}_\infty$-ring structure on $PMGL$. 
\end{enumerate}

\textbf{Allen, there is something I am confused about.  In Gepner-Snaith, they claim that there is an $\mathbb{E}_\infty$ ring homomorphism from $PMGL$ to $KU$ given by inverting $\beta$ in the suspension of the determinant map
$$BU \rightarrow \mathbb{CP}^{\infty}.$$
I thought though that we decided there is no $\mathbb{E}_\infty$-ring map to $H\mathbb{Z}P$.  What's up with that?}

However, as it turns out (though it is not at all obvious from the modern literature and in particular not mentioned in \cite{GepnerSnaith}), another old theorem of Snaith \cite{SnaithNotMultiplicative} shows that our idea was entirely unreasonable:

\begin{thm}[Snaith]
As $\mathbb{E}_\infty$-rings,
$$MUP \not \simeq \Sigma^{\infty}_+ BU[\beta^{-1}].$$
\end{thm}

In the final Section \ref{sec:SnaithSplitting} of this paper, we refine Snaith's results in to what we consider their definitive form:

\begin{thm}
There is an equivalence of $\mathbb{E}_2$-ring spectra
$$MUP \simeq \Sigma^{\infty}_+ BU[\beta^{-1}],$$
but $MUP \not \simeq \Sigma^{\infty}_+ BU[\beta^{-1}]$ as $\mathbb{E}_3$-ring spectra.  There is an equivalence of $\mathbb{A}_\infty$-ring spectra
$$\bigvee MU(n) \simeq \Sigma^{\infty}_+ BU,$$
but $\bigvee MU(n) \not \simeq \Sigma^{\infty}_+ BU$ as $\mathbb{E}_2$-ring spectra.
\end{thm}

With the focus now on $\mathbb{E}_2$-algebras, it is natural to view Snaith's splitting result as the limiting case of a sequence of other stable splittings.  Consider the filtration
$$* \simeq \Omega SU(1) \longrightarrow \Omega SU(2) \longrightarrow \cdots \longrightarrow \Omega SU(n) \longrightarrow \cdots \longrightarrow \Omega SU \simeq BU,$$
where the last equivalence is by Bott periodicity.  Taking suspension spectra, we obtain a filtration of $\mathbb{E}_2$-ring spectra
$$\mathbb{S} \longrightarrow \Sigma^{\infty}_+ \Omega SU(2) \longrightarrow \cdots \longrightarrow \Sigma^{\infty}_+ \Omega SU(n) \longrightarrow \cdots \longrightarrow \Sigma^{\infty}_+ BU.$$

It is a theorem of Crabb and Mitchell \cite{CrabbMitchell} that, for $n>1$, $\Sigma^{\infty}_+ \Omega SU(n)$ splits as an infinite wedge sum.  We study the coherence of their splitting in Sections \ref{sec:AooSplit} and \textbf{BLAH}:

\begin{thm}
The Crabb--Mitchell stable splitting of $\Sigma^{\infty}_+ \Omega SU(n)$ is a splitting of $\mathbb{A}_\infty$-ring spectra, but not of $\mathbb{E}_2$-ring spectra.
\end{thm}

To be precise about what we mean by a splitting of $\mathbb{E}_n$-ring spectra, we need to introduce a bit of abstract terminology.

\begin{rmk}
We freely use the language of $\infty$-categories throughout this paper, referring to an $\infty$-category simply as a category.
\end{rmk}

%I stole a bunch of the stuff written here, so maybe it can be omitted or simplified.  
In Section \ref{sec:FilGra} we will review the symmetric monoidal categories $\Fil$ and $\Gr$ of filtered and graded spectra, respectively.  A filtered spectrum is an infinite sequence
$$X_0 \longrightarrow X_1 \longrightarrow X_2 \longrightarrow X_3 \longrightarrow \cdots$$
of spectra.  The tensor product $$\left(X_0 \longrightarrow X_1 \longrightarrow X_2 \longrightarrow \cdots \right) \otimes \left(Y_0 \longrightarrow Y_1 \longrightarrow Y_2 \longrightarrow \cdots \right)$$
of two filtered spectra is computed as a Day convolution

\begin{center}
$X_0 \otimes Y_0 \longrightarrow \colim $
\adjustbox{scale=0.7} 
{$ \left(\begin{tikzcd} X_0 \smsh Y_1 \\  X_0 \smsh Y_0 \arrow{u} \arrow{r} & X_1 \smsh Y_0 \end{tikzcd} \right) $} 
$\longrightarrow \colim$
\adjustbox{scale=0.7} {$ \left( \begin{tikzcd} X_0 \smsh Y_2 \\ X_0 \smsh Y_1 \arrow{r} \arrow{u} & X_1 \smsh Y_1  \\ X_0 \smsh Y_0 \arrow{r} \arrow{u} & X_1 \smsh Y_0 \arrow{u} \arrow{r} & X_2 \smsh Y_0 \end{tikzcd} \right) $}
$\longrightarrow \cdots.$
\end{center}

A graded spectrum, on the other hand, is simply an ordered sequence $(A_0,A_1,A_2, \cdots)$ of spectra.  The tensor product  is computed as
$$(A_0,A_1,A_2,\cdots) \otimes (B_0,B_1,B_2,\cdots) \simeq \left( A_0 \smsh B_0, (A_1 \smsh B_0) \vee (A_0 \smsh B_1), \cdots, \bigvee_{i+j=n} A_i \smsh B_j, \cdots \right).$$

There is a sequence of symmetric monoidal functors
$$
\begin{tikzcd}[column sep=huge]
\Gr \arrow{r}{I} & \Fil \arrow{r}{\text{colim}} & \Sp,
\end{tikzcd}
$$
where $I$ sends the graded spectrum $(A_0,A_1,A_2,\cdots)$ to the filtered spectrum
$$
I(A_0,A_1,A_2,\cdots) = \left( A_0 \longrightarrow A_0 \vee A_1 \longrightarrow A_0 \vee A_1 \vee A_2 \longrightarrow \cdots\right).
$$

\begin{dfn}
We say that an $\mathbb{E}_n$-ring spectrum is $\mathbb{E}_n$-\textbf{split} if it is equivalent to the image of an $\mathbb{E}_n$-algebra in $\Gr$ under the composite $\text{colim} \circ I.$  Similarly, a filtered $\mathbb{E}_n$-algebra is $\mathbb{E}_n$-split if it is equivalent to the image under $I$ of an $\mathbb{E}_n$-algebra in $\Gr$.
\end{dfn}

\textbf{BLAH}

In section \ref{sec:MUE2} we will show:
\begin{thm}
The ??? filtration on $\Sigma^{\infty}_+ \Omega SU(n)$ is $\mathbb{E}_2$-split after smashing with $MU$.
\end{thm}

\textbf{BLAH}
%acknowledge
%1. Jacob, Mike
%2. arone? akhil, denis?, dyang?, 
%3. Arpon if he reads it,

%Notations, todo?
%\Sp for spectra, S for spaces? should also say that by default, these are given \smash, and \times.

\section{Filtered and Graded Ring Spectra} \label{sec:FilGra}

It will be important for us to have a precise language for discussing filtered and graded spectra, what it means to be split, what it means to take associated graded, and the multiplicative aspects of these constructions. Here we review a framework from \cite{LurieRot} for studying graded and filtered objects.  The reader is referred to \cite{LurieRot} for a more thorough treatment and all proofs.  

\subsection{Definitions}

%Let $\C$ be a symmetric monoidal stable $\infty$-category which admits filtered colimits.  
Let $\D$ be an $\infty$-category which we will regard as the diagram category.  Our filtered objects will be valued in the functor category $\Sp^{\D}.$  This will be no more difficult than just ordinary spectra because limits, colimits, and smash products will be considered pointwise.  

Denote by $\Z_{\geq 0}$ the poset of non-negative integers, denoted $[n]$, thought of as an ordinary category where $\Hom([a],[b])$ is a singleton if $a\leq b$, and empty otherwise.  Denote by $\Z_{\geq 0}^{ds}$ the corresponding discrete category.  We may then take nerves to obtain $\infty$-categories $N(\Z_{\geq 0})$ and $N(\Z_{\geq 0}^{ds})$, which will serve as the indexing sets for filtered and graded spectra.  The reader is warned that our numbering conventions are opposite the ones in \cite{LurieRot}.

\begin{dfn} 
Let $\Gr(\Sp^{\D})$ denote the functor category $\Fun(\Z_{\geq 0}^{ds}, \Sp^{\D}).$  We shall refer to $\Gr(\Sp^{\D})$ as the category of graded objects in $\Sp^{\D}$.  Its objects can be thought of as sequences $X_0, X_1,X_2,\cdots \in \Sp^{\D}$.
\end{dfn}

\begin{dfn} 
Let $\Fil(\Sp^{\D})$ denote the functor category $\Fun(\Z_{\geq 0}, \Sp^{\D}).$  We shall refer to $\Fil(\Sp^{\D})$ as the category of filtered objects in $\Sp^{\D}$.  Its objects can be thought of as sequences $Y_0\to Y_1\to Y_2 \to \cdots \in \Sp^{\D}$ filtering $\colim_i Y_i$.  
\end{dfn}


The obvious map $N(\Z_{\geq 0}^{ds}) \to N(\Z_{\geq 0})$ induces a restriction functor $\text{res}: \Fil(\Sp^{\D}) \to \Gr(\Sp^{\D})$ which can be thought of as forgetting the maps in the filtered object.  The restriction fits into an adjunction  
$$I:\Gr(\Sp^{\D}) \xrightleftharpoons{\quad} \Fil(\Sp^{\D}) : \text{res}$$

where the left adjoint $I: \Gr(\Sp^{\D}) \to \Fil(\Sp^{\D})$ is given by left Kan extension.  The functor $I$ can be described explicitly as taking a graded object $X_0,X_1,X_2,\cdots$ to the filtered object $$I(X_0, X_1, \cdots) = (X_0\to X_0\oplus X_1\to X_0 \oplus X_1\oplus X_2\to \cdots).$$   


\subsection{Monoidal structures}\label{sect:monoidal}
By \cite[Corollary 2.3.9]{LurieRot}, the categories $\Gr(\Sp)$ and $\Fil(\Sp)$ may be given symmetric monoidal structures via the Day convolution.  Then, via the identifications $\Gr(\Sp^{\D}) = \Gr(\Sp)^{\D}$ and $\Fil(\Sp^{\D}) = \Fil(\Sp)^{\D}$, the categories $\Gr(\Sp^{\D})$ and $\Fil(\Sp^{\D})$ may be given symmetric monoidal structures pointwise on $\D$.  In both cases, we denote the resulting operation by $\otimes$.  Explicitly, the filtered tensor product $$\left(X_0 \longrightarrow X_1 \longrightarrow X_2 \longrightarrow \cdots \right) \otimes \left(Y_0 \longrightarrow Y_1 \longrightarrow Y_2 \longrightarrow \cdots \right)$$
of two filtered spectra is computed as

\begin{center}
$X_0 \otimes Y_0 \longrightarrow \colim $
\adjustbox{scale=0.7} 
{$ \left(\begin{tikzcd} X_0 \smsh Y_1 \\  X_0 \smsh Y_0 \arrow{u} \arrow{r} & X_1 \smsh Y_0 \end{tikzcd} \right) $} 
$\longrightarrow \colim$
\adjustbox{scale=0.7} {$ \left( \begin{tikzcd} X_0 \smsh Y_2 \\ X_0 \smsh Y_1 \arrow{r} \arrow{u} & X_1 \smsh Y_1  \\ X_0 \smsh Y_0 \arrow{r} \arrow{u} & X_1 \smsh Y_0 \arrow{u} \arrow{r} & X_2 \smsh Y_0 \end{tikzcd} \right) $}
$\longrightarrow \cdots.$
\end{center}

For graded spectra, the analogous formula is:

$$(A_0,A_1,A_2,\cdots) \otimes (B_0,B_1,B_2,\cdots) \simeq \left( A_0 \smsh B_0, (A_1 \smsh B_0) \vee (A_0 \smsh B_1), \cdots, \bigvee_{i+j=n} A_i \smsh B_j, \cdots \right).$$





The unit $\S0^{gr}_{\D}$ of $\otimes$ in $\Gr(\Sp^{\D})$ is the constant diagram at $S^0$ in degree 0 and $*$ otherwise; the unit $\S0^{fil}_{\D}$ in $\Fil(\Sp^{\D})$ is $I\S0^{gr}_{\D}.$  We may then talk about $\E_n$-algebras in $\Gr(\Sp^{\D})$ and $\Fil(\Sp^{\D})$.  



There is also an associated graded functor $\text{gr }: \Fil(\Sp^{\D}) \to \Gr(\Sp^{\D})$ such that the composite $\text{gr }\circ I : \Gr(\Sp^{\D}) \to \Gr(\Sp^{\D})$ is an equivalence.   This can be thought of pointwise by the formula $$\text{gr}(X_0\to X_1\to X_2\to \cdots) = X_0, X_1/X_0, X_2/X_1, \cdots.$$
%should say a little about how to construct the associated graded



%reminder to discuss the colim functor

The functors $I$ and $\text{gr}$ can be given symmetric monoidal structures such that the composite $\text{gr }\circ I : \Gr(\Sp^{\D}) \to \Gr(\Sp^{\D})$ is a symmetric monoidal equivalence.  It follows in particular that they extend to functors between the categories of $\E_n$-algebras in $\Gr(\Sp^{\D})$ and $\Fil(\Sp^{\D})$.  Thus, given an $\E_n$-algebra $Y$ in filtered spectra, we obtain a canonical $\E_n$ structure on its associated graded $\text{gr}(Y).$  Conversely, given $X\in \Alg_{\E_n}(\Gr(\Sp^{\D}))$, we obtain $IX\in \Alg_{\E_n}(\Fil(\Sp^{\D})).$  

%%%%
\begin{dfn}
An object $X\in \Alg_{\E_n}(\Fil(\Sp^{\D}))$ is called \emph{$\E_n$-split} if there exists $Y\in \Alg_{\E_n}(\Gr(\Sp^{\D}))$ and an equivalence $X \simeq IY$ in $\Alg_{\E_n}(\Fil(\Sp^{\D}))$.  
\end{dfn}

Given an $\E_n$-split filtered spectrum $X$, we can recover the underlying graded spectrum by taking the associated graded. 
%%%%

\subsection{Square zero rings}

We will now discuss square zero extensions in our framework.  For this, it will be convenient to work with the category $\Gr_u$ of \emph{unital} graded spectra in the strong sense that the unit map induces an equivalence in grading 0.  
Note that there is a fully faithful functor $T:\Sp \to \Gr_u$ which sends a spectrum $A$ to the graded spectrum $$S^0, A, *, *, \cdots.$$  Its essential image is the full subcategory $i: \Gr^{\leq 1}_u \to \Gr_u$ consisting of unital graded spectra $X$ such that $X_k$ is contractible for $k>1$.  In this section, we analyze graded spectra in this subcategory $\Gr^{\leq 1}_u$. Our goal is to show any such graded spectrum admits an essentially unique $\E_n$-algebra structure for any $0\leq n\leq \infty.$  This goal is realized in Proposition \ref{prop:sq0unique}.  


The inclusion $i$ fits into an adjunction
$$L^{\leq 1}:  \Gr_u \xrightleftharpoons{\quad} \Gr_u^{\leq 1} : i$$ where the left adjoint $L^{\leq 1}$ can be thought of as truncating above grading 1.  The localization $L^{\leq 1}$ is visibly compatible with the monoidal structure in the sense that for any $f:X\to Y$ in $\Gr_u$ such that $L^{\leq 1}f$ is an equivalence and any $Z\in \Gr_u$, the natural map $L^{\leq 1} (X\wedge Z) \to L^{\leq 1}(Y\wedge Z)$ is an equivalence.  We are now in the situation of Proposition 2.2.1.9 of \cite{HA}, and so we may conclude that $\Gr_u^{\leq 1}$ inherits a symmetric monoidal structure such that $L^{\leq 1}$ is symmetric monoidal and the inclusion $i$ is lax monoidal.  This monoidal structure can be described explicitly by the formula $$X \otimes_{\Gr_u^{\leq 1}} Y = L^{\leq 1}(X \otimes_{\Gr_u} Y).$$

We may then apply Remark 7.3.2.13 of \cite{HA} to obtain an adjunction at the level of algebras for any integer $0\leq n\leq \infty$:
$$L^{\leq 1}_{\text{alg}}: \Alg_{\E_n}( \Gr_u)  \xrightleftharpoons{\quad} \Alg_{\E_n}(\Gr_u^{\leq 1}) : i_{\text{alg}}.$$

Since the counit $Li \to \text{id}$ before lifting to algebras is an equivalence, we have that the counit $L^{\leq 1}_{\text{alg}}  i_{\text{alg}} \to \text{id}$ is also an equivalence.  This implies in particular that $i_{\text{alg}}$ is fully faithful.  We are now in position to prove the main proposition of this section:

\begin{prop}\label{prop:sq0unique}
Let $0\leq n\leq \infty$ be an integer.  Then, there is a sequence of equivalences of categories $$\Sp \xrightarrow{\bar{T}} \Gr_u^{\leq 1} \longrightarrow \Alg_{\E_n}(\Gr_u^{\leq 1}) \longrightarrow  \Alg_{\E_n}( \Gr_u) \times_{\Gr_u} \Gr^{\leq 1}_u $$ where the first functor $\bar{T}$ is obtained by restricting the codomain of the functor $T:\Sp \to \Gr_u.$  In particular, for any $X\in \Gr_u^{\leq 1}$, the graded spectrum $iX\in \Gr_u$ has an essentially unique $\E_n$-algebra structure.  
\end{prop}
\begin{proof}
The third arrow is defined by $i_{\text{alg}}$, and is an equivalence because $i_{\text{alg}}$ is fully faithful, so it remains to consider the first two arrows.  

We have already seen that the functor $\bar{T}: \Sp \to \Gr^{\leq 1}_u$ is an equivalence of categories.  However, it may be promoted to a symmetric monoidal equivalence when $\Sp$ is given the cocartesian monoidal structure - that is, the monoidal structure defined by $\vee$, the coproduct.  This monoidal structure has a very special property: by Proposition 2.4.3.9 of \cite{HA}, there is for each $n$ an equivalence $\Sp \simeq \Alg^{\vee}_{\E_n}(\Sp)$, where the superscript $\vee$ indicates that we are considering algebras under the wedge.  Informally, this says that any $Y\in \Sp$ admits an essentially unique $\E_n$-algebra structure under the coproduct.  It follows that the same holds for any $X\in \Gr^{\leq 1}_u$, and so there is an equivalence $\Gr^{\leq 1}_u \to \Alg_{\E_n}(\Gr^{\leq 1}_u)$, as desired.    
\end{proof}


\begin{term}
Let $0\leq n\leq \infty$ be an integer.  By taking composing with the colimit functor, Proposition \ref{prop:sq0unique} provides a functor $$\omega_n: \Sp \to \Alg_{\E_n}(\Sp)$$ which we will refer to as the square zero extension.  It sends a spectrum $X$ to a ring with underlying spectrum $S^0\vee X$.  We will call any $\E_n$-algebra structure produced via Proposition \ref{prop:sq0unique} or $\omega_n$ a \emph{square zero} $\E_n$ structure.  
\end{term}

\begin{rmk} \label{rmk:maptosq0}
For any $X\in \Alg_{\E_n}(\Gr_u)$, we have a map $X\to i_{alg}L^{\leq 1}_{alg}X$ of $\E_n$-algebras.  Taking colimits, we obtain a map $\colim X \to \colim i_{alg}L^{\leq 1}_{alg}X$ of $\E_n$ ring spectra.
 We may summarize this informally by saying that any $\E_n$-split ring spectrum $X$ has an $\E_n$ map to the square zero extension determined by its degree one component $X_1$.  
 \end{rmk}

We will need to understand structured maps into square zero extensions.  This amounts to understanding the space of units.  In classical algebra, given a commutative ring $A$ and an $A$-module $M$, the group of units of the square zero extension are given by the formula $$(A\oplus M)^{\times} \simeq A^{\times} \times M.$$ A similar formula holds in our context for suspension spectra of connected spaces.  

\begin{prop}\label{prop:sq0units}
Let $0\leq n\leq \infty$ be an integer and $X\in \cS$ a connected space.  
 There is a canonical equivalence $$GL_1(\omega_n (\Sigma^{\infty} X)) \simeq GL_1(S^0) \times QX$$ of $\E_n$ algebras in spaces, where $QX$ is our notation for $\Omega^{\infty}\Sigma^\infty X$.  
\end{prop}
\begin{proof}
The functors $\omega_n$ are compatible under restriction, so it suffices to prove the statement for $n=\infty.$  For this case, we will show that there is a splitting $$gl_1(\omega_\infty(\Sigma^{\infty}X)) \simeq gl_1(S^0) \vee \Sigma^\infty X$$ of spectra, where $gl_1$ denotes the spectrum of units of an $\E_\infty$-ring introduced in \cite{MQRT}.  We first look at what happens on homotopy.  Recall that for any $\E_\infty$ ring spectrum $Y$, we have the formula $$\pi_*(gl_1(Y)) \simeq (\pi_*(Y))^\times$$ where on the right hand side, we consider $\pi_*(Y)$ as a graded ring.  In our case, this yields an identification $$\pi_*(gl_1(\omega_\infty (\Sigma^{\infty}X))) \simeq (\pi_*(S^0) \oplus \pi_*(\Sigma^{\infty}X))^\times \simeq \pi_*(S^0)^{\times} \times \pi_*(\Sigma^\infty X)$$
where we have used that on homotopy, $\omega_\infty (\Sigma^\infty X)$ is a square zero extension.  To conclude the proof, it suffices to show that the two factors on the right hand side can be realized by maps of spectra.  

The first factor is realized simply by $gl_1$ of the unit map $S^0 \to \omega_\infty (\Sigma^\infty X).$  In fact, it is not difficult to see directly that this map is split.  
%First, note that there is a tautological retraction sequence $$*\longrightarrow \Sigma^\infty X \longrightarrow *$$ of spectra, where $*\in \Sp$ denotes the zero object.  Applying $gl_1\circ \omega_\infty$ allows us to conclude that $gl_1(S^0) = gl_1(\omega_\infty(*))$ splits off of $gl_1(\omega_\infty(\Sigma^\infty X))$ in spectra.  In particular, this determines a map $$a:gl_1(S^0) \to gl_1(\omega_\infty (\Sigma^\infty X)).$$

For the second factor, observe that since $\omega_\infty (\Sigma^\infty X)$ is an $\E_\infty$-ring, it receives a canonical $\E_\infty$ map $$\Sigma^{\infty}_+ QX \longrightarrow \omega_\infty (\Sigma^\infty X)$$ from the free $\E_\infty$ ring on $\Sigma^\infty X$ which extends the canonical map of spectra $\Sigma^\infty X \to \omega_\infty (\Sigma^\infty X).$  Now, note that there is an adjunction \cite{MQRT} $$\Sigma^{\infty}_+\Omega^\infty : \Sp \xrightleftharpoons{\quad} \Alg_{\E_\infty}(\Sp) : gl_1$$ under which the above map may be identified with a map $$b: \Sigma^\infty X \to gl_1 (\omega_\infty (\Sigma^{\infty} X))$$ of spectra.  \textbf{NEED TO SAY A TINY BIT MORE}
%how to compute this on htpy...more or less obv but maybe should explain


Finally, we may take the map $a\vee b: gl_1(S^0)\vee \Sigma^\infty X \to gl_1(\omega_\infty (\Sigma^{\infty} X))$ and the above comments show that it is an equivalence, as desired.  


%bigbus.com







%recall the symmetric monoidal equivalence $\bar{T}:\Sp \to \Gr^{\leq 1}_u$ of Proposition \ref{prop:sq0unique}, where $\Sp$ is given a monoidal structure under $\vee$.  By the proof of Proposition \ref{prop:sq0unique}, there is a canonical retraction sequence of $\E_\infty$ objects in $\Sp$ (under $\vee$) $$* \longrightarrrow X \longrightarrow *.$$  Taking the associated square $0$ spectra



%for general reasons we know gl_1 S^0 splits off gl_1 S^0 v X
%by the \Sigma^\infty \Omega^\infty // gl_1 adjunction and the fact that \Sigma^\infty Q is free E_\infty, you get a map from \Sigma^\infty X to gl_1 S^0 v X that's nontrivial enough on homotopy...so then you just add em up and you're done

\end{proof}


\section{The (Segal-Mitchell-Richter?) Filtration on \texorpdfstring{$\Omega SU(n)$}{Loops SU(n)}}

I believe Mitchell shows in \cite{MitchellSU(n)} that the filtration is filtered $\mathbb{A}_\infty$.  We need to check this.

\begin{cnj} The filtration is $\mathbb{E}_2$.
\end{cnj}
I guess now we know this is just true.  We should probably thank Jacob for bringing to our attention that this conjecture of Mahowald is actually well-known by geometric representation theorists.  
%We can leave this as a conjecture if we don't make easy progress on it.  Seems potentially hard.  We should cite Rotation Invariance for the $n \rightarrow \infty$ case and explain that our conjecture would be a refinement of the one in \cite{MahowaldRichter}.  It might also be worth briefly digging into \cite{MitchellLoopGroup} to see if our arguments generalize to broader loop group contexts.  We could also see if the splitting of all loops of Steifel manifolds is $\mathbb{A}_\infty$-structure, and not just $\Omega SU(n)$.  We can always write a sequel to this preprint if we feel like it.

\section{An \texorpdfstring{$\mathbb{A}_\infty$}{Aoo}-splitting by Weiss Calculus} \label{sec:AooSplit}

The main result of \cite{Arone} shows that the Mitchell-Richter filtration on $\Omega SU(n)$ (and more generally, for loops on a Stiefel manifold) stably splits.  The key insight is that this filtration has extra structure: it is a particular value of a \emph{functor} which has a natural filtration.  The tool that allows for the exploitation of this structure is Weiss's theory of orthogonal or unitary calculus.  

In this section, we extend the methods of \cite{Arone} to produce $\mathbb{A}_\infty$ stable splittings of Stiefel manifolds.  We will begin this section by reviewing the theory of calculus introduced in \cite{Weiss}.  We then make a statement about the multiplicativity of the construction which ...%ok I'll finish this after the section is actually written

\subsection{Weiss Calculus}
In this section, we will briefly review notions of Weiss calculus to set notation and then prove a statement about its multiplicative properties.  The reader is referred to \cite{Weiss} for proofs and additional details.  We note that the discussion there is in the case of real vector spaces, but the results work just the same in the complex case.  We shall also work in the language of $\infty$-categories rather than topological categories, and Remark \ref{rmk:infinityweiss} justifies this passage.  

Let $\J$ be the $\infty$-category which is the nerve of the topological category whose objects are finite dimensional complex vector spaces equipped with a Hermitian inner product and whose morphisms are spaces of linear isometries.  

The theory of Weiss calculus studies functors out of $\J$ in a way analogous to Goodwillie calculus, by understanding successive ``polynomial approximations'' to these functors.  Here, we will discuss only the stable setting where we apply the theory to the functor category $\Sp^{\J}$. The central definition is:

\begin{dfn}\label{dfn:polyfun}
A functor $F\in \Sp^{\J}$ is polynomial of degree at most $n$ if the natural map $$F(V) \to \lim_U F(U\oplus V)$$ is an equivalence, where the limit is indexed over the $\infty$-category of nonzero subspaces $U\subset \mathbb{C}^{n+1}.$
\end{dfn}

As in Goodwillie calculus, the inclusion of the full subcategory $\Poly^{\leq n}(\Sp^{\J}) \subset \Sp^{\J}$ of functors which are polynomial of degree at most $n$ admits a left adjoint $$P_n: \Sp^{\J} \xrightleftharpoons{\quad} \Poly^{\leq n}(\Sp^{\J}): j_n.$$ %center this...
 The unit $\eta_n$ of this adjunction provides for each $F\in \Sp^{\J}$ a natural transformation $F \to P_nF$ which we will refer to as the \emph{degree $n$ polynomial approximation} of $F$. 

\begin{rmk}\label{rmk:infinityweiss}
This universal property was not explicitly stated in \cite{Weiss}, but it follows formally from Weiss's results as follows: the functor $P_n$ and the transformation $\eta_n$ can be defined explicitly as in \cite{Weiss} by iteratively applying the functor $\tau_n: \Sp^{\J} \to \Sp^{\J}$ defined by the formula $$\tau_n F(V) = \lim_U F(U\oplus V)$$ with the limit indexed as in Definition \ref{dfn:polyfun}.   The facts required of the functors $P_n$ in the proof of Theorem 6.1.1.10 in \cite{HA} are precisely the content of Theorem 6.3 of \cite{Weiss}.  
\end{rmk}

Given this universal property, Proposition 5.4 of \cite{Weiss} ensures the existence of a natural Taylor tower $$F \longrightarrow \cdots \longrightarrow P_{n} F \xrightarrow{p_{n-1}} P_{n-1} F \longrightarrow \cdots \longrightarrow P_0F$$ living under any functor $F\in \Sp^{\J}.$  The fiber $D_n F$ of $p_{n-1}$ has the special property that it is polynomial of degree at most $n$ and $P_{n-1} D_n F \simeq 0$.  Such a functor is called \emph{$n$-homogeneous}; such functors are completely classified by the following theorem:

\begin{thm}[{{\cite[Theorem 7.3]{Weiss}}}]
Let $F\in \Sp^{\J}$.  Then $F$ is an $n$-homogeneous functor if and only if there exists a spectrum $\Theta$ with an action of the unitary group $U(n)$ such that $$F(V) = (\Theta \wedge S^{nV})_{hU(n)}.$$
\end{thm}


The observation of Goodwillie, as exploited by \cite{Arone}, is that this provides a canonical way to split certain functorial filtrations whose successive quotients are homogeneous.  More precisely, we have the following theorem:

\begin{thm}[\cite{Arone}]
Suppose $F \in \Sp^{\J}$ is a functor together with an increasing filtration $$0 = F^{(0)} \longrightarrow F^{(1)}\longrightarrow F^{(2)} \longrightarrow  \cdots F$$ by functors $F^{(i)}\in \Sp^{\J}$ with the property that the successive quotients $F^{(n)}/F^{(n-1)}$ are $n$-homogeneous for all integers $n>0$.  Then, each functor $F^{(n)}$ is polynomial of degree at most $n$ and each composite $F^{(n-1)} \longrightarrow F^{(n)} \xrightarrow{\eta_{n-1}} P_{n-1} F^{(n)}$ is an equivalence.
\end{thm}
%we need to at some point deal with the fact that Weiss doesn't actually deal with homog. functors to spectra


%set up the monoidal structures
In order to upgrade the results of \cite{Arone} to structured multiplicative splittings, we must understand the multiplicative properties of the polynomial approximation functors.  More precisely, for a functor $F\in \Sp^{\J}$, we aim to understand the Taylor tower of $F\wedge F$ in terms of the tower for $F.$  The results in this section are likely known to experts, but the authors were not able to locate it in the literature.  They thank Jacob Lurie for suggesting that Proposition \ref{prop:weissmonoidal} is true.  

The idea is to consider all the polynomial approximations at once.  To do so, we first set some additional notation:

\begin{dfn} 
Let $\Cofil(\Sp^{\J})$ denote the functor category $\Fun(\Z_{\geq 0}^{op}, \Sp^{\J}).$  We shall refer to $\Cofil(\Sp^{\J})$ as the category of cofiltered objects in $\Sp^{\J}$.  Its objects can be thought of as towers of functors $Y_0\leftarrow Y_1\leftarrow Y_2 \leftarrow \cdots \in \Sp^{\J}$.
\end{dfn}%ok maybe I'll want cofil^+ or some garbage....

The category $\Cofil(\Sp^{\J})$ is the natural target for the Weiss tower.  The following construction makes this precise:

\begin{cnstr}\label{cnstr:tower}
We now construct a functor $$\text{Tow}: \Sp^{\J} \to \Cofil(\Sp^{\J})$$ with the property that it sends a functor $F\in \Sp^{\J}$ to its Taylor tower $$\text{Tow}(F) = P_0F \longleftarrow P_1F \longleftarrow P_2F \longleftarrow \cdots.$$


Recall that the $P_n$ functors are given as left adjoints of the fully faithful inclusions $\Poly^{\leq n}(\Sp^{\J}) \subset \Sp^{\J}$.  We proceed by telling a parametrized version of this story that includes all $n$ simultaneously.  The proper framework for such a story is the formalism of \emph{relative adjunctions}; these are developed in the $\infty$-categorical context in \cite{HA}, Section 7.3.2.  

Consider the category $\Sp^{\J}\times \Z^{op}_{\geq 0}$ together with the full subcategory $(\Sp^{\J}\times \Z^{op}_{\geq 0})_{\text{poly}} \subset \Sp^{\J}\times \Z^{op}_{\geq 0}$ on the pairs $(F, [n])$ such that $F\in \Poly^{\leq n}(\Sp^{\J}).$  Via projection, these fit into a diagram
%***draw the diagram; the morphisms down are q, p; leftpointing arrow is i, label dia:reladj
 This will be relevant to us because the category of sections of $q$ are precisely $\Cofil(\Sp^{\J}).$  The sections of $p$ can be thought of those cofiltered functors such that the $n$th piece is polynomial of degree at most $n$.  We will denote this category of sections of $p$ by $\Cofil(\Sp^{\J})_{\text{poly}}.$ 

On the fibers over an integer $[n] \in \Z^{op}_{\geq 0}$, we see the inclusion $\Sp^{\J} \leftarrow \Poly^{\leq n}(\Sp^{\J}).$  It is in this sense that the current picture is a parametrized version of the ordinary polynomial approximations.  We now claim that $i$ admits a left adjoint $P^{\text{total}}: \Sp^{\J}\times \Z^{op}_{\geq 0} \to (\Sp^{\J}\times \Z^{op}_{\geq 0})_{\text{poly}}$ \emph{relative} to $\Z^{op}_{\geq 0}.$    The strategy is to use Proposition 7.3.2.6 of \cite{HA}, which tells us that we need to check the following three statements:
\begin{enumerate}
\item The functors $p$ and $q$ are locally Cartesian categorical fibrations.
\item For each $[n]\in \Z^{op}_{\geq 0}$, the functor on fibers $i|_{p^{-1}[n]}:p^{-1}[n] \to q^{-1}[n]$ admits a right adjoint.  
\item The functor $i$ carries locally $p$-Cartesian morphisms of $(\Sp^{\J}\times \Z^{op}_{\geq 0})_{\text{poly}}$ to locally $q$-Cartesian morphisms of $\Sp^{\J}\times \Z^{op}_{\geq 0}$.
\end{enumerate}

Condition (2) is clear from the existence of polynomial approximations in Weiss calculus.  To see conditions (1) and (3), we first note that $q$ is in fact a Cartesian fibration because it is a projection from a product.  Moreover, the $q$-Cartesian morphisms are precisely those morphisms which are equivalences on the $\Sp^{\J}$ coordinate.  We now observe that for any pair $(F, [m]) \in \Sp^{\J}\times \Z^{op}_{\geq 0}$ such that $F\in \Poly^{\leq m}(\Sp^{\J})$ and morphism $\sigma :[n]\to [m]$, any $q$-Cartesian edge lying over $\sigma$ with target $(F, [m])$ is also in the full subcategory $(\Sp^{\J}\times \Z^{op}_{\geq 0})_{\text{poly}}.$  This implies that $p$ is also a Cartesian fibration and that the inclusion $i$ carries $p$-Cartesian edges to $q$-Cartesian edges.  Since any Cartesian fibration is a categorical fibration (\cite[Proposition 3.3.1.7]{HTT}), conditions (1) and (3) are verified.  

We now wish to look at the adjunction at the level of sections of $q$ and $p$.  Considering functors from $\Z_{\geq 0}^{op}$ into Diagram \ref{dia:reladj}, we obtain a new diagram %***draw the diagram, q_*, p_*, P^{\text{total}}_*

which exhibits $P^{\text{total}}$ as a left adjoint of $i_*$ relative to $\Fun(\Z_{\geq 0}^{op},\Z_{\geq 0}^{op}).$  Proposition 7.3.2.5 of \cite{HA} ensures that there is an adjunction at the level of fibers above $\text{id}\in \Fun(\Z_{\geq 0}^{op},\Z_{\geq 0}^{op})$: $$\text{Tow}^*: \Cofil(\Sp^{\J}) \xrightleftharpoons{\quad} \Cofil(\Sp^{\J})_{\text{poly}}:j .$$  Finally, we may precompose with the constant functor $\Sp^{\J}\to \Cofil(\Sp^{\J})$ to obtain the functor $\text{Tow}:\Sp^{\J} \to \Cofil(\Sp^{\J})_{\text{poly}}.$  

It remains to check that $\text{Tow}$ actually recovers the Weiss tower.  %Tow is the right functor...what does that mean?  


This concludes the construction of $\text{Tow}.$ 
\end{cnstr}


The next task is to understand the monoidal structure on $\text{Tow}$.  The idea is that we would like to express $\text{Tow}(F\wedge F)$ in terms of $\text{Tow}(F)$ and a ``Day convolution'' monoidal structure on $\Cofil(\Sp^{\J}).$  However, there is trouble defining the convolution as in Section \ref{sect:monoidal} because smash product does not preserve \emph{limits} of spectra in each variable separately.  The situation becomes better if one restricts to the category $\Sp_{\fin}$ of \emph{finite} spectra.  The full subcategory $\Fun(\Z_{\geq 0}, \Sp_{\fin}^{\J})\subset \Fil(\Sp^{\J})$ is closed under the convolution product defined in Section \ref{sect:monoidal}, and therefore inherits a symmetric monoidal structure.  By Spanier-Whitehead duality, this induces a symmetric monoidal structure on $\Fun(\Z_{\geq 0},(\Sp_{\fin}^{\J})^{op}),$ which in turn induces a symmetric monoidal structure on its opposite, $\Cofil(\Sp^{\J}_{\fin}).$  This can be described explicitly as sending %insert some pictures!

%restrict to functors, all of whose P_n's are finite
One would now hope that Tow restricts to a symmetric monoidal functor when we replace $\Sp^{\J}$ by $\Sp^{\J}_{\fin}.$  However, it is not clear that the derivatives of a functor $\J \to \Sp$ factoring through $\Sp_{\fin}$ will still factor through $\Sp_{\fin}.$  We may simply restrict to this situation.  

\begin{dfn} \label{dfn:stronglyfinite}

Let $\overline{\Sp}^{\J}_{\fin}$ denote the full subcategory of functors $F\in \Sp^{\J}$ such that the functors $F$ and $P_nF$ factor through $\Sp_{\fin}^{\J}$ for all $n$.  We will refer to objects of $\overline{\Sp}^{\J}_{\fin}$ as \emph{strongly finite} functors.  
\end{dfn}

\begin{prop}\label{prop:weissmonoidal}
The Weiss tower functor $\text{Tow}$ restricts to a symmetric monoidal functor $$\text{Tow}_{\text{fin}}: \overline{\Sp}^{\J}_{\fin} \to \Cofil(\overline{\Sp}^{\J}_{\fin}).$$
\end{prop}
\begin{proof}
Combining Construction \ref{cnstr:tower} with Definition \ref{dfn:stronglyfinite}, we obtain the diagram

$$
\begin{tikzcd}
 \Cofil(\Sp^{\J})  &\Cofil(\Sp^{\J})_{\text{poly}} \arrow[l,"j"]  \\
 \Cofil(\overline{\Sp}^{\J}_{\fin}) \arrow[u,hook] &  \Cofil(\overline{\Sp}^{\J}_{\fin})_{\text{poly}} \arrow[l,"\overline{j}"]  \arrow[u,hook]\\
 \end{tikzcd}
$$
of fully faithful functors.  Recall that $j$ admits a left adjoint $\text{Tow}_*.$  It is clear that $\text{Tow}_*$ restricts to a functor $\overline{\text{Tow}}_*:  \Cofil(\overline{\Sp}^{\J}_{\fin}) \to  \Cofil(\overline{\Sp}^{\J}_{\fin})_{\text{poly}}.$  It follows that $\overline{\text{Tow}}_*$ is left adjoint to $\overline{j}.$  

Observe that a finite limit of functors which are polynomial of degree at most $n$ is itself polynomial of degree at most $n$.  This implies that the full subcategory $\overline{j}$ is closed under the symmetric monoidal structure.  It follow that the left adjoint $\overline{\text{Tow}}_*$ naturally admits the structure of a lax symmetric monoidal functor.  It remains to show that $\overline{\text{Tow}}_*$ is in fact symmetric monoidal.  


\end{proof}

\begin{rmk}
Proposition \ref{prop:weissmonoidal} is written in the language of Weiss calculus as that is the present application, but the proof works equally well in Goodwillie calculus.  
\end{rmk}



\subsection{General splitting machinery}


Let $[n]$ denote the linearly ordered set of integers $0\leq i\leq n$.  Define $\Fil_n = \text{Fun}([n], \Sp^{\J})$ and $\Cofil_n = \text{Fun}([n]^{\text{op}},\Sp^{\J})$.  These categories admit functors to $\Sp^{\J}$ by taking colimit and limit, respectively.  Let $\C_n = \Fil_n \times_{\Sp^{\J}} \Cofil_n.$  Finally, let $\Gr_n = \text{Fun}([n]^{\text{ds}}, \Sp^{\J})$ where $[n]^{\text{ds}}$ denotes the underlying discrete category.  We have the following lemma:

\begin{lem}
For all integers $n>0$, there is a fully faithful functor $i_n:\Gr_{n+1} \to \C_n.$  
\end{lem}
\begin{proof}
An element of $\C_n$ is given by a sequence of functors 
\begin{center}
$X_0 \longrightarrow X_1 \longrightarrow \cdots \longrightarrow X_n \simeq Y_n \longrightarrow \cdots \longrightarrow Y_1 \longrightarrow Y_0$ 
\end{center}
where the middle 
\end{proof}

We may then take inverse limits to get a category $\C_\infty = \Fil(\Sp^{\J}) \times_{\Sp^{\J}} \Cofil(\Sp^{\J})$ and a functor $i: \Gr \to \C_\infty$. 

\begin{cor}
The functor $i$ is fully faithful.
\end{cor}
\begin{proof}
%This amounts to checking that taking inverse limits retains fully faithfulness.  this is obvious, thanks to arpon
\end{proof}

at the end, restrict connectivity so that it's monoidal

\section{An \texorpdfstring{$\mathbb{E}_2$}{E2}-splitting in Complex Cobordism} \label{sec:MUE2}

In this brief section, we remark that the $\mathbb{A}_\infty$-splitting $$\Sigma^{\infty}_+ \Omega SU(n) \simeq ???$$ becomes $\mathbb{E}_2$ after smashing with $MU$.  More precisely, we show that there is an equivalence of $\mathbb{E}_2$-$MU$-algebras
$$MU \smsh \Sigma^{\infty}_+ \Omega SU(n) \simeq ???.$$

The $\mathbb{A}_\infty$-$MU$-algebra equivalence constructed in Section \ref{sec:AooSplit} is realized by a map of $\mathbb{A}_\infty$-$\mathbb{S}$-algebras
\begin{equation} \label{SplittingMap}
\Sigma^{\infty} \Omega SU(n) \longrightarrow ???.
\end{equation}

Our task is to show that (\ref{SplittingMap}) may be refined to a morphism of $\mathbb{E}_2$-ring spectra.  We do so by obstruction theory--the key fact powering our proof is that 
$$MU_*\left(\Omega SU(n)\right) \cong 0$$
whenever $*=0$.  \textbf{FIND A REFERENCE}.  In fact, inspired by \cite{ChadwickMandell}, we prove the following more general result:

\begin{thm}
Suppose that $R$ is an $\mathbb{E}_2$-ring spectrum with no homotopy groups in odd degrees.  Then any $\mathbb{A}_\infty$-ring homomorphism %did you need to start with an A\infty map, or just a homotopy ring map
$$\Sigma^{\infty}_+ \Omega SU(n) \rightarrow R$$
lifts to a morphism of $\mathbb{E}_2$-ring spectra.
\end{thm}

\begin{proof} 
By taking connective covers, one learns that any $\mathbb{A}_\infty$-ring homomorphism
$$\Sigma^{\infty}_+ \Omega SU(n) \rightarrow R$$
must factor through the natural $\mathbb{E}_2$-algebra map $\tau_{\ge 0} R \rightarrow R$.  Thus, without loss of generality we will assume that $R$ is $(-1)$-connected.

It is clear that the composite $\mathbb{A}_\infty$-ring homomorphism
$$\Sigma^{\infty}_+ \Omega SU(n) \longrightarrow R \longrightarrow \tau_{\le 0} R \simeq H\pi_0(R)$$
may be lifted to an $\mathbb{E}_2$-ring homomorphism factoring through $\tau_{\le 0} \Sigma^{\infty}_+ \Omega SU(n) \simeq H\mathbb{Z}$.   Suppose now for $q>0$ that we have chosen an $\mathbb{E}_2$-ring homomorphism 
$$\Sigma^{\infty}_+ \Omega SU(n) \longrightarrow \tau_{\le q-1} R$$
lifting the given $\mathbb{A}_\infty$-algebra map
$$\Sigma^{\infty}_+ \Omega SU(n) \longrightarrow R \longrightarrow \tau_{\le q-1} R.$$
We will show that there is no obstruction to the existence of a further $\mathbb{E}_2$-lift $$\Sigma^{\infty}_+ \Omega SU(n) \longrightarrow \tau_{\le q} R.$$
According to \cite[Theorem $4.1$]{ChadwickMandell}, there is a diagram of principal fibrations
$$
\begin{tikzcd}
\mathbb{E}_2\text{-Ring}(\Sigma^{\infty}_+ \Omega SU(n), \tau_{\le q} R) \arrow{r} \arrow{d} & \mathbb{A}_\infty\text{-Ring}(\Sigma^{\infty}_+ \Omega SU(n), \tau_{\le q} R) \arrow{d} \\
\mathbb{E}_2\text{-Ring}(\Sigma^{\infty}_+ \Omega SU(n), \tau_{\le q-1} R) \arrow{r} \arrow{d} & \mathbb{A}_\infty\text{-Ring}(\Sigma^{\infty}_+ \Omega SU(n), \tau_{\le q-1} R) \arrow{d} \\
\cS_*(BSU(n),K(\pi_q R,q+3)) \arrow{r} & \cS_*(SU(n),K(\pi_q R,q+2))
\end{tikzcd}
$$
For $q$ odd, $\tau_{\le q-1} R \simeq \tau_{\le q} R$, so there is no obstruction.  Let us therefore assume that $q$ is even.

Since the cohomology of $BSU(n)$ is even-concentrated with coefficients in any abelian group, we have that $\pi_0 \cS_*(BSU(n),K(\pi_q R,q+3)) \cong H^{q+3}(BSU(n);\pi_q R)$ is zero.  It follows then that the given class $$x \in \pi_0 \mathbb{E}_2\text{-Ring}(\Sigma^{\infty}_+ \Omega SU(n), \tau_{\le q-1} R)$$ admits some lift $$\widetilde{x} \in \mathbb{E}_2\text{-Ring}(\Sigma^{\infty}_+ \Omega SU(n), \tau_{\le q} R).$$  We may need to modify $\widetilde{x}$ to match our chosen $\mathbb{A}_\infty$-ring homomorphism.  This is always possible so long as the map
$$\pi_1(\cS_*(BSU(n),K(\pi_q R,q+3))) \longrightarrow \pi_1(\cS_*(SU(n),K(\pi_q R,q+2)))$$
is surjective.  Said in other terms, this is just the map
$$H^{2q+2}(BSU(n);\pi_q R) \longrightarrow H^{2q+1}(SU(n);\pi_q R) \cong H^{2q+2}(\Sigma SU(n);\pi_q R)$$
induced by the natural map $\Sigma SU(n) \rightarrow BSU(n)$.  It is a classical fact that this map is surjective (it follows from a calculation with the bar spectral sequence, using the fact that the cohomology of $SU(n)$ is exterior). \textbf{Maybe you can check this Allen}.
\end{proof}

\section{Obstructions to a General \texorpdfstring{$\mathbb{E}_2$}{E2}-splitting}

Let $3< n\leq \infty$ be an integer.  We will now show that the $\mathbb{A}_\infty$ splitting $$\Sigma^{\infty}_+ \Omega SU(n) \simeq ???$$ cannot be promoted to an $\mathbb{E}_2$-splitting before smashing with $MU$.  

Suppose that such a splitting existed.  By Remark \ref{rmk:maptosq0}, we would obtain an $\mathbb{E}_2$-ring homomorphism $\Sigma^{\infty}_+ \Omega SU(n) \rightarrow \Sigma^{\infty}_+ \mathbb{CP}^{n-1}$, where $\Sigma^{\infty}_+ \mathbb{CP}^{n-1}$ is given the square-zero multiplication.  Furthermore, the precomposition with the inclusion $\Sigma^{\infty}_+ \mathbb{CP}^{n-1} \longrightarrow \Sigma^{\infty}_+ \Omega SU(n)$ must yield the identity map.  In particular, the map sends the generator of $\pi_2(\Sigma^{\infty}\CP^{n-1})$ to the generator of $\pi_2(\Sigma^{\infty}\Omega SU(n)).$  



Recall now that there is an adjunction \cite{MQRT}
$$\Sigma^{\infty}_+:\textbf{Double Loop Spaces}  \xrightleftharpoons{\quad} \Alg_{\E_2}(\Sp) :GL_1.$$

Using this, we may form the adjoint $\E_2$ map
$$\Omega SU(n) \rightarrow GL_1(\Sigma^{\infty}_+ \mathbb{CP}^{n-1}).$$

The right hand side is identified as an $\E_2$ algebra by Proposition \ref{prop:sq0units}.  In particular, we obtain an $\E_2$ composite $$\phi: \Omega SU(n) \to GL_1(\Sigma^{\infty}_+ \mathbb{CP}^{n-1}) \simeq  GL_1(S^0) \times Q\CP^{n-1} \to Q\CP^{n-1}$$ which has the additional property that it is an isomorphism on $\pi_2$.

We now show that such a map $\phi$ cannot exist due to the operations that exist in the homotopy of an $\E_2$ algebra.  

\begin{obs}Let $Y\in \Alg_{\E_2}(\cS)$, and suppose we are given a map $S^2\to Y$.  This extends to an $\E_2$ map $\Omega^2 S^4 \to Y.$  We may precompose with the map $S^5 \to \Omega^2 S^4$ adjoint to the Hopf map $S^7\to S^4$.  This procedure determines a natural operation $$\nu^u: \pi_2(Y) \to \pi_5(Y)$$ in the homotopy of any $\E_2$-algebra in spaces.  
\end{obs}

\begin{rmk} \label{rmk:multnu}
The notation is meant to hint at the fact that if $Y = \Omega^\infty X$ comes from a spectrum, then the operation $\nu^u$ is given by multiplication by the element $\nu \in \pi_3(\mathbb{S})^{\wedge}_2$ from the $2$-primary homotopy groups of the sphere spectrum.  Thus, $\nu^u$ is an unstable version of $\nu$ that is already seen in any $\E_2$ algebra in spaces.    
\end{rmk}

Finally, we show that $\phi$ cannot be compatible with $\nu^u$ on homotopy by directly computing $\nu^u$ on either side.

For $n>3$, observe that the natural map $\Omega SU(n) \to BU$ is an isomorphism in homology up to degree $7$.  This implies that $\pi_5(\Omega SU(n)) \simeq \pi_5(BU) \simeq 0$ because $BU$ is even.  Hence, $\nu^u$ is trivial on the generator of $\pi_2(\Omega SU(n)).$  

Similarly, the map $Q\CP^{n-1} \to Q\CP^\infty$ is an isomorphism on $\pi_5$ for $n>3$.  However, it was computed in \cite[Theorem II.8]{Liulevicius} that $\pi_5(\CP^{\infty})=\mathbb{Z}/2$ generated by $\nu$ times the degree $2$ generator.  Hence, by Remark \ref{rmk:multnu}, if $\beta\in \pi_2(Q\CP^{n-1})$ denotes the generator, then $\nu^u(\beta)\in \pi_5(Q\CP^{n-1})$ is nontrivial.  This implies that there can be no $\E_2$ map $\phi$ which induces an isomorphism on $\pi_2$ and concludes the proof.  

\begin{rmk}%say this better
Taking the limit as $n\to\infty$, we obtain the statement that the map $BU\to Q\CP^{\infty}$ implementing the splitting principle does not lift to an $\E_2$ map.  This map is well-studied: among other places, it appears as the first connecting map in the Weiss tower for the functor $V\mapsto BU(V)$.  As such, it can be seen as a ``$BU$-analog'' to the Kahn-Priddy map.  
\end{rmk}

%maybe remark that BU -> QCP^\infty not being E_2 is still true after taking \Sigma^\infty_+, and remark that this is like a version of the splitting principle. perhaps also remark about stuff being known about this map, its place in the weiss tower, etc.

\section{Snaith's Construction of Periodic Complex Bordism} \label{sec:SnaithSplitting}

A classical theorem of Snaith \cite{SnaithOriginal} gives an equivalence of homotopy commutative ring spectra $$\Sigma^{\infty}_+ BU [\beta^{-1}] \simeq MUP.$$  The equivalence arises from considering the total $MU$-Chern class map $BU \to GL_1(MUP).$  It is known from \cite{SnaithNotMultiplicative} that the total Chern class in integral homology is not an infinite loop map.  It follows from the existence of an $\E_\infty$ map $MUP \to H\Z P$ from periodic complex bordism to periodic integral homology that Snaith's equivalence is not an equivalence of $\E_\infty$ ring spectra.  The following theorem refines this observation:


%should check if the obstruction in SnaithNotMultiplicative is also E_3...I would think it is
%should we also cite the totaro paper that does literally the same thing as SnaithNotMultiplicative?

\begin{thm}
The equivalence $\Sigma^{\infty}_+ BU [\beta^{-1}] \simeq MUP$ is $\mathbb{E}_2$ but not $\mathbb{E}_3$.
\end{thm}

\begin{proof}
Proof goes here
\end{proof}

Comment now about GepnerSnaith.
We should cite at some point here or the introduction all of \cite{SnaithNotMultiplicative},  \cite{GepnerSnaith}, and the Snaith book with the original splitting.

\section{Miscellaneous stuff here}

It would be nice to at some point deal with showing the associated graded $E_2$ structure of BU is the thom spectrum VMU(n).  I've directly pasted in some writing from a previous argument I claimed, but it definitely uses that $\coprod BU(n)$ is an $E_2$ algebra over $\Z _{\geq 0}$ which I never got straight an actual proof of.  

\begin{prop}The associated graded of $\Sigma^{\infty}_+BU$ is $E_2$ equivalent to the Thom spectrum $\bigvee MU(n).$
\end{prop}
\begin{proof}
Let $R = BU$ with its natural filtration, and let $R^{\oplus} = \coprod BU(n)$ with its natural filtration.  Let $M$ be the ($E_\infty$) filtered spectrum which is $MU(n)$ in degree $n$, and all maps are $0$.  In other words, $\bigvee MU(n)$ with its natural filtration is $I(res(M))$.  

We begin with a filtered $E_\infty$ map $z:R^\oplus \to I(res(M))$ coming from the zero section.  Then, $R^\oplus$ comes with the structure of an $E_2$ algebra over $\Z_{\geq 0}^{fil}$.  In fact, $I(res(M))$  has a trivial structure as an $E_\infty$-algebra over $\Z_{\geq 0}^{fil}$ via the augmentation $\Z_{\geq 0}^{fil}\to S^{0,fil} \to I(res(M))$.  We may then tensor $z$ along the augmentation to get a map of $E_2$ filtered spectra $z':R \to I(res(M))$.  

There is a canonical equivalence $I(res(M)) \otimes \mathbb{A} \simeq M$ because $M$ is in the image of $\mathbb{A} \otimes I(-)$ (that is, all the maps in the filtration of $M$ were zero).  As such, $M$ acquires a canonical structure as an $\mathbb{A}$ algebra such that the map $M \otimes \mathbb{A} \to M$ is a map of $E_2$ rings (in fact I think it's $E_\infty$?).  

Finally, we observe that we may tensor $z'$ with $\mathbb{A}$ and compose with the multiplication map to get an $E_2$ map $R\otimes A \to M \otimes A \to M$ which is the right thing up to homotopy, so it's an equivalence.  
\end{proof}

******  

What is $\Sigma^{\infty}_+ \Omega SU(n)[\beta^{-1}]$, by the way?  Is it related to a periodic version of the $X(n)$-filtration of $MU$??

\bibliographystyle{amsalpha}
\bibliography{Bibliography}

\end{document}