\documentclass[reqno, oneside]{amsart}
\usepackage{hyperref}

%\usepackage[export]{adjustbox}
%\usepackage[dvips]{pict2e}
%\usepackage{amsmath,amsthm}
\usepackage{comment}
\usepackage{amsfonts, fullpage, fancyhdr, qtree, amsmath, tipa, amssymb, hyperref, url, amsthm, subfigure, xy, tikz-cd, verbatim}

\usepackage[color=cyan!40]{todonotes}



\usepackage[a4paper]{geometry}


%----------------------------------------------------------------------%

%\renewcommand{\appendixname}{Example}
%\swapnumbers

\theoremstyle{definition}
\newtheorem{nul}{}[section]
\newtheorem{dfn}[nul]{Definition}
\newtheorem{axm}[nul]{Axiom}
\newtheorem{rmk}[nul]{Remark}
\newtheorem{cnstr}[nul]{Construction}
\newtheorem{ntn}[nul]{Notation}
\newtheorem{exm}[nul]{Example}
\newtheorem{ctrexm}[nul]{Counterexample}
\newtheorem{rec}[nul]{Recollection}
\newtheorem{exr}[nul]{Exercise}
\newtheorem{wrn}[nul]{Warning}
\newtheorem*{dfn*}{Definition}
\newtheorem*{axm*}{Axiom}
\newtheorem*{ntn*}{Notation}
\newtheorem*{exm*}{Example}
\newtheorem*{exr*}{Exercise}
\newtheorem*{int*}{Intuition}
\newtheorem*{qst*}{Question}


\theoremstyle{plain}
\newtheorem{sch}[nul]{Scholium}
\newtheorem{thm}[nul]{Theorem}
\newtheorem{prop}[nul]{Proposition}
\newtheorem{lem}[nul]{Lemma}
\newtheorem{sublem}{Lemma}[nul]
\newtheorem{por}[nul]{Porism}
\newtheorem{cnj}[nul]{Conjecture}
\newtheorem{cor}{Corollary}[nul]
\newtheorem*{thm*}{Theorem}
\newtheorem*{prop*}{Proposition}
\newtheorem*{cor*}{Corollary}
\newtheorem*{lem*}{Lemma}
\newtheorem*{cnj*}{Conjecture}


%----------------------------------------------------------------------%

\DeclareMathOperator{\Aut}{\text{Aut}}
\DeclareMathOperator{\Tr}{\text{Tr}}
\DeclareMathOperator{\Res}{\text{Res}}
\DeclareMathOperator{\im}{\text{im}}
\DeclareMathOperator*{\colim}{\text{colim}}
\DeclareMathOperator{\Map}{\text{Map}}
\DeclareMathOperator{\cofiber}{\text{cofiber}}
\DeclareMathOperator{\fiber}{\text{fiber}}
\DeclareMathOperator{\Hom}{\text{Hom}}
\DeclareMathOperator{\Skel}{\text{Skel}}
\DeclareMathOperator*{\hocolim}{\text{hocolim}}
\DeclareMathOperator*{\holim}{\text{holim}}
\DeclareMathOperator{\smsh}{\wedge}



\DeclareMathOperator{\C}{\mathcal{C}}
\DeclareMathOperator{\CP}{\mathbb{CP}}
\DeclareMathOperator{\Z}{\mathbb{Z}}
\DeclareMathOperator{\E}{\mathbb{E}}
\DeclareMathOperator{\Q}{\mathbb{Q}}
\DeclareMathOperator{\m}{\mathfrak{m}}
\DeclareMathOperator{\G}{\mathbb{G}}
\DeclareMathOperator{\F}{\mathbb{F}}
\DeclareMathOperator{\cG}{\mathcal{G}}
\DeclareMathOperator{\cF}{\mathcal{F}}
\DeclareMathOperator{\cS}{\mathcal{S}}
\DeclareMathOperator{\Ring}{\textbf{Ring}}
\DeclareMathOperator{\Aff}{\textbf{Aff}}
\DeclareMathOperator{\Spec}{\text{Spec}}
\DeclareMathOperator{\Poly}{\text{Poly}}
\DeclareMathOperator{\Ext}{\text{Ext}}
\DeclareMathOperator{\nil}{\text{nil}}
\DeclareMathOperator{\Gr}{\textbf{Gr}}
\DeclareMathOperator{\Fil}{\textbf{Fil}}
\DeclareMathOperator{\Fun}{\text{Fun}}
\DeclareMathOperator{\Alg}{\text{Alg}}
\DeclareMathOperator{\Sp}{\text{Sp}}
\DeclareMathOperator{\Spaces}{\mathcal{S}}
\DeclareMathOperator{\S0}{\mathbb{S}}
\DeclareMathOperator{\J}{\mathcal{J}}
\DeclareMathOperator{\Cofil}{\textbf{Cofil}}

%----------------------------------------------------------------------%

\usepackage{tikz}
\usetikzlibrary{matrix,arrows,decorations}
\usepackage{tikz-cd}

\usepackage{adjustbox}

%----------------------------------------------------------------------%

\hyphenation{Mack-ey mon-oid-al Wald-hau-sen}

%----------------------------------------------------------------------%
%----------------------------------------------------------------------%


\begin{document}

\title{Structured Splittings of $\Omega SU(n)$ and Snaith's Construction of Periodic Complex Bordism}
\author{Jeremy Hahn and Allen Yuan}

\begin{abstract}
We show that the ??? filtration on the affine Grassmannian of $SL_n(\mathbb{C})$, known to topologists as the Bott filtration on $\Omega SU(n)$, stably splits as an $\mathbb{A}_\infty$ but not as an $\mathbb{E}_2$ filtration.  We further prove that the splitting becomes $\mathbb{E}_2$ after smashing with any complex-oriented homology theory.  As a limiting case, we study the coherence of Snaith's construction of periodic complex cobordism.  We determine that it is a construction of $MUP$ as an $\mathbb{E}_2$-ring spectrum, but not as an $\mathbb{E}_3$-ring spectrum.
\end{abstract}



%----------------------------------------------------------------------%
%----------------------------------------------------------------------%


\setcounter{tocdepth}{1}
\maketitle

\tableofcontents


%----------------------------------------------------------------------%

\section{Introduction}

The complex cobordism spectrum $MU$ has played a central role in homotopy theory ever since Quillen \cite{Quillen} connected its homotopy with the theory of one-dimensional formal group laws.  As the Thom spectrum of the canonical bundle over $BU$, $MU$ naturally acquires an immense amount of structure: it is an $\mathbb{E}_\infty$-ring spectrum.  To this day, however, much remains unknown about the full nature of this $\mathbb{E}_\infty$ structure.  For example, almost nothing is known about the k-invariants of the spectrum of units $gl_1(MU)$ \cite{LawsonBP}.

As a spectrum, $MU$ is classically constructed as the direct limit of the sequence
$$\mathbb{S}=MU(0) \longrightarrow \Sigma^{-2} MU(1) \longrightarrow \cdots \longrightarrow \Sigma^{-2n} MU(n) \longrightarrow \cdots,$$
where $MU(n)$ is the Thom spectrum of the canonical bundle over $BU(n)$.

This suggests that perhaps the more elemental object is $$\bigvee MU(n),$$ the Thom spectrum of the $J$-homomorphism
$$\textbf{Vect} \stackrel{J}{\longrightarrow} \text{Pic}(\mathbb{S})$$
that takes a vector space $V$ to its one-point compactification $J(V)=S^V$.  Since $$J(V \oplus W) \simeq S^V \smsh S^W,$$ the Thom spectrum $\bigvee MU(n)$
is naturally an $\mathbb{E}_\infty$-ring spectrum.  Inverting the Bott element $\beta \in \pi_2(MU(1)) \cong \pi_2(\mathbb{CP}^{\infty})$, one obtains the \textit{periodic} complex cobordism spectrum
$$MUP \simeq \left(\bigvee MU(n) \right)[\beta^{-1}].$$

This periodic spectrum $MUP$ is a minor variation on $MU$ itself--there is a wedge decomposition $$MUP \simeq \bigvee_{a \in \mathbb{Z}} \Sigma^{2a} MU,$$ and the inclusion $MU \rightarrow MUP$ onto the $a=0$ factor is an an $\mathbb{E}_\infty$-ring homomorphism.

In 1979, Victor Snaith \cite{SnaithBook} gave a second, and fundamentally different, presentation of periodic complex cobordism:

\begin{thm}[Snaith] \label{SnaithSplitting}
As homotopy commutative ring spectra, $$MUP \simeq \Sigma^{\infty}_+ BU[\beta^{-1}].$$  More generally, there is an equivalence of homotopy commutative rings
$$\bigvee MU(n) \simeq \Sigma^{\infty}_+ BU.$$
\end{thm}

\begin{rmk} In the above, $\Sigma^{\infty}_+ BU$ acquires an $\mathbb{E}_\infty$-structure (and hence a homotopy commutative ring structure) from the fact that $BU$ is an infinite loop space.  One may think of $\Sigma^{\infty}_+ BU$ as the group ring of the topological group $BU$.
\end{rmk}

The genesis of this paper was an attempt to use Snaith's theorem to study the $\mathbb{E}_\infty$-ring structure on $MUP$.  This seemed like an especially reasonable idea in light of two facts:

\begin{enumerate}
\item By another theorem of Snaith, periodic $K$-theory $KU$ may be constructed as $$KU \simeq \Sigma^{\infty}_+ \mathbb{CP}^{\infty}[\beta^{-1}].$$  This is an equivalence of $\mathbb{E}_\infty$-ring spectra \textbf{CITE}.
\item In \cite{GepnerSnaith}, Gepner and Snaith prove a motivic analogue of Theorem \ref{SnaithSplitting}.  In particular, they prove an equivalence 
$$\Sigma^{\infty}_+ BGL[\beta^{-1}] \simeq PMGL.$$
They then use the natural $\mathbb{E}_\infty$-ring structure on $\Sigma^{\infty}_+ BGL[\beta^{-1}]$ to \textbf{define} an $\mathbb{E}_\infty$-ring structure on $PMGL$. 
\end{enumerate}

\textbf{Allen, there is something I am confused about.  In Gepner-Snaith, they claim that there is an $\mathbb{E}_\infty$ ring homomorphism from $PMGL$ to $KU$ given by inverting $\beta$ in the suspension of the determinant map
$$BU \rightarrow \mathbb{CP}^{\infty}.$$
I thought though that we decided there is no $\mathbb{E}_\infty$-ring map to $H\mathbb{Z}P$.  What's up with that?}

However, as it turns out (though it is not at all obvious from the modern literature and in particular not mentioned in \cite{GepnerSnaith}), another old theorem of Snaith \cite{SnaithNotMultiplicative} shows that our idea was entirely unreasonable:

\begin{thm}[Snaith]
As $\mathbb{E}_\infty$-rings,
$$MUP \not \simeq \Sigma^{\infty}_+ BU[\beta^{-1}].$$
\end{thm}

In the final Section \ref{sec:SnaithSplitting} of this paper, we refine Snaith's results in to what we consider their definitive form:

\begin{thm}
There is an equivalence of $\mathbb{E}_2$-ring spectra
$$MUP \simeq \Sigma^{\infty}_+ BU[\beta^{-1}],$$
but $MUP \not \simeq \Sigma^{\infty}_+ BU[\beta^{-1}]$ as $\mathbb{E}_3$-ring spectra.  There is an equivalence of $\mathbb{A}_\infty$-ring spectra
$$\bigvee MU(n) \simeq \Sigma^{\infty}_+ BU,$$
but $\bigvee MU(n) \not \simeq \Sigma^{\infty}_+ BU$ as $\mathbb{E}_2$-ring spectra.
\end{thm}

With the focus now on $\mathbb{E}_2$-algebras, it is natural to view Snaith's splitting result as the limiting case of a sequence of other stable splittings.  Consider the filtration
$$* \simeq \Omega SU(1) \longrightarrow \Omega SU(2) \longrightarrow \cdots \longrightarrow \Omega SU(n) \longrightarrow \cdots \longrightarrow \Omega SU \simeq BU,$$
where the last equivalence is by Bott periodicity.  Taking suspension spectra, we obtain a filtration of $\mathbb{E}_2$-ring spectra
$$\mathbb{S} \longrightarrow \Sigma^{\infty}_+ \Omega SU(2) \longrightarrow \cdots \longrightarrow \Sigma^{\infty}_+ \Omega SU(n) \longrightarrow \cdots \longrightarrow \Sigma^{\infty}_+ BU.$$

It is a theorem of Crabb and Mitchell \cite{CrabbMitchell} that, for $n>1$, $\Sigma^{\infty}_+ \Omega SU(n)$ splits as an infinite wedge sum.  We study the coherence of their splitting in Sections \ref{sec:AooSplit} and \textbf{BLAH}:

\begin{thm}
The Crabb--Mitchell stable splitting of $\Sigma^{\infty}_+ \Omega SU(n)$ is a splitting of $\mathbb{A}_\infty$-ring spectra, but not of $\mathbb{E}_2$-ring spectra.
\end{thm}

To be precise about what we mean by a splitting of $\mathbb{E}_n$-ring spectra, we need to introduce a bit of abstract terminology.

\begin{rmk}
We freely use the language of $\infty$-categories throughout this paper, refering to an $\infty$-category simply as a category.
\end{rmk}

In Section \ref{sec:FilGra} we will review the symmetric monoidal categories $\Fil$ and $\Gr$ of filtered and graded spectra, respectively.  A filtered spectrum is an infinite sequence
$$X_0 \longrightarrow X_1 \longrightarrow X_2 \longrightarrow X_3 \longrightarrow \cdots$$
of spectra.  The tensor product $$\left(X_0 \longrightarrow X_1 \longrightarrow X_2 \longrightarrow \cdots \right) \otimes \left(Y_0 \longrightarrow Y_1 \longrightarrow Y_2 \longrightarrow \cdots \right)$$
of two filtered spectra is computed as a Day convolution

\begin{center}
$X_0 \otimes Y_0 \longrightarrow \colim $
\adjustbox{scale=0.7} 
{$ \left(\begin{tikzcd} X_0 \smsh Y_1 \\  X_0 \smsh Y_0 \arrow{u} \arrow{r} & X_1 \smsh Y_0 \end{tikzcd} \right) $} 
$\longrightarrow \colim$
\adjustbox{scale=0.7} {$ \left( \begin{tikzcd} X_0 \smsh Y_2 \\ X_0 \smsh Y_1 \arrow{r} \arrow{u} & X_1 \smsh Y_1  \\ X_0 \smsh Y_0 \arrow{r} \arrow{u} & X_1 \smsh Y_0 \arrow{u} \arrow{r} & X_2 \smsh Y_0 \end{tikzcd} \right) $}
$\longrightarrow \cdots.$
\end{center}

A graded spectrum, on the other hand, is simply an ordered sequence $(A_0,A_1,A_2, \cdots)$ of spectra.  The tensor product  is computed as
$$(A_0,A_1,A_2,\cdots) \otimes (B_0,B_1,B_2,\cdots) \simeq \left( A_0 \smsh B_0, (A_1 \smsh B_0) \vee (A_0 \smsh B_1), \cdots, \bigvee_{i+j=n} A_i \smsh B_j, \cdots \right).$$

There is a sequence of symmetric monoidal functors
$$
\begin{tikzcd}[column sep=huge]
\Gr \arrow{r}{I} & \Fil \arrow{r}{\text{colim}} & \Sp,
\end{tikzcd}
$$
where $I$ sends the graded spectrum $(A_0,A_1,A_2,\cdots)$ to the filtered spectrum
$$
I(A_0,A_1,A_2,\cdots) = \left( A_0 \longrightarrow A_0 \vee A_1 \longrightarrow A_0 \vee A_1 \vee A_2 \longrightarrow \cdots\right).
$$

\begin{dfn}
We say that an $\mathbb{E}_n$-ring spectrum is $\mathbb{E}_n$-\textbf{split} if it is equivalent to the image of an $\mathbb{E}_n$-algebra in $\Gr$ under the composite $\text{colim} \circ I.$  Similarly, a filtered $\mathbb{E}_n$-algebra is $\mathbb{E}_n$-split if it is equivalent to the image under $I$ of an $\mathbb{E}_n$-algebra in $\Gr$.
\end{dfn}

\textbf{BLAH}

In section \ref{sec:MUE2} we will show:
\begin{thm}
The ??? filtration on $\Sigma^{\infty}_+ \Omega SU(n)$ is $\mathbb{E}_2$-split after smashing with $MU$.
\end{thm}

\textbf{BLAH}


%Notations
%Sp for spectra, S for spaces?

\section{Filtered and Graded Ring Spectra} \label{sec:FilGra}

%Cite Rotation Invariance \cite{LurieRot} and lay out the basic theory.  Explain why an $\mathbb{E}_n$ filtered spectrum with $\mathbb{E}_n$-backwards maps must be split.
%I'd be happy to do this.  It's in progress below, but I'm not great with texing and notation so please give me suggestions.

Here we review a framework from \cite{LurieRot} for studying graded and filtered objects.  The reader is referred to \cite{LurieRot} for a more thorough treatment and all proofs.  

Let $\C$ be a stable $\infty$-category.  Denote by $\Z_{\geq 0}$ the poset of non-negative integers, and by $\Z_{\geq 0}^{ds}$ the corresponding discrete category.  The reader is warned that our numbering conventions are opposite the ones in \cite{LurieRot}.

\begin{dfn} 
Let $\Gr(\C)$ denote the functor category $\Fun(\Z_{\geq 0}^{ds}, \C).$  We shall refer to $\Gr(\C)$ as the category of graded objects in $\C$.  Its objects can be thought of as sequences $X_0, X_1,X_2,\cdots \in \C$.
\end{dfn}

 \begin{dfn} 
Let $\Fil(\C)$ denote the functor category $\Fun(\Z_{\geq 0}, \C).$  We shall refer to $\Fil$ as the category of filtered objects in $\C$.  Its objects can be thought of as sequences $Y_0\to Y_1\to Y_2 \to \cdots \in \C$ filtering $\colim_i Y_i$.  
 \end{dfn}
 
 For us, the category $\C$ will always be either spectra or the category of functors from an indexing category to spectra.  Because limits, colimits, and smash products of functors are taken pointwise, these cases are essentially the same.  Thus, we will suppress $\C$ in what follows and refer to its objects as spectra.

%paragraph describes the functors
The obvious map $\Z_{\geq 0}^{ds} \to \Z_{\geq 0}$ induces a restriction functor $\text{res}: \Fil \to \Gr.$  The restriction is right adjoint to a functor $I: \Gr \to \Fil$ given by left Kan extension.  The functor $I$ can be thought of as taking a graded spectrum $X_0,X_1,X_2,\cdots$ to the filtered spectrum $X_0\to X_0\vee X_1\to X_0\vee X_1\vee X_2\to \cdots.$    There is also an associated graded functor $\text{gr }: \Fil \to \Gr$ such that the composite $\text{gr }\circ I : \Gr \to \Gr$ is an equivalence.   

%paragraph describes the monoidal structures
The categories $\Gr$ and $\Fil$ are given symmetric monoidal structures via the Day convolution; we denote the operation in both cases by $\otimes$.  The unit $\S0^{gr}$ of $\otimes$ in graded spectra is $S^0$ in degree 0 and $*$ otherwise; the unit $\S0^{fil}$ in filtered spectra is $I\S0^{gr}.$  We may then talk about $\E_n$-algebras in $\Gr$ and $\Fil$.  

%We will require these to be unital in the strong sense that the unit map is an equivalence in degree $0$.  

%maybe we have to be even more careful about unital stuff with this extra assumption, or we want to just assume ALL graded spectra are unital in this sense.  I think that might be fine.  

The functors $I$ and $\text{gr}$ can be given symmetric monoidal structures such that the composite $\text{gr }\circ I : \Gr \to \Gr$ is a symmetric monoidal equivalence.  It follows in particular that they extend to functors between the categories of $\E_n$-algebras in $\Gr$ and $\Fil$.  Thus, given an $\E_n$ algebra $Y$ in filtered spectra, we obtain a canonical $\E_n$ structure on its associated graded $\text{gr}(Y).$  Conversely, given $X\in \Alg_{\E_n}(\Gr)$, we obtain $IX\in \Alg_{\E_n}(\Fil).$  

%%%%
\begin{dfn}
An object $X\in \Alg_{\E_n}(\Fil)$ is called \emph{$\E_n$-split} if there exists $Y\in \Alg_{\E_n}(\Gr)$ and an equivalence $X \simeq IY$ of $\E_n$ filtered spectra.  
\end{dfn}

Given an $\E_n$-split filtered spectrum $X$, we can recover the underlying graded spectrum by taking the associated graded. 
%%%%

\subsection{Square zero rings}
%being careful about square zero extensions


We will now discuss square zero extensions in our framework.  For this, it will be convenient to work with the category $\Gr_u$ of \emph{unital} graded spectra in the strong sense that the unit map induces an equivalence in grading 0.  
Note that there is a fully faithful functor $T:\Sp \to \Gr_u$ which sends a spectrum $A$ to the graded spectrum $$S^0, A, *, *, \cdots.$$  Its essential image is the full subcategory $i: \Gr^{\leq 1}_u \to \Gr_u$ consisting of unital graded spectra $X$ such that $X_k$ is contractible for $k>1$.  The inclusion $i$ admits a left adjoint $L^{\leq 1}$ which can be thought of as truncating above grading 1.  
%need to do some more HA citing here
This localization is visibly compatible with the monoidal structure, and so $\Gr_u^{\leq 1}$ inherits a symmetric monoidal structure such that $L^{\leq 1}$ is symmetric monoidal and the inclusion $i$ is lax monoidal.  Furthermore, the restricted functor $\bar{T}: \Sp \to \Gr^{\leq 1}_u$ may be promoted to a symmetric monoidal equivalence where $\Sp$ is given the cocartesian monoidal structure.   By Proposition 2.4.3.9 of \cite{HA}, any $Y\in \Gr_u{\leq 1}$ admits an essentially unique $\E_n$-algebra structure for any integer $0\leq n\leq \infty$.  

\begin{dfn}
Let $Y\in \Gr_u{\leq 1}$.  We will refer to $iY\in \Alg_{\E_n}(\Gr_u)$ and $\colim iY\in \Alg_{\E_n}(\Sp)$ with their induced monoidal structures as having the \emph{square zero} $\E_n$ structure.
\end{dfn}

In particular, for any $X\in \Alg_{\E_n}(\Gr_u)$, we have a map $X\to iL^{\leq 1}X$ which by Remark 7.3.2.13 of \cite{HA} is a map of $\E_n$-algebras when $iL^{\leq 1}X$ is given the induced monoidal structure.  However, this is exactly the square zero structure by uniqueness.  We may then take colimits to obtain a canonical $\E_n$ map $\colim X \to \colim iL^{\leq 1}X.$  We may summarize this discussion informally by saying that any $\E_n$-split ring spectrum $X$ has an $\E_n$ map to the square zero extension determined by $X_1$.  

A nice feature of square zero $\E_n$ rings is that it is easy to understand their space of units.  

\begin{prop}
Let $0\leq n\leq \infty$ be an integer and $X\in \Sp$.  Give the spectrum $S^0\vee X$ the square zero $\E_n$ structure.  There is a canonical equivalence $$GL_1(S^0\vee X) \simeq GL_1(S^0) \times \Omega^{\infty} X.$$
\end{prop}
\begin{proof}
TBD...suffices to take $n=\infty$ by uniqueness.  
\end{proof}


\section{The (Segal-Mitchell-Richter?) Filtration on \texorpdfstring{$\Omega SU(n)$}{Loops SU(n)}}

I believe Mitchell shows in \cite{MitchellSU(n)} that the filtration is filtered $\mathbb{A}_\infty$.  We need to check this.

\begin{cnj} The filtration is $\mathbb{E}_2$.
\end{cnj}
I guess now we know this is just true.  We should probably thank Jacob for bringing to our attention that this conjecture of Mahowald is actually well-known by geometric representation theorists.  
%We can leave this as a conjecture if we don't make easy progress on it.  Seems potentially hard.  We should cite Rotation Invariance for the $n \rightarrow \infty$ case and explain that our conjecture would be a refinement of the one in \cite{MahowaldRichter}.  It might also be worth briefly digging into \cite{MitchellLoopGroup} to see if our arguments generalize to broader loop group contexts.  We could also see if the splitting of all loops of Steifel manifolds is $\mathbb{A}_\infty$-structure, and not just $\Omega SU(n)$.  We can always write a sequel to this preprint if we feel like it.

\section{An \texorpdfstring{$\mathbb{A}_\infty$}{Aoo}-splitting by Weiss Calculus} \label{sec:AooSplit}

The main result of \cite{Arone} shows that the Mitchell-Richter filtration on $\Omega SU(n)$ (and more generally, for loops on a Stiefel manifold) stably splits.  The key insight is that this filtration has extra structure: it is a particular value of a \emph{functor} which has a natural filtration.  The tool that allows for the exploitation of this structure is Weiss's theory of orthogonal or unitary calculus.  

In this section, we extend the methods of \cite{Arone} to produce $\mathbb{A}_\infty$ stable splittings of Stiefel manifolds.  We will begin this section by reviewing the theory of calculus introduced in \cite{Weiss}.  We then make a statement about the multiplicativity of the construction which ...%ok I'll finish this after the section is actually written

\subsection{Weiss Calculus}
In this section, we will briefly review notions of Weiss calculus to set notation and then prove a statement about its multiplicative properties.  The reader is referred to \cite{Weiss} for proofs and additional details.  We note that the discussion there is in the case of real vector spaces, but the results work just the same in the complex case.  We shall also work in the language of $\infty$-categories rather than topological categories, and Remark \ref{rmk:infinityweiss} justifies this passage.  

Let $\J$ be the $\infty$-category which is the nerve of the topological category whose objects are finite dimensional complex vector spaces equipped with a Hermitian inner product and whose morphisms are spaces of linear isometries.  

The theory of Weiss calculus studies functors out of $\J$ in a way analogous to Goodwillie calculus, by understanding successive ``polynomial approximations'' to these functors.  Here, we will discuss only the stable setting where we apply the theory to the functor category $\Sp^{\J}$. The central definition is:

\begin{dfn}\label{dfn:polyfun}
A functor $F\in \Sp^{\J}$ is polynomial of degree at most $n$ if the natural map $$F(V) \to \lim_U F(U\oplus V)$$ is an equivalence, where the limit is indexed over the $\infty$-category of nonzero subspaces $U\subset \mathbb{C}^{n+1}.$
\end{dfn}

As in Goodwillie calculus, the inclusion of the full subcategory $\Poly^{\leq n}(\Sp^{\J}) \subset \Sp^{\J}$ of functors which are polynomial of degree at most $n$ admits a left adjoint $P_n: \Sp^{\J} \to \Poly^{\leq n}(\Sp^{\J}).$  The unit $\eta_n$ of this adjunction provides for each $F\in \Sp^{\J}$ a natural transformation $F \to P_nF$ which we will refer to as the \emph{degree $n$ polynomial approximation} of $F$. 

\begin{rmk}\label{rmk:infinityweiss}
This universal property was not explicitly stated in \cite{Weiss}, but it follows formally from Weiss's results as follows: the functor $P_n$ and the transformation $\eta_n$ can be defined explicitly as in \cite{Weiss} by iteratively applying the functor $\tau_n: \Sp^{\J} \to \Sp^{\J}$ defined by the formula $$\tau_n F(V) = \lim_U F(U\oplus V)$$ with the limit indexed as in Definition \ref{dfn:polyfun}.   The facts required of the functors $P_n$ in the proof of Theorem 6.1.1.10 in \cite{HA} are precisely the content of Theorem 6.3 of \cite{Weiss}.  
\end{rmk}

Given this universal property, Proposition 5.4 of \cite{Weiss} ensures the existence of a natural Taylor tower $$F \longrightarrow \cdots \longrightarrow P_{n} F \xrightarrow{p_{n-1}} P_{n-1} F \longrightarrow \cdots \longrightarrow P_0F$$ living under any functor $F\in \Sp^{\J}.$  The fiber $D_n F$ of $p_{n-1}$ has the special property that it is polynomial of degree at most $n$ and $P_{n-1} D_n F \simeq 0$.  Such a functor is called \emph{$n$-homogeneous}; such functors are completely classified by the following theorem:

\begin{thm}[{{\cite[Theorem 7.3]{Weiss}}}]
Let $F\in \Sp^{\J}$.  Then $F$ is an $n$-homogeneous functor if and only if there exists a spectrum $\Theta$ with an action of the unitary group $U(n)$ such that $$F(V) = (\Theta \wedge S^{nV})_{hU(n)}.$$
\end{thm}

%maybe mention how Arone gets his splitting and set up the notation
We will now 


%set up the monoidal structures
In order to upgrade the results of \cite{Arone} to structured multiplicative splittings, we must first understand the multiplicative properties of the polynomial approximation functors.  


\subsection{General splitting machinery}


Let $[n]$ denote the linearly ordered set of integers $0\leq i\leq n$.  Define $\Fil_n = \text{Fun}([n], \Sp^{\J})$ and $\Cofil_n = \text{Fun}([n]^{\text{op}},\Sp^{\J})$.  These categories admit functors to $\Sp^{\J}$ by taking colimit and limit, respectively.  Let $\C_n = \Fil_n \times_{\Sp^{\J}} \Cofil_n.$  Finally, let $\Gr_n = \text{Fun}([n]^{\text{ds}}, \Sp^{\J})$ where $[n]^{\text{ds}}$ denotes the underlying discrete category.  We have the following lemma:

\begin{lem}
For all integers $n>0$, there is a fully faithful functor $i_n:\Gr_{n+1} \to \C_n.$  
\end{lem}
\begin{proof}
An element of $\C_n$ is given by a sequence of functors 
\begin{center}
$X_0 \longrightarrow X_1 \longrightarrow \cdots \longrightarrow X_n \simeq Y_n \longrightarrow \cdots \longrightarrow Y_1 \longrightarrow Y_0$ 
\end{center}
where the middle 
\end{proof}

We may then take inverse limits to get a category $\C_\infty = \Fil(\Sp^{\J}) \times_{\Sp^{\J}} \Cofil(\Sp^{\J})$ and a functor $i: \Gr \to \C_\infty$. 

\begin{cor}
The functor $i$ is fully faithful.
\end{cor}
\begin{proof}
%This amounts to checking that taking inverse limits retains fully faithfulness.  I think this is totally obvious if you were taking an inverse limit of simplicial sets, but maybe we need to be a little careful since we want a homotopy limit?
\end{proof}

at the end, restrict connectivity so that it's monoidal

\section{An \texorpdfstring{$\mathbb{E}_2$}{E2}-splitting in Complex Cobordism} \label{sec:MUE2}

In this brief section, we remark that the $\mathbb{A}_\infty$-splitting $$\Sigma^{\infty}_+ \Omega SU(n) \simeq ???$$ becomes $\mathbb{E}_2$ after smashing with $MU$.  More precisely, we show that there is an equivalence of $\mathbb{E}_2$-$MU$-algebras
$$MU \smsh \Sigma^{\infty}_+ \Omega SU(n) \simeq ???.$$

The $\mathbb{A}_\infty$-$MU$-algebra equivalence constructed in Section \ref{sec:AooSplit} is realized by a map of $\mathbb{A}_\infty$-$\mathbb{S}$-algebras
\begin{equation} \label{SplittingMap}
\Sigma^{\infty} \Omega SU(n) \longrightarrow ???.
\end{equation}

Our task is to show that (\ref{SplittingMap}) may be refined to a morphism of $\mathbb{E}_2$-ring spectra.  We do so by obstruction theory--the key fact powering our proof is that 
$$MU_*\left(\Omega SU(n)\right) \cong 0$$
whenever $*=0$.  \textbf{FIND A REFERENCE}.  In fact, inspired by \cite{ChadwickMandell}, we prove the following more general result:

\begin{thm}
Suppose that $R$ is an $\mathbb{E}_2$-ring spectrum with no homotopy groups in odd degrees.  Then any $\mathbb{A}_\infty$-ring homomorphism
$$\Sigma^{\infty}_+ \Omega SU(n) \rightarrow R$$
lifts to a morphism of $\mathbb{E}_2$-ring spectra.
\end{thm}

\begin{proof} 
By taking connective covers, one learns that any $\mathbb{A}_\infty$-ring homomorphism
$$\Sigma^{\infty}_+ \Omega SU(n) \rightarrow R$$
must factor through the natural $\mathbb{E}_2$-algebra map $\tau_{\ge 0} R \rightarrow R$.  Thus, without loss of generality we will assume that $R$ is $(-1)$-connected.

It is clear that the composite $\mathbb{A}_\infty$-ring homomorphism
$$\Sigma^{\infty}_+ \Omega SU(n) \longrightarrow R \longrightarrow \tau_{\le 0} R \simeq H\pi_0(R)$$
may be lifted to an $\mathbb{E}_2$-ring homomorphism factoring through $\tau_{\le 0} \Sigma^{\infty}_+ \Omega SU(n) \simeq H\mathbb{Z}$.   Suppose now for $q>0$ that we have chosen an $\mathbb{E}_2$-ring homomorphism 
$$\Sigma^{\infty}_+ \Omega SU(n) \longrightarrow \tau_{\le q-1} R$$
lifting the given $\mathbb{A}_\infty$-algebra map
$$\Sigma^{\infty}_+ \Omega SU(n) \longrightarrow R \longrightarrow \tau_{\le q-1} R.$$
We will show that there is no obstruction to the existence of a further $\mathbb{E}_2$-lift $$\Sigma^{\infty}_+ \Omega SU(n) \longrightarrow \tau_{\le q} R.$$
According to \cite[Theorem $4.1$]{ChadwickMandell}, there is a diagram of principal fibrations
$$
\begin{tikzcd}
\mathbb{E}_2\text{-Ring}(\Sigma^{\infty}_+ \Omega SU(n), \tau_{\le q} R) \arrow{r} \arrow{d} & \mathbb{A}_\infty\text{-Ring}(\Sigma^{\infty}_+ \Omega SU(n), \tau_{\le q} R) \arrow{d} \\
\mathbb{E}_2\text{-Ring}(\Sigma^{\infty}_+ \Omega SU(n), \tau_{\le q-1} R) \arrow{r} \arrow{d} & \mathbb{A}_\infty\text{-Ring}(\Sigma^{\infty}_+ \Omega SU(n), \tau_{\le q-1} R) \arrow{d} \\
\cS_*(BSU(n),K(\pi_q R,q+3)) \arrow{r} & \cS_*(SU(n),K(\pi_q R,q+2))
\end{tikzcd}
$$
For $q$ odd, $\tau_{\le q-1} R \simeq \tau_{\le q} R$, so there is no obstruction.  Let us therefore assume that $q$ is even.

Since the cohomology of $BSU(n)$ is even-concentrated with coefficients in any abelian group, we have that $\pi_0 \cS_*(BSU(n),K(\pi_q R,q+3)) \cong H^{q+3}(BSU(n);\pi_q R)$ is zero.  It follows then that the given class $$x \in \pi_0 \mathbb{E}_2\text{-Ring}(\Sigma^{\infty}_+ \Omega SU(n), \tau_{\le q-1} R)$$ admits some lift $$\widetilde{x} \in \mathbb{E}_2\text{-Ring}(\Sigma^{\infty}_+ \Omega SU(n), \tau_{\le q} R).$$  We may need to modify $\widetilde{x}$ to match our chosen $\mathbb{A}_\infty$-ring homomorphism.  This is always possible so long as the map
$$\pi_1(\cS_*(BSU(n),K(\pi_q R,q+3))) \longrightarrow \pi_1(\cS_*(SU(n),K(\pi_q R,q+2)))$$
is surjective.  Said in other terms, this is just the map
$$H^{2q+2}(BSU(n);\pi_q R) \longrightarrow H^{2q+1}(SU(n);\pi_q R) \cong H^{2q+2}(\Sigma SU(n);\pi_q R)$$
induced by the natural map $\Sigma SU(n) \rightarrow BSU(n)$.  It is a classical fact that this map is surjective (it follows from a calculation with the bar spectral sequence, using the fact that the cohomology of $SU(n)$ is exterior). \textbf{Maybe you can check this Allen}.
\end{proof}

\section{Obstructions to a General \texorpdfstring{$\mathbb{E}_2$}{E2}-splitting}

We will now show that the $\mathbb{A}_\infty$ splitting $$\Sigma^{\infty}_+ \Omega SU(n) \simeq ???$$ cannot be promoted to an $\mathbb{E}_2$-splitting before smashing with complex-cobordism.  According to ????, such a splitting would yield an $\mathbb{E}_2$-ring homomorphism $\Sigma^{\infty}_+ \Omega SU(n) \rightarrow \Sigma^{\infty}_+ \mathbb{CP}^{n-1}$, where $\Sigma^{\infty}_+ \mathbb{CP}^{n-1}$ is given the square-zero multiplication.  Furthermore, the precomposition with the inclusion $\Sigma^{\infty}_+ \mathbb{CP}^{n-1} \longrightarrow \Sigma^{\infty}_+ \Omega SU(n)$ must yield the identity map.

Recall now that there is an adjunction \textbf{CITE}
$$\Sigma^{\infty}_+:\textbf{Double Loop Spaces} \leftrightarrows \mathbb{E}_2\text{-Rings}:GL_1.$$
Using this, we may form the adjoint double loop map
$$\Omega SU(n) \rightarrow GL_1(\Sigma^{\infty}_+ \mathbb{CP}^{n-1}).$$

\textbf{I'll wait to write this until we work out the previous sections, since I think it will be helpful to reference abstract nonsense about filtered stuff we prove earlier.}

Suppose there is a map $\Sigma^{\infty}_+ \Omega SU(n) \rightarrow \Sigma^{\infty}_+ \mathbb{CP}^{n-1}$.  This is adjoint to a double loop map $\Omega SU(n) \rightarrow GL_1(\Sigma^{\infty}_+\mathbb{CP}^{n-1})$ which lands in the component $SL_1(\Sigma^{\infty}_+ \mathbb{CP}^{n-1}) \simeq \Omega^2 \Sigma^2 \mathbb{CP}^{n-1}$.  BLAH BLAH
%maybe remark that BU -> QCP^\infty not being E_2 is still true after taking \Sigma^\infty_+, and remark that this is like a version of the splitting principle.

\section{Snaith's Construction of Periodic Complex Bordism} \label{sec:SnaithSplitting}

A classical theorem of Snaith \cite{SnaithOriginal} gives an equivalence of homotopy commutative ring spectra $$\Sigma^{\infty}_+ BU [\beta^{-1}] \simeq MUP.$$  The equivalence arises from considering the total $MU$-Chern class map $BU \to GL_1(MUP).$  It is known from \cite{SnaithNotMultiplicative} that the total Chern class in integral homology is not an infinite loop map.  It follows from the existence of an $\E_\infty$ map $MUP \to H\Z P$ from periodic complex bordism to periodic integral homology that Snaith's equivalence is not an equivalence of $\E_\infty$ ring spectra.  The following theorem refines this observation:


%should check if the obstruction in SnaithNotMultiplicative is also E_3...I would think it is
%should we also cite the totaro paper that does literally the same thing as SnaithNotMultiplicative?

\begin{thm}
The equivalence $\Sigma^{\infty}_+ BU [\beta^{-1}] \simeq MUP$ is $\mathbb{E}_2$ but not $\mathbb{E}_3$.
\end{thm}

\begin{proof}
Proof goes here
\end{proof}

Comment now about GepnerSnaith.
We should cite at some point here or the introduction all of \cite{SnaithNotMultiplicative},  \cite{GepnerSnaith}, and the Snaith book with the original splitting.

\section{Miscellaneous stuff here}

It would be nice to at some point deal with showing the associated graded $E_2$ structure of BU is the thom spectrum VMU(n).  I've directly pasted in some writing from a previous argument I claimed, but it definitely uses that $\coprod BU(n)$ is an $E_2$ algebra over $\Z _{\geq 0}$ which I never got straight an actual proof of.  

\begin{prop}The associated graded of $\Sigma^{\infty}_+BU$ is $E_2$ equivalent to the Thom spectrum $\bigvee MU(n).$
\end{prop}
\begin{proof}
Let $R = BU$ with its natural filtration, and let $R^{\oplus} = \coprod BU(n)$ with its natural filtration.  Let $M$ be the ($E_\infty$) filtered spectrum which is $MU(n)$ in degree $n$, and all maps are $0$.  In other words, $\bigvee MU(n)$ with its natural filtration is $I(res(M))$.  

We begin with a filtered $E_\infty$ map $z:R^\oplus \to I(res(M))$ coming from the zero section.  Then, $R^\oplus$ comes with the structure of an $E_2$ algebra over $\Z_{\geq 0}^{fil}$.  In fact, $I(res(M))$  has a trivial structure as an $E_\infty$-algebra over $\Z_{\geq 0}^{fil}$ via the augmentation $\Z_{\geq 0}^{fil}\to S^{0,fil} \to I(res(M))$.  We may then tensor $z$ along the augmentation to get a map of $E_2$ filtered spectra $z':R \to I(res(M))$.  

There is a canonical equivalence $I(res(M)) \otimes \mathbb{A} \simeq M$ because $M$ is in the image of $\mathbb{A} \otimes I(-)$ (that is, all the maps in the filtration of $M$ were zero).  As such, $M$ acquires a canonical structure as an $\mathbb{A}$ algebra such that the map $M \otimes \mathbb{A} \to M$ is a map of $E_2$ rings (in fact I think it's $E_\infty$?).  

Finally, we observe that we may tensor $z'$ with $\mathbb{A}$ and compose with the multiplication map to get an $E_2$ map $R\otimes A \to M \otimes A \to M$ which is the right thing up to homotopy, so it's an equivalence.  
\end{proof}

******  

What is $\Sigma^{\infty}_+ \Omega SU(n)[\beta^{-1}]$, by the way?  Is it related to a periodic version of the $X(n)$-filtration of $MU$??

\bibliographystyle{amsalpha}
\bibliography{Bibliography}

\end{document}